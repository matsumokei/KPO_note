\section{5,6, September, 2022}
SSDMに向けてスライドを作成,そして発表練習を行った.
以下の質問を頂いた.\\
・mollow triplet の実験は他の系で試されているか?\\
・我々の手法のエラーのオリジンはなにか?ー解答は,quantum effective Hamiltonian に近似したときに発生する.





{\LARGE{
\section{原稿}
\subsection{page 0}
Hi everyone! This is Keisuke Matsumoto.

Today, I am talking about this title.

%this work was done in collaboration between tokyo univercity of science and Aist and NEC.

Hereafter, I call A as KPO.

\subsection{QUantum adiabatic evolution}
First, I introduce about KPO.
First we start from initial Hmiltonian like this. then the initial ground state is vacuum state.
By performing adiabatic evolution with increasing $\beta$.

through a squeezed state, finally one can arrive at a cat state. the cat state is the ground state of final Hamiltonian.

% we start from vacuum states, and we gradually increase parametric driving terms with $\beta$ in an adiabatic way. Then, we get a cat state as a final state. A cat state is a super position of two coherent states.
Here, $\ket{\alpha}$ is corresponding to up spin, and $\ket{-\alpha}$ is corresponding to down spin.

\subsection{page 1}


Next, I briefly introduce about Quantum annealing with KPOs and our motivation to estimat the photon number of KPOs.

% Recently, Quantum annealing with KPOs was proposed by these people

 Ising Hamiltonian can be mapped to N body KPO. 
 Therefore, one can perform quantum annealing with KPO.
 
Adiabatic increasing $\beta$, a ground state of N body KPO is tensor product of the coherent state 

So the final energy is given by this.

From this equation We find for accurately mapping the Ising Hamiltonian to the KPO Hamiltonian, we need to precisely control and estimate the photon number of KPO.

\subsection{page 2}

To address this problem, We propose a experimentally feasible method to estimate the mean photon number of KPO $\alpha$ squared by using spectroscopic measurement.

In conventional method, one use a formula as follows to calculate the photon number of KPO.

But this value can be different from this actual vale.

On the other, In our method, we consider a KPO coupled with an ancillary qubit system as this schematic.

%This schematic shows our system. 
Here we use Transmon as an ancillary qubit.

Our method allow us to estimate the photon number from peak or dip positions of the spectra of the qubit.

We find our method is more accurate than the conventional method from a numerical simulation.



\subsection{page 3}
Next, I introducce our model Hamiltonian.
Our model is KPO coupled with ancillary qubit system.


This total Hamiltonian is this one which move into a rotating frame at the frequency
of $\omega_{\rm{p}}/2$ and adapt the rotating wave approximation.

1st term is KPO Hamiltonian

2nd term is qubit Hamiltonian

3rd term is coupling interaction between KPO and qubit

finally $H_D$ is coherent drive for Qubit



\subsection{page 4}
Fouthermore, Let us consider our model with decoherence.

To consider the effect of decoherence such as a single photon disspation of the KPO and a Spontaneous emission of the qubit, we use the followng GKSL type master equation.
In this master equation, 2nd term is a single photon disspation and 3rd term is a spontaneous emission.


\subsection{page 5}
For a large $\beta$ the ground state of KPO  become a superposition of two coherent state.

Then, a Qubit Effective Hamiltonian is approximately wrriten as this, where $H_I$ is rewrriten as this.

the information of the photon number of KPO can be encoded to this interaction.

Let us consider this Hamiltonian, We can observe a Mollow triplet via spectroscopic measurement.

This is an example of Mollow triplet, then, two peaks at $\pm2g\alpha$ are important.

Because we can estimte the value of $\alpha$ from the peak positions in the Mollow triplet






\subsection{page 6,7}
Next, I explain our protocol.

Step1, For a large $\beta$, we prepare a steady state without coherent drive as an initial state.

Step2, we perform time evolution with coherent drive.
when the sysytem is steady, then, we measure the population of the qubit as this.

Step3, we calculate the time integrated spectra of measured population.

Finally, we estimate the photon number of KPO from the peak positions of the spectra.
This shows a example of the spectram.

We can find two dips, then we can obtain the frequency difference of two dips corresponding to $4g\alpha$.

Therefore, we can estimate the photon number of KPO $|\alpha|^2$ as this equation.



\subsection{page 8}
We show our numerical results.

In this slide, we discuss the relation between spectra and an energy diagram.

our numerical result gives these spectra, where we set these parameters.
and By diagonalizing this Hamiltonian, we obtain the energy diagram.

We find two dips.

As shown spectra and energy diagram, two dips are corresponding to these transition.

Next, We find another peak.
this peak correspond to this transition.




\subsection{page 9}
we evaluate the performance of our method by comparing with the conventional method.


This spectra is same as one before slide

In our method, we can estimate the photon number  from the frequency difference of two dips.

In conventional method, using the analytical formula, one can estimate the photon number.

Here, we define exact photon number as this, where $\rho_{\rm{ss}}$ is asteady state of this master equation.

Also, we define relative error between our method and exact photon number.

the conventional method is similar.

This table shows our method is more accurate than the conventional method.



\subsection{page 10}
Furthermore, we evaluate the performance of our method with other parameters.

We plot these relative error against the detuning of KPO.

red plot is our method. blue plot is the conventional method.

This figure shows our method is more accurate than the conventional method.




Finaly, My presentation is closed by summary.

\section{Appendix}
By diagonalizing the Hamiltonian, we recognized that
we have a transition from the second excited state to the
third excited state with an energy difference of $2π × 0.5$
MHz. However, we cannot resolve this peak in the numerical simulation possibly due to the large width of the
peak at $\Delta_{\rm{q}}^{(0)}$.


It
is worth mentioning that, in the original proposal of QA
with KPO [1], KPO has a finite detuning during QA.
Therefore, our scheme is useful for such circumstances.
}}

\section{}
Quantum Singular Value Transform (QSVT)は,任意の行列の特異値に対して,多項式変換を行う手法である.多項式変換から,Taylor展開を使い任意の関数変換を生成することができるため,QSVTはとても強力である.QSVT protocolは,より制約の多い量子信号処理(QSP, Quantum signal processing)protocolの拡張である.QSPは,単一量子ビットユニタリー演算子の行列要素を多項式変換する方法である.QSVT-protocolは洗練されているが,そのコアとなる考え方は非常にシンプルである.


QSPはある制約を満たす,区間$[-1, 1]$上の任意の関数を近似するとても強力なtoolである.まずいくつかの直感的な例を見てみる.以下の$a\in[-1,1]$によって,パラメータ化されたsingle量子ビット演算子を考える:
\begin{equation}
    \hat{W}(\alpha)=\left(
        \begin{array}{cc}
       a&i\sqrt{1-a^2}\\[10pt]
        i\sqrt{1-a^2}&a \\
        \end{array}
        \right),
\end{equation}
ここで,$\hat{W}(a)$はsignal rotation operator (SRO)と呼ばれている.この演算子を用いることで,signal procesing operator (SPO)と呼ばれる別の演算子を定義することができる:
\begin{equation}
    \hat{U}_{\mathrm{sp}} = \hat{R}_z(\phi_0)\ \prod_{k=1}^{d}\hat{W}(a)\hat{R}_z(\phi_k).
\end{equation}
SPO $\hat{U}_{\mathrm{sp}}$はベクトル$\vec{\phi}=(\phi_0,\phi_1,\ldots,\phi_k,\ldots,\phi_d)\in\mathbb{}{R}^{d+1}$によって,パラメータ化されている.ここで,$d$はSRO $\hat{W}(a)$を作用する繰り返し数を表しており,我々が自由に決めることができるパラメータである.


例として$d=2$, $\vec{\phi}=(0,0,0)$の場合を考えてみよう.このとき,行列要素$\braket{0|\hat{U}_{\mathrm{sp}}|0}$を計算すると,
\begin{align}
    \braket{0|\hat{U}_{\mathrm{sp}}|0} 
    &= \braket{0|\hat{R}_z(0)\ \prod_{k=1}^{2}\hat{W}(a)\hat{R}_z(0)|0}\nn[10pt]
    &=\braket{0|\hat{W}^2(a)|0}\nn[10pt]
    &=\braket{0|
    \left(
        \begin{array}{cc}
       a&i\sqrt{1-a^2}\\[10pt]
        i\sqrt{1-a^2}&a \\
        \end{array}
        \right)
    \left(
        \begin{array}{cc}
       a&i\sqrt{1-a^2}\\[10pt]
        i\sqrt{1-a^2}&a \\
        \end{array}
        \right)
    |0}\nn[10pt]
    %
    &=\braket{0|
    \left(
        \begin{array}{cc}
       2a^2-1&2a i\sqrt{1-a^2}\\[10pt]
        2a i\sqrt{1-a^2} & 2a^2-1\\
        \end{array}
        \right)
    |0}\nn[10pt]
    \therefore  \braket{0|\hat{U}_{\mathrm{sp}}|0} & = 2a^2-1
\end{align}
このとき,行列要素は$a$の多項式となることがわかる. 同様に,写像 $\mathcal{S} : a \to 2 a^ 2 - 1$を表現したいとすると,この期待値を求めることが,そのような写像を与える手段となる.この処理は写像 $\mathcal{S} : a \to \mathrm{poly}(a)$を生成するために一般化できることが分かる.次の定理はその方法を示している.

\section{}
\begin{theorem}[Quantum signal processing]\label{thm:qsp}
There exists a set of phase factors $\Phi := (\phi_0, \cdots, \phi_d) \in \mathbb{R}^{d+1}$ such that
\begin{equation}
  U_{\Phi}(x) = e^{i \phi_0 Z} \prod_{j=1}^{d} \left[ O(x) e^{\I \phi_j Z} \right] = \left( \begin{array}{cc}
    P(x) & -Q(x) \sqrt{1 - x^2}\\
    Q^*(x) \sqrt{1 - x^2} & P^*(x)
  \end{array} \right)
  \label{eqn:QSP_representation}
\end{equation}
if and only if $P,Q\in \mathbb{C}[x]$ satisfy
\begin{itemize}
     \item $\deg(P) \leq d, \deg(Q) \leq d-1$,
    \item $P$ has parity $d \bmod 2$ and $Q$ has parity $d-1 \bmod 2$, and
    \item $|P(x)|^2 + (1-x^2) |Q(x)|^2 = 1, \forall x \in [-1, 1]$.
\end{itemize}
Here $\deg Q=-1$ means $Q=0$.
\end{theorem}



