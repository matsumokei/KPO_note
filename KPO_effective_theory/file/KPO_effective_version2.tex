\section{高次の摂動の計算について}
今,KPOの光子数固有状態について,$\ket{0}$,$\ket{4}$が縮退している場合を考える.また,高励起エネルギー固有状態は考えないとする.この場合のHamiltonianは,
\begin{align}
    \hat{H} &= \hat{H}_0 + \hat{H}^{\prime}\\[10pt]
    \hat{H}_0 &= \Delta \ket{2}\bra{2}\nn[10pt]
    \hat{H}^{\prime} &
    = p(\ket{0}\bra{2} + \ket{4}\bra{2} + \ket{2}\bra{0} + \ket{2}\bra{4})
    = p(\hat{A} + \hat{A}^{\dagger})
\end{align}
ここで,$\hat{A} = \ket{0}\bra{2} + \ket{4}\bra{2}$である.Hamiltonianを$\ket{0}, \ket{2}, \ket{4}$の基底を使って,行列表示すると,
\begin{align}
    \hat{H}=
    \left(
    \begin{array}{ccc}
    0 & p & 0 \\[5pt]
    p & \Delta & p \\[5pt]
    0 & p & 0
    \end{array}
    \right)
\end{align}
このHamiltonianを対角化すると



\subsection{KPOに関する摂動論}
偶数番目のFock状態で表現すれば
\begin{align}
     \hat{H}_{\rm{KPO}}=
   \bordermatrix{     
    & \bra{0} &  \bra{2} &  \bra{4}&  \bra{6}&  \bra{8}& \cdots& \cdots \cr
   \ket{0}&E_0&\sqrt{2\cdot1}\beta&0&0&0&\cdots& \cdots\cr
  \ket{2}&\sqrt{2\cdot1}\beta&E_2&\sqrt{4\cdot3}\beta&0&0& \cdots& \cdots\cr
  \ket{4}&0&\sqrt{4\cdot3}\beta&E_4&\sqrt{6\cdot5}\beta&0& \cdots& \cdots\cr
  \ket{6}&0&0&\sqrt{6\cdot5}\beta&E_6&\sqrt{8\cdot7}\beta& \cdots& \cdots\cr
  \ket{8}&0&0&0&\sqrt{8\cdot7}\beta&E_8& \cdots& \cdots\cr
  \vdots&\vdots & \vdots &\vdots  & \vdots & \dots& \ddots & \vdots \cr
  \vdots&\dots & \dots&\dots &\dots & \dots& \dots  & \vdots\cr
            }
\end{align}
まず,$n=0$と$n=8$が縮退している場合を考えよう.この場合のHamiltonianは以下のように書ける:
\begin{align}
     \hat{H}_{\rm{KPO}}&=
   \bordermatrix{     
    & \bra{0} &  \bra{2} &  \bra{4}&  \bra{6}&  \bra{8}\cr
   \ket{0}&E_0&\sqrt{2\cdot1}\beta&0&0&0\cr
  \ket{2}&\sqrt{2\cdot1}\beta&E_2&\sqrt{4\cdot3}\beta&0&0\cr
  \ket{4}&0&\sqrt{4\cdot3}\beta&E_4&\sqrt{6\cdot5}\beta&0\cr
  \ket{6}&0&0&\sqrt{6\cdot5}\beta&E_6&\sqrt{8\cdot7}\beta\cr
  \ket{8}&0&0&0&\sqrt{8\cdot7}\beta&E_8\cr
            }\\[10pt]
    &=
   \bordermatrix{     
    & \bra{0} &  \bra{2} &  \bra{4}&  \bra{6}&  \bra{8}\cr
   \ket{0}&0&\sqrt{2\cdot1}\beta&0&0&0\cr
  \ket{2}&\sqrt{2\cdot1}\beta&\Delta_1&\sqrt{4\cdot3}\beta&0&0\cr
  \ket{4}&0&\sqrt{4\cdot3}\beta&\Delta_2&\sqrt{6\cdot5}\beta&0\cr
  \ket{6}&0&0&\sqrt{6\cdot5}\beta&\Delta_1&\sqrt{8\cdot7}\beta\cr
  \ket{8}&0&0&0&\sqrt{8\cdot7}\beta&0\cr
            }
\end{align}
ここで,$E_0=E_8=0$, $E_2=E_6=\Delta_1$, $E_4=\Delta_2$である.


\section*{\textcolor{red}{Step1 : $n=2,4,6$の部分空間での厳密解}}
\begin{align}
    \hat{H}=
    \left(
    \begin{array}{ccc}
    \Delta_1 & p_1 & 0 \\[5pt]
    p_1 & \Delta_2 & p_2 \\[5pt]
    0 & p_2 & \Delta_1
    \end{array}
    \right)
\end{align}


エネルギー固有値
\begin{align}
    E_D&=
    {\Delta_{1}}\\[10pt]
    %
    E_B&=
    \frac{1}{2}\left({\Delta_{1}}+{\Delta_{2}}-\sqrt{{\Delta_{1}}^2-2{\Delta_{1}}{\Delta_{2}}+{\Delta_{2}}^2+8p^2}\right)\nn[10pt]
    &=\frac{1}{2}\left({\Delta_{1}}+{\Delta_{2}}-\sqrt{{
    (\Delta_{1}}-{\Delta_{2}})^2+8p^2}\right)\\[10pt]
    %%
    E_4=&\frac{1}{2}\left({\Delta_{1}}+{\Delta_{2}}+\sqrt{{\Delta_{1}}^2-2{\Delta_{1}}{\Delta_{2}}+{\Delta_{2}}^2+8p^2}\right)
    \nn[10pt]
    &=\frac{1}{2}\left({\Delta_{1}}+{\Delta_{2}}
    +\sqrt{{
    (\Delta_{1}}-{\Delta_{2}})^2+8p^2}\right)
    \end{align}


固有状態$\ket{2}, \ket{4}, \ket{6}$で展開する.
\begin{align}
    \ket{\psi_D}&=C_0\left\{-\frac{{p_{2}}}{{p_{1}}},0,1\right\}\\[10pt]
    \ket{\psi_B}&=C_1\left\{\frac{{p_{1}}}{{p_{2}}},-\frac{{\Delta_{1}}-{\Delta_{2}}+\sqrt{{\Delta_{1}}^2-2{\Delta_{1}}{\Delta_{2}}+{\Delta_{2}}^2+8p^2}}{2{p_{2}}},1\right\}\nn[10pt]
    &=C_1\left\{\frac{{p_{1}}}{{p_{2}}},-\frac{{\Delta_{1}}-{\Delta_{2}}
    +\sqrt{({\Delta_{1}}-{\Delta_{2}})^2+8p^2}}{2{p_{2}}},1\right\}\nn[10pt]
    &=C_1\left\{\frac{{p_{1}}}{{p_{2}}},
    \epsilon_+({\Delta_{1}},{\Delta_{2}},
    p_1,p_2)
    ,1\right\}\\[10pt]
    %
    %
    \ket{\psi_4}&=C_2\left\{\frac{{p_{1}}}{{p_{2}}},-\frac{{\Delta_{1}}-{\Delta_{2}}-\sqrt{{\Delta_{1}}^2-2{\Delta_{1}}{\Delta_{2}}+{\Delta_{2}}^2+8p^2}}{2{p_{2}}},1\right\}\nn[10pt]
    &=C_2\left\{\frac{{p_{1}}}{{p_{2}}},
    \epsilon_-({\Delta_{1}},{\Delta_{2}},
    p_1,p_2)
    ,1\right\}
\end{align}

\begin{align}
    \epsilon_{\textcolor{red}{\pm}}({\Delta_{1}},{\Delta_{2}},
    p_1,p_2)
    =-\frac{{\Delta_{1}}-{\Delta_{2}}
    \textcolor{red}{\pm}\sqrt{({\Delta_{1}}-{\Delta_{2}})^2+8p^2}}{2{p_{2}}}
\end{align}


規格化
\begin{equation}
        \ket{D}\equiv\ket{\psi_0}=\left(
        \begin{array}{c}
       -\frac{{p_2}}{{p_1}\sqrt{\left|\frac{{p_2}}{{p_1}}\right|^2+1}}\\[20pt]
       0\\[10pt]
       \frac{1}{\sqrt{\left|\frac{{p_2}}{{p_1}}\right|^2+1}}\\[10pt]
        \end{array}
        \right)
\end{equation}

\begin{equation}
     \ket{B}\equiv\ket{\psi_1}=\left(
        \begin{array}{c}
       \frac{{p_1}}{{p_2}\sqrt{
       |\epsilon_{+}|^2
       +\left|\frac{{p_1}}{{p_2}}\right|^2+1}}\\[20pt]
       %
       %
       -\frac{{\Delta_1}-{\Delta_2}+\sqrt{{\Delta_1}^2-2{\Delta_1}{\Delta_2}+{\Delta_2}^2+4{p_1}^2+4{p_2}^2}}{2{p_2}\sqrt{
       |\epsilon_{+}|^2
       +\left|\frac{{p_1}}{{p_2}}\right|^2+1}}\\[20pt]
       %
       %
       \frac{1}{\sqrt{
       |\epsilon_{+}|^2
       +\left|\frac{{p_1}}{{p_2}}\right|^2+1}}
        \end{array}
        \right)
        %%%%%%%%%%%%%%%%%%%%%%%%%%%%
        %%%%%%%%%%%%%%%%%%%%%%%%%%%%
        =\left(
        \begin{array}{c}
       \frac{{p_1}}{{p_2}\sqrt{
       |\epsilon_{+}|^2
       +\left|\frac{{p_1}}{{p_2}}\right|^2+1}}\\[20pt]
       %
       %
       \frac{\epsilon_+}{
       \sqrt{
       |\epsilon_{+}|^2
       +\left|\frac{{p_1}}{{p_2}}\right|^2+1}}\\[20pt]
       %
       %
       \frac{1}{\sqrt{
       |\epsilon_{+}|^2
       +\left|\frac{{p_1}}{{p_2}}\right|^2+1}}
        \end{array}
        \right)
\end{equation}

\begin{equation}
     \ket{4}\equiv\ket{\psi_2}=\left(
        \begin{array}{c}
       \frac{{p_1}}{{p_2}\sqrt{
       |\epsilon_{-}|^2
       +\left|\frac{{p_1}}{{p_2}}\right|^2+1}}\\[20pt]
       %
       %
       -\frac{{\Delta_1}-{\Delta_2}-\sqrt{{\Delta_1}^2-2{\Delta_1}{\Delta_2}+{\Delta_2}^2+4{p_1}^2+4{p_2}^2}}{2{p_2}\sqrt{
       |\epsilon_{-}|^2
       +\left|\frac{{p_1}}{{p_2}}\right|^2+1}}\\[20pt]
       %
       %
       \frac{1}{\sqrt{
       |\epsilon_{-}|^2
       +\left|\frac{{p_1}}{{p_2}}\right|^2+1}}\\[10pt]
        \end{array}
        \right)
    %%%%%%%%%%%%%%%%%%%%%%%%%%%
    %%%%%%%%%%%%%%%%%%%%%%%%%%
    =\left(
        \begin{array}{c}
       \frac{{p_1}}{{p_2}\sqrt{
       |\epsilon_{-}|^2
       +\left|\frac{{p_1}}{{p_2}}\right|^2+1}}\\[20pt]
       %
       %
       \frac{
       \epsilon_-
       }{\sqrt{
       |\epsilon_{-}|^2
       +\left|\frac{{p_1}}{{p_2}}\right|^2+1}}\\[20pt]
       %
       %
       \frac{1}{\sqrt{
       |\epsilon_{-}|^2
       +\left|\frac{{p_1}}{{p_2}}\right|^2+1}}\\[10pt]
        \end{array}
        \right)
\end{equation}

ここで,パラメトリックドライブの振幅$\beta$が十分小さい場合,

$1/\sqrt{2} \simeq 0.70710678118$

\begin{align}
    \hat{H}_0 &= E_{0}\ket{0}\bra{0} + E_8 \ket{8}\bra{8} + E_{D} \ket{D}\bra{D}+ E_{B} \ket{B}\bra{B}\\[10pt]
    \hat{V}&= \sqrt{2\cdot1}\beta\ket{0}\bra{2} + \sqrt{8\cdot7}\beta\ket{8}\bra{6} + {\rm{h.c.}}
\end{align}


\begin{align}
     \hat{H}_{\rm{KPO}}
    &=
   \bordermatrix{     
    & \bra{0} &  \bra{2} &  \bra{6}&  \bra{8}\cr
   \ket{0}&0&\sqrt{2\cdot1}\beta&0&0\cr
  \ket{2}&\sqrt{2\cdot1}\beta&\Delta_1&0&0\cr
  \ket{6}&0&0&\Delta_1&\sqrt{8\cdot7}\beta\cr
  \ket{8}&0&0&\sqrt{8\cdot7}\beta&0\cr
            }
\end{align}
次の状態を定義する:
\begin{align}
    \ket{D} &= \frac{1}{\sqrt{2}}(-\ket{2}+\ket{6})\\[10pt]
    \ket{B} &= \frac{1}{\sqrt{2}}(\ket{2}+\ket{6})
\end{align}
摂動Hamiltonianを基底$\{\ket{0},\ket{8},\ket{D},\ket{B}\}$で展開する:
\begin{align}
     \hat{V}
    &=
   \bordermatrix{     
    & \bra{0} &  \bra{8} &  \bra{D}&  \bra{B}\cr
   \ket{0}&\langle{0|\hat{V}|0}\rangle&\langle{0|\hat{V}|8}\rangle&\langle{0|\hat{V}|D}\rangle
   &\langle{0|\hat{V}|B}\rangle\cr
  \ket{8}&\langle{8|\hat{V}|0}\rangle&\langle{8|\hat{V}|8}\rangle&\langle{8|\hat{V}|D}\rangle
   &\langle{8|\hat{V}|B}\rangle\cr
  \ket{D}&\langle{D|\hat{V}|0}\rangle&\langle{D|\hat{V}|8}\rangle&\langle{D|\hat{V}|D}\rangle
   &\langle{D|\hat{V}|B}\rangle\cr
  \ket{B}&\langle{B|\hat{V}|0}\rangle&\langle{B|\hat{V}|8}\rangle&\langle{B|\hat{V}|D}\rangle
   &\langle{B|\hat{V}|B}\rangle\cr
    }\\[10pt]
    &=
   \bordermatrix{     
    & \bra{0} &  \bra{8} &  \bra{D}&  \bra{B}\cr
   \ket{0}&0&0&-\sqrt{2}\beta/\sqrt{2}&\sqrt{2}\beta/\sqrt{2}\cr
  \ket{8}&0&0&\sqrt{8\cdot7}\beta/\sqrt{2}&\sqrt{8\cdot7}\beta/\sqrt{2}\cr
  \ket{D}&-\sqrt{2}\beta/\sqrt{2}&\sqrt{2}\beta/\sqrt{2}&0&0\cr
  \ket{B}&\sqrt{8\cdot7}\beta/\sqrt{2}&\sqrt{8\cdot7}\beta/\sqrt{2}&0&0&\cr
            }
\end{align}

\begin{align}
    \langle{0|\hat{V}|D}\rangle&=\langle{D|\hat{V}|0}\rangle=-\sqrt{2}\beta/\sqrt{2}\\[10pt]
    \langle{0|\hat{V}|B}\rangle&=\langle{B|\hat{V}|0}\rangle=\sqrt{2}\beta/\sqrt{2}\\[10pt]
    \langle{8|\hat{V}|D}\rangle&=\langle{D|\hat{V}|8}\rangle=\sqrt{8\cdot7}\beta/\sqrt{2}\\[10pt]
   \langle{8|\hat{V}|B}\rangle&=\langle{B|\hat{V}|8}\rangle=\sqrt{8\cdot7}\beta/\sqrt{2}
\end{align}


\begin{align}
    \ket{\varphi_{n,0}^{(1)}}
    &=
    \sum_{\textcolor{red}{\gamma\neq0}}|\varphi^{(0)}_n;8\rangle\rangle
    \textcolor{red}{\frac{1}{(E^{(2)}_{n,0}-E^{(2)}_{n,8})}
    \sum_{m\neq n}\Biggl[\sum_{p\neq n}
    \frac{\langle\braket{\varphi^{(0)}_{n};\gamma|\hat{V}|\varphi^{(0)}_m}
    \langle{\varphi^{(0)}_{m}|\hat{V}|\varphi^{(0)}_p}\rangle
    \braket{\varphi^{(0)}_{p}|\hat{V}|\varphi^{(0)}_n;0}\rangle}
    {(\epsilon_n-\epsilon_m)(\epsilon_n-\epsilon_p)}}\nn[10pt]
    %
    &\hspace{100pt}
    \textcolor{red}{-E^{(1)}_n
    \frac{\langle\braket{\varphi^{(0)}_{n};\gamma|\hat{V}|\varphi^{(0)}_m}
    \braket{\varphi^{(0)}_{m}|\hat{V}|\varphi^{(0)}_n;0}\rangle}{(\epsilon_n-\epsilon_m)^2}
    \Biggr]}
    \nn[10pt]
    %
    &\hspace{200pt}+
    \sum_{m\neq n}\ket{\varphi^{(0)}_m}
    \textcolor{blue}{
    \frac{\braket{\varphi^{(0)}_{m}|\hat{V}|\varphi^{(0)}_n;0}\rangle}{(\epsilon_n-\epsilon_m)}}\nn[10pt]
    %%%%%%%%%%%%%%%%%%%%
    %%%%%%%%%%%%%%%%%%%%
    &=
    |\varphi^{(0)}_n;8\rangle\rangle
    \textcolor{red}{\frac{1}{(E^{(2)}_{n,0}-E^{(2)}_{n,8})}
    \sum_{m\neq n}\Biggl[\sum_{p\neq n}
    \frac{\langle\braket{\varphi^{(0)}_{n};8|\hat{V}|\varphi^{(0)}_m}
    \langle{\varphi^{(0)}_{m}|\hat{V}|\varphi^{(0)}_p}\rangle
    \braket{\varphi^{(0)}_{p}|\hat{V}|\varphi^{(0)}_n;0}\rangle}
    {(\epsilon_n-\epsilon_m)(\epsilon_n-\epsilon_p)}}\nn[10pt]
    %
    &\hspace{100pt}
    \textcolor{red}{-E^{(1)}_n
    \frac{\langle\braket{\varphi^{(0)}_{n};8|\hat{V}|\varphi^{(0)}_m}
    \braket{\varphi^{(0)}_{m}|\hat{V}|\varphi^{(0)}_n;0}\rangle}{(\epsilon_n-\epsilon_m)^2}
    \Biggr]}
    \nn[10pt]
    %
    &\hspace{200pt}+
    \sum_{m\neq n}\ket{\varphi^{(0)}_m}
    \textcolor{blue}{
    \frac{\braket{\varphi^{(0)}_{m}|\hat{V}|\varphi^{(0)}_n;0}\rangle}{(\epsilon_n-\epsilon_m)}}\nn[10pt]
    %%%%%%%%%%%%%%%%%%%%
    %%%%%%%%%%%%%%%%%%%%
    &=
    |\varphi^{(0)}_n;8\rangle\rangle
    \textcolor{red}{\frac{1}{(E^{(2)}_{n,0}-E^{(2)}_{n,8})}
    \sum_{m\neq n}\Biggl[
    \frac{\langle\braket{\varphi^{(0)}_{n};8|\hat{V}|\varphi^{(0)}_m}
    \langle{\varphi^{(0)}_{m}|\hat{V}|\varphi^{(0)}_D}\rangle
    \braket{\varphi^{(0)}_{D}|\hat{V}|\varphi^{(0)}_n;0}\rangle}
    {(\epsilon_n-\epsilon_m)(\epsilon_n-\epsilon_D)}}\nn[10pt]
    %
    &\hspace{100pt}
    +{\frac{\langle\braket{\varphi^{(0)}_{n};8|\hat{V}|\varphi^{(0)}_m}
    \langle{\varphi^{(0)}_{m}|\hat{V}|\varphi^{(0)}_B}\rangle
    \braket{B|\hat{V}|\varphi^{(0)}_n;0}\rangle}
    {(\epsilon_n-\epsilon_m)(\epsilon_n-\epsilon_B)}}\nn[10pt]
    &\hspace{100pt}
    \textcolor{red}{-E^{(1)}_n
    \frac{\langle\braket{\varphi^{(0)}_{n};8|\hat{V}|\varphi^{(0)}_m}
    \braket{\varphi^{(0)}_{m}|\hat{V}|\varphi^{(0)}_n;0}\rangle}{(\epsilon_n-\epsilon_m)^2}
    \Biggr]}
    \nn[10pt]
    %
    &\hspace{200pt}+
    \sum_{m\neq n}\ket{\varphi^{(0)}_m}
    \textcolor{blue}{
    \frac{\braket{\varphi^{(0)}_{m}|\hat{V}|\varphi^{(0)}_n;0}\rangle}{(\epsilon_n-\epsilon_m)}}
\end{align}



\begin{align}
    \ket{\varphi_{n,0}^{(1)}}
    &=
    |\varphi^{(0)}_n;8\rangle\rangle
    \textcolor{red}{\frac{1}{(E^{(2)}_{n,0}-E^{(2)}_{n,8})}}\nn[10pt]
    &\times
    \textcolor{red}{\Biggl[\Biggl\{
    \frac{\langle\braket{\varphi^{(0)}_{n};8|\hat{V}|\varphi^{(0)}_D}
    \langle{\varphi^{(0)}_{D}|\hat{V}|\varphi^{(0)}_D}\rangle
    \braket{\varphi^{(0)}_{D}|\hat{V}|\varphi^{(0)}_n;0}\rangle}
    {(\epsilon_n-\epsilon_D)(\epsilon_n-\epsilon_D)}}\nn[10pt]
    %
    &\hspace{100pt}
    +{\frac{\langle\braket{\varphi^{(0)}_{n};8|\hat{V}|\varphi^{(0)}_D}
    \langle{\varphi^{(0)}_{D}|\hat{V}|\varphi^{(0)}_B}\rangle
    \braket{\varphi^{(0)}_{B}|\hat{V}|\varphi^{(0)}_n;0}\rangle}
    {(\epsilon_n-\epsilon_D)(\epsilon_n-\epsilon_B)}}\nn[10pt]
    &\hspace{100pt}
    \textcolor{red}{-E^{(1)}_n
    \frac{\langle\braket{\varphi^{(0)}_{n};8|\hat{V}|\varphi^{(0)}_D}
    \braket{\varphi^{(0)}_{D}|\hat{V}|\varphi^{(0)}_n;0}\rangle}{(\epsilon_n-\epsilon_D)^2}
    \Biggr\}}
    \nn[10pt]
    %
    %
    &+\textcolor{red}{\Biggl\{
    \frac{\langle\braket{\varphi^{(0)}_{n};8|\hat{V}|\varphi^{(0)}_B}
    \langle{\varphi^{(0)}_{B}|\hat{V}|\varphi^{(0)}_D}\rangle
    \braket{\varphi^{(0)}_{D}|\hat{V}|\varphi^{(0)}_n;0}\rangle}
    {(\epsilon_n-\epsilon_B)(\epsilon_n-\epsilon_D)}}\nn[10pt]
    %
    &\hspace{100pt}
    +{\frac{\langle\braket{\varphi^{(0)}_{n};8|\hat{V}|\varphi^{(0)}_B}
    \langle{\varphi^{(0)}_{B}|\hat{V}|\varphi^{(0)}_B}\rangle
    \braket{\varphi^{(0)}_{B}|\hat{V}|\varphi^{(0)}_n;0}\rangle}
    {(\epsilon_n-\epsilon_B)(\epsilon_n-\epsilon_B)}}\nn[10pt]
    &\hspace{100pt}
    \textcolor{red}{-E^{(1)}_n
    \frac{\langle\braket{\varphi^{(0)}_{n};8|\hat{V}|\varphi^{(0)}_B}
    \braket{\varphi^{(0)}_{B}|\hat{V}|\varphi^{(0)}_n;0}\rangle}{(\epsilon_n-\epsilon_B)^2}
    \Biggr\}\Biggr]}
    \nn[10pt]
    %
    %
    &+
    \ket{\varphi^{(0)}_D}
    \textcolor{blue}{
    \frac{\braket{\varphi^{(0)}_{D}|\hat{V}|\varphi^{(0)}_n;0}\rangle}{(\epsilon_n-\epsilon_D)}}
    +
    \ket{\varphi^{(0)}_B}
    \textcolor{blue}{
    \frac{\braket{\varphi^{(0)}_{B}|\hat{V}|\varphi^{(0)}_n;0}\rangle}{(\epsilon_n-\epsilon_B)}}
\end{align}

$|\varphi^{(0)}_{n};0\rangle\rangle=C_{0,0}\ket{0}+C_{8,0}\ket{8}$, 
$|\varphi^{(0)}_{n};8\rangle\rangle=C_{0,8}\ket{0}+C_{8,8}\ket{8}$,  $|\varphi^{(0)}_{D}\rangle=\ket{D}$, $|\varphi^{(0)}_{B}\rangle=\ket{B}$であるから,


\begin{align}
    \ket{\varphi_{n,0}^{(1)}}
    &=
    |\varphi^{(0)}_n;8\rangle\rangle
    \textcolor{red}{\frac{1}{(E^{(2)}_{n,0}-E^{(2)}_{n,8})}}\nn[10pt]
    &\times
    \textcolor{red}{\Biggl[\Biggl\{
    \frac{\langle\braket{\varphi^{(0)}_{n};8|\hat{V}|D}
    \langle{D|\hat{V}|D}\rangle
    \braket{D|\hat{V}|\varphi^{(0)}_n;0}\rangle}
    {(\epsilon_n-\epsilon_D)(\epsilon_n-\epsilon_D)}}\nn[10pt]
    %
    &\hspace{100pt}
    +{\frac{\langle\braket{\varphi^{(0)}_{n};8|\hat{V}|D}
    \langle{D|\hat{V}|B}\rangle
    \braket{B|\hat{V}|\varphi^{(0)}_n;0}\rangle}
    {(\epsilon_n-\epsilon_D)(\epsilon_n-\epsilon_B)}}\nn[10pt]
    &\hspace{100pt}
    \textcolor{red}{-E^{(1)}_n
    \frac{\langle\braket{\varphi^{(0)}_{n};8|\hat{V}|D}
    \braket{D|\hat{V}|\varphi^{(0)}_n;0}\rangle}{(\epsilon_n-\epsilon_D)^2}
    \Biggr\}}
    \nn[10pt]
    %
    %
    &+\textcolor{red}{\Biggl\{
    \frac{\langle\braket{\varphi^{(0)}_{n};8|\hat{V}|B}
    \langle{B|\hat{V}|D}\rangle
    \braket{D|\hat{V}|\varphi^{(0)}_n;0}\rangle}
    {(\epsilon_n-\epsilon_B)(\epsilon_n-\epsilon_D)}}\nn[10pt]
    %
    &\hspace{100pt}
    +{\frac{\langle\braket{\varphi^{(0)}_{n};8|\hat{V}|B}
    \langle{B|\hat{V}|B}\rangle
    \braket{B|\hat{V}|\varphi^{(0)}_n;0}\rangle}
    {(\epsilon_n-\epsilon_B)(\epsilon_n-\epsilon_B)}}\nn[10pt]
    &\hspace{100pt}
    \textcolor{red}{-E^{(1)}_n
    \frac{\langle\braket{\varphi^{(0)}_{n};8|\hat{V}|B}
    \braket{B|\hat{V}|\varphi^{(0)}_n;0}\rangle}{(\epsilon_n-\epsilon_B)^2}
    \Biggr\}\Biggr]}
    \nn[10pt]
    %
    %
    &+
    \ket{D}
    \textcolor{blue}{
    \frac{\braket{D|\hat{V}|\varphi^{(0)}_n;0}\rangle}{(\epsilon_n-\epsilon_D)}}
    +
    \ket{B}
    \textcolor{blue}{
    \frac{\braket{B|\hat{V}|\varphi^{(0)}_n;0}\rangle}{(\epsilon_n-\epsilon_B)}}
\end{align}

ここで,
\begin{equation}
    \langle{D|\hat{V}|D}\rangle
    =\langle{B|\hat{V}|B}\rangle
    =\langle{D|\hat{V}|B}\rangle
    =\langle{B|\hat{V}|D}\rangle=0
\end{equation}

\begin{equation}
    E^{(1)}_n = 0
\end{equation}

\begin{align}
    \braket{D|\hat{V}|\varphi^{(0)}_n;0}\rangle
    &=C_{0,0}\braket{D|\hat{V}|0} + C_{8,0}\braket{D|\hat{V}|8}
    =C_{0,0}(-\sqrt{2}\beta/\sqrt{2}) + C_{8,0} (\sqrt{8\cdot7}\beta/\sqrt{2})\\[10pt]
    \braket{B|\hat{V}|\varphi^{(0)}_n;0}\rangle
    &=C_{0,0}\braket{B|\hat{V}|0} + C_{8,0}\braket{B|\hat{V}|8}
    =C_{0,0}(\sqrt{2}\beta/\sqrt{2}) + C_{8,0} (\sqrt{8\cdot7}\beta/\sqrt{2})
\end{align}

であるから,

\begin{align}
    \ket{\varphi_{n,0}^{(1)}}
    &=
    \ket{D}
    \textcolor{blue}{
    \frac{\braket{D|\hat{V}|\varphi^{(0)}_n;0}\rangle}{(\epsilon_{0}-\epsilon_D)}}
    +
    \ket{B}
    \textcolor{blue}{
    \frac{\braket{B|\hat{V}|\varphi^{(0)}_n;0}\rangle}{(\epsilon_0-\epsilon_B)}}\\[10pt]
    &=\ket{D}
    \frac{1}{(\epsilon_{0}-\epsilon_D)}
    \Bigl\{C_{0,0}(-\sqrt{2}\beta/\sqrt{2}) + C_{8,0} (\sqrt{8\cdot7}\beta/\sqrt{2})\Bigr\}\nn[10pt]
    &+
    \ket{B}
    \frac{1}{(\epsilon_{0}-\epsilon_B)}
    \Bigl\{C_{0,0}(\sqrt{2}\beta/\sqrt{2}) + C_{8,0} (\sqrt{8\cdot7}\beta/\sqrt{2})\Bigr\}
\end{align}






\section*{\textcolor{red}{永年方程式を解き,2次の補正項と規格化定数を求める}}
永年方程式
\begin{align}\label{2ndpertubation_matrix}
\sum_{\beta=1}^{N}\Biggl[
E^{(2)}_n \delta_{\alpha,\beta}
-(\hat{V}^{(2)})_{\alpha,\beta}
\biggr]C_{\beta,\alpha}=0,
\end{align}
ここで,
\begin{equation}
    (\hat{V}^{(2)})_{\alpha,\beta}
    \equiv\sum_{m\neq n}
    \frac{\langle{\varphi^{(0)}_n;\alpha|\hat{V}|\varphi^{(0)}_m}\rangle
    \langle{\varphi^{(0)}_{m}|\hat{V}|\varphi^{(0)}_n;\beta}\rangle}
    {(\epsilon_n-\epsilon_m)}
\end{equation}
を解き,エネルギー固有値の2次の補正項$E_{n,0}^{(2)}$, $E_{n,8}^{(2)}$と固有状態の第ゼロ近似に関する展開係数,$C_{0,0}$, $C_{8,0}$, $C_{0,8}$, $C_{8,8}$を求める.





% この$C_{\beta}$に関する斉1次連立方程式が0以外の解を持つためには,$C_{\beta}$の係数のつくる行列が0でなくてはならない.すなわち,固有値$E^{(2)}_{n}$は以下の特性方程式の解である:
% \begin{equation}\label{2nd_eigen_eq}
%     \det{(E^{(2)}_n \delta_{\alpha,\beta}
%     -(\hat{V}^{(2)})_{\alpha,\beta})}=0
% \end{equation}
% この固有値方程式は重解も含めて$N$個の解 : $E^{(2)}_{n,\alpha}$, $(\alpha=1,2,\ldots,N)$を持つ.そして,$N$個のそれぞれの解$E^{(2)}_{n,\alpha}$を行列方程式に代入し,規格化条件$\sum_{\beta}|C_{\beta}|=1$のもとで,\eqref{2ndpertubation_matrix}を解くことにより,それぞれの解$E^{(2)}_{n,\alpha}$に対する係数が決定する.その係数を改めて$C_{\beta,\alpha}$と書き,これに対応する固有ベクトルを
% \begin{equation}
%     |\varphi^{(0)}_{n};\alpha\rangle\rangle
%     =\sum_{\beta=1}^{N}
%     \ket{\varphi^{(0)}_{n};\beta}C_{\beta,\alpha}
% \end{equation}
% と書き直す.これで第0近似での固有状態を決めることができた.固有状態$|\varphi^{(0)}_{n};\alpha\rangle\rangle$を\eqref{pereq2-1_degenerate}の$|\varphi^{(0)}_{n}\rangle\rangle$へ代入すると,
% \begin{align}
% (\epsilon_n-\epsilon_n)\braket{\varphi^{(0)}_n;\gamma|\varphi^{(2)}_n;\alpha}
% &=\langle{\varphi^{(0)}_n;\gamma|\hat{V}|\varphi^{(1)}_n;\alpha}\rangle
% -E^{(1)}_n\langle{\varphi^{(0)}_n;\gamma|\varphi^{(1)}_n;\alpha}\rangle
% -E^{(2)}\langle{\varphi^{(0)}_n;\gamma|\varphi^{(0)}_n;\alpha}\rangle\rangle
% %
% \end{align}
まず,非摂動ハミルトニアンのエネルギー固有値のうち,縮退していない状態のエネルギー固有値は以下のように与えられる:
\begin{align}
    \epsilon_D&=
    {\Delta_{1}}\\[10pt]
    %
    \epsilon_B&=
    \frac{1}{2}\left({\Delta_{1}}+{\Delta_{2}}-\sqrt{{\Delta_{1}}^2-2{\Delta_{1}}{\Delta_{2}}+{\Delta_{2}}^2+8p^2}\right)\nn[10pt]
    &=\frac{1}{2}\left({\Delta_{1}}+{\Delta_{2}}-\sqrt{{
    (\Delta_{1}}-\Delta_{2})^2+8p^2}\right)\nn[10pt]
    &=\frac{1}{2}\left({\Delta_{1}}+{\Delta_{2}}-\epsilon\right)\nn[10pt]
\end{align}
ここで,
\begin{align}
    \epsilon\equiv
    \sqrt{{
    (\Delta_{1}}-\Delta_{2})^2+8p^2}
    ={(\Delta_{1}}-\Delta_{2})
    \sqrt{1+\frac{8p^2}{(\Delta_{2}-\Delta_{1})^2}}
\end{align}
とおいた.
\begin{equation}
    x\equiv\frac{8p^2}{(\Delta_{2}-\Delta_{1})^2}
\end{equation}
とおき,次のTaylor展開の公式を
\begin{equation}
    \sqrt{1+x}
    =1+\frac{x}{2}-\frac{x^2}{8}+\frac{x^3}{16}-\frac{5x^4}{128}+\frac{7x^5}{256}+O\left(x^6\right)
\end{equation}
使うと,
\begin{align}
    \epsilon
    &\simeq{(\Delta_{1}}-\Delta_{2})
    \sqrt{1+\frac{8p^2}{(\Delta_{2}-\Delta_{1})^2}}\nn[10pt]
    &={(\Delta_{2}}-{\Delta_{1}})
    \Bigl(1+\frac{8p^2}{2(\Delta_{2}-\Delta_{1})^2}\Bigr)
    ={(\Delta_{2}}-{\Delta_{1}})
    \Bigl(1+\frac{2({p_{1}}^2+{p_{2}}^2)}{(\Delta_{2}-\Delta_{1})^2}\Bigr)\nn[10pt]
    &={(\Delta_{2}}-{\Delta_{1}})
    +\frac{2({p_{1}}^2+{p_{2}}^2)}{(\Delta_{2}-\Delta_{1})}
\end{align}
となり,エネルギー固有値$\epsilon_B$は以下のようになる:
\begin{equation}
    \epsilon_B
    \simeq\frac{1}{2}\left({\Delta_{1}}
    +{\Delta_{2}}-{(\Delta_{2}}-{\Delta_{1}})
    +\frac{2({p_{1}}^2+{p_{2}}^2)}{(\Delta_{2}-\Delta_{1})}\right)
    =\Delta_1 - \frac{{p_{1}}^2+{p_{2}}^2}{(\Delta_{2}-\Delta_{1})}
    =\Delta_1 - \delta_0
\end{equation}
ここで,
\begin{equation}
    \delta_0 \equiv \frac{{p_{1}}^2+{p_{2}}^2}{(\Delta_{2}-\Delta_{1})}
\end{equation}

まず,2次摂動に関する行列要素を計算する:
\begin{align}
    (\hat{V}^{(2)})_{0,0}
    &=\sum_{m\neq n}
    \frac{\langle{0|\hat{V}|\varphi^{(0)}_m}\rangle
    \langle{\varphi^{(0)}_{m}|\hat{V}|0}\rangle}
    {(\epsilon_0-\epsilon_m)}
    =
    \frac{\langle{0|\hat{V}|B}\rangle
    \langle{B|\hat{V}|0}\rangle}
    {(-\epsilon_B)}
    +\frac{\langle{0|\hat{V}|D}\rangle
    \langle{D|\hat{V}|0}\rangle}
    {(-\epsilon_D)}\nn[10pt]
    %
    &=\frac{|\langle{0|\hat{V}|B}\rangle|^2}
    {(-\epsilon_B)}
    +\frac{|\langle{0|\hat{V}|D}\rangle|^2}
    {(-\epsilon_D)}
    =\frac{(2\cdot1)  / 2}
    {(-\epsilon_B)}
    +\frac{(2\cdot1)  / 2}
    {(-\epsilon_D)}
\end{align}


\begin{align}
    (\hat{V}^{(2)})_{8,8}
    &=\sum_{m\neq n}
    \frac{\langle{8|\hat{V}|\varphi^{(8)}_m}\rangle
    \langle{\varphi^{(8)}_{m}|\hat{V}|8}\rangle}
    {(\epsilon_8-\epsilon_m)}
    =
    \frac{\langle{8|\hat{V}|B}\rangle
    \langle{B|\hat{V}|8}\rangle}
    {(-\epsilon_B)}
    +\frac{\langle{8|\hat{V}|D}\rangle
    \langle{D|\hat{V}|8}\rangle}
    {(-\epsilon_D)}\nn[10pt]
    %
    &=\frac{|\langle{8|\hat{V}|B}\rangle|^2}
    {(-\epsilon_B)}
    +\frac{|\langle{8|\hat{V}|D}\rangle|^2}
    {(-\epsilon_D)}
    =\frac{(8\cdot7) / 2}
    {(-\epsilon_B)}
    +\frac{(8\cdot7) / 2}
    {(-\epsilon_D)}
\end{align}




\begin{align}
    (\hat{V}^{(2)})_{0,8}
    &=(\hat{V}^{(2)})_{8,0}
    =\sum_{m\neq n}
    \frac{\langle{0|\hat{V}|\varphi^{(8)}_m}\rangle
    \langle{\varphi^{(0)}_{m}|\hat{V}|8}\rangle}
    {(\epsilon_0-\epsilon_m)}
    =
    \frac{\langle{0|\hat{V}|B}\rangle
    \langle{B|\hat{V}|8}\rangle}
    {(-\epsilon_B)}
    +\frac{\langle{8|\hat{V}|D}\rangle
    \langle{D|\hat{V}|0}\rangle}
    {(-\epsilon_D)}\nn[10pt]
    & =
    \frac{\sqrt{8\cdot7\cdot2\cdot1}/2}
    {(-\epsilon_B)}
    +\frac{-\sqrt{8\cdot7\cdot2\cdot1}/2}
    {(-\epsilon_D)}
\end{align}





$\epsilon_B = \Delta_1 - \delta_0$, $\epsilon_D=\Delta_1$であるから,
\begin{align}
    (\hat{V}^{(2)})_{0,0}
    &
    =\frac{1 / 2}
    {(-\epsilon_B)}
    +\frac{1 / 2}
    {(-\epsilon_D)}
    =\frac{1}
    {-\Delta_1 + \delta_0}
    +\frac{1}
    {-\Delta_1}\nn[10pt]
    &
    =\frac{-\Delta_1-\Delta_1 + \delta_0}{-\Delta_1(-\Delta_1 + \delta_0)}
    =\frac{-2\Delta_1 + \delta_0}{-\Delta_1(-\Delta_1 + \delta_0)}
\end{align}


\begin{align}
    (\hat{V}^{(2)})_{8,8}
    &
    =\frac{(8\cdot7) / 2}
    {(-\epsilon_B)}
    +\frac{(8\cdot7) / 2}
    {(-\epsilon_D)}
    =\frac{(8\cdot7) / 2}
    {-\Delta_1 + \delta_0}
    +\frac{(8\cdot7) / 2}
    {-\Delta_1}
    =\frac{8\cdot7}{2}\frac{-2\Delta_1 + \delta_0}{-\Delta_1(-\Delta_1 + \delta_0)}
\end{align}




\begin{align}
    (\hat{V}^{(2)})_{0,8}
    & =
    \frac{\sqrt{8\cdot7\cdot2\cdot1}/2}
    {(-\epsilon_B)}
    +\frac{-\sqrt{8\cdot7\cdot2\cdot1}/2}
    {(-\epsilon_D)}
    =\frac{\sqrt{8\cdot7\cdot2\cdot1}}{2}
    \Bigl(
    \frac{1}
    {-\Delta_1 + \delta_0}
    -\frac{1}
    {-\Delta_1}
    \Bigr)
    \nn[10pt]
    %
    &=\frac{\sqrt{8\cdot7\cdot2\cdot1}}{2}
    \Bigl(
    \frac{-\delta_0}
    {-\Delta_1(-\Delta_1 + \delta_0)}
    \Bigr)
    =\frac{\epsilon^{\prime}}{B}
\end{align}
ここで,$\epsilon^{\prime}=\sqrt{8\cdot7\cdot2\cdot1}(-\delta_0)$とおいた.

\begin{equation}
    E^{(2)}_n=\frac{
    (\hat{V}^{(2)})_{0,0} + (\hat{V}^{(2)})_{8,8}
    \mp \sqrt{D}
    }{2},
\end{equation}

\begin{align}
    D&=[(\hat{V}^{(2)})_{0,0} - \hat{V}^{(2)})_{8,8}]^2 + 4(\hat{V}^{(2)})_{0,8})^2\nn[10pt]
    &=(28-1)^2\Biggl(
    \frac{-2\Delta_1 + \delta_0}{-\Delta_1(-\Delta_1 + \delta_0)}
    \Biggr)^2
    +4\frac{8\cdot7\cdot2}{4}\Bigl(
    \frac{-\delta_0}
    {-\Delta_1(-\Delta_1 + \delta_0)}
    \Bigr)^2\nn[10pt]
    %%
    &=\frac{27^2(-2\Delta_1 + \delta_0)^2 + 8\cdot7\cdot2\delta_0^2}
    {\{-\Delta_1(-\Delta_1 + \delta_0)\}^2}
\end{align}

よって,
\begin{align}
    E^{(2)}_n&=\frac{
    27\frac{-2\Delta_1 + \delta_0}{-\Delta_1(-\Delta_1 + \delta_0)}
    \mp \sqrt{
    \frac{27^2(-2\Delta_1 + \delta_0)^2 + 8\cdot7\cdot2\delta_0^2}
    {\{-\Delta_1(-\Delta_1 + \delta_0)\}^2}
    }
    }{2}\nn[10pt]
    &=\frac{
    27(-2\Delta_1 + \delta_0)
    \mp \sqrt{
    27^2(-2\Delta_1 + \delta_0)^2 + 8\cdot7\cdot2\delta_0^2  
    }
    }{2\{-\Delta_1(-\Delta_1 + \delta_0)\}}\nn[10pt]
    &=\frac{
    27(-2\Delta_1 + \delta_0)}
    {2\{-\Delta_1(-\Delta_1 + \delta_0)\}}
    \Biggl[
    1\mp \sqrt{1+
    \frac{8\cdot7\cdot2\delta_0^2}{27^2(-2\Delta_1 + \delta_0)^2}  
    }
    \Biggr]
    \nn[10pt]
    &=\frac{A}{B}[1\mp\sqrt{1+\delta}]
\end{align}
ここで,
\begin{align}
    A&\equiv 27(-2\Delta_1 + \delta_0)\\[10pt]
    B&\equiv 2\{-\Delta_1(-\Delta_1 + \delta_0)\}\\[10pt]
    C&\equiv\epsilon^{\prime2}=8\cdot7\cdot2\delta_0^2\\[10pt]
    \delta&=C/A^2\\[10pt]
    \delta_0 &= \frac{{p_{1}}^2+{p_{2}}^2}{(\Delta_{2}-\Delta_{1})}
\end{align}
とおいた.

$E_n^{(2)}=E_{n,0}^{(2)}$のとき,規格化定数は
\begin{equation}
    \frac{C_{0,0}}{C_{0,8}} 
    = \frac{(\hat{V}^{(2)})_{0,0} - E_{n,0}^{(2)}}{2(\hat{V}^{(2)})_{0,8} }
    =  \frac{(\hat{V}^{(2)})_{0,0} - (\hat{V}^{(2)})_{8,8} \pm \sqrt{D}}
    {2(\hat{V}^{(2)})_{0,8} }
\end{equation}
\begin{align}
    %
\end{align}






%%%%%%%%%%%%%%%%%%%%%%%%%%%%%%%%%%%%%
%%%%%%%%%%%%%%%%%%%%%%%%%%%%%%%%%%%%%
\subsection{$n=0$と$n=12$が縮退している場合}
まず,$n=0$と$n=12$が縮退している場合を考えよう.この場合のHamiltonianは以下のように書ける:
\begin{align}
     \hat{H}_{\rm{KPO}}&=
   \bordermatrix{     
    & \bra{0} &  \bra{2} &  \bra{4}&  \bra{6}&  \bra{8} &\bra{10} &\bra{12}\cr
   \ket{0}&E_0&\sqrt{2\cdot1}\beta&0&0&0&0&0\cr
  \ket{2}&\sqrt{2\cdot1}\beta&E_2&\sqrt{4\cdot3}\beta&0&0&0&0\cr
  \ket{4}&0&\sqrt{4\cdot3}\beta&E_4&\sqrt{6\cdot5}\beta&0&0&0\cr
  \ket{6}&0&0&\sqrt{6\cdot5}\beta&E_6&\sqrt{8\cdot7}\beta&0&0\cr
  \ket{8}&0&0&0&\sqrt{8\cdot7}\beta&E_8&\sqrt{10\cdot9}\beta&0\cr
  \ket{10}&0&0&0&0&\sqrt{10\cdot9}\beta&E_{10}&\sqrt{12\cdot11}\beta\cr
  \ket{12}&0&0&0&0&0&\sqrt{12\cdot11}\beta&E_{12}\cr
            }\\[10pt]
    &=
   \bordermatrix{     
    & \bra{0} &  \bra{2} &  \bra{4}&  \bra{6}&  \bra{8} &\bra{10} &\bra{12}\cr
   \ket{0}&0&\sqrt{2\cdot1}\beta&0&0&0&0&0\cr
  \ket{2}&\sqrt{2\cdot1}\beta&\Delta_1&\sqrt{4\cdot3}\beta&0&0&0&0\cr
  \ket{4}&0&\sqrt{4\cdot3}\beta&\Delta_2&\sqrt{6\cdot5}\beta&0&0&0\cr
  \ket{6}&0&0&\sqrt{6\cdot5}\beta&\Delta_3&\sqrt{8\cdot7}\beta&0&0\cr
  \ket{8}&0&0&0&\sqrt{8\cdot7}\beta&\Delta_2&\sqrt{10\cdot9}\beta&0\cr
  \ket{10}&0&0&0&0&\sqrt{10\cdot9}\beta&\Delta_1&\sqrt{12\cdot11}\beta\cr
  \ket{12}&0&0&0&0&0&\sqrt{12\cdot11}\beta&0\cr
            }
\end{align}
ここで,$E_0=E_12=0$, $E_2=E_{10}=\Delta_1$, $E_4=E_{8}=\Delta_2$, $E_=\Delta_3$である.




\begin{align}
    \hat{H}_0 &= E_{2}\ket{2}\bra{2} + E_{10} \ket{10}\bra{10} + E_{D_{4,8}} \ket{D_{4,8}}\bra{D_{4,8}}+ E_{B_{4,8}} \ket{B_{4,8}}\bra{B_{4,8}}\\[10pt]
    \hat{V}&= \sqrt{4\cdot3}\beta\ket{2}\bra{4} + \sqrt{10\cdot9}\beta\ket{10}\bra{8} + {\rm{h.c.}}
\end{align}


\begin{align}
     \hat{H}_{\rm{KPO}}
    &=
   \bordermatrix{     
    & \bra{2} &  \bra{2} &  \bra{6}&  \bra{8}\cr
   \ket{0}&0&\sqrt{2\cdot1}\beta&0&0\cr
  \ket{2}&\sqrt{2\cdot1}\beta&\Delta_1&0&0\cr
  \ket{6}&0&0&\Delta_1&\sqrt{8\cdot7}\beta\cr
  \ket{8}&0&0&\sqrt{8\cdot7}\beta&0\cr
            }
\end{align}
次の状態を定義する:
\begin{align}
    \ket{D_{4,8}} &= \frac{1}{\sqrt{2}}(-\ket{4}+\ket{8})\\[10pt]
    \ket{B_{4,8}} &= \frac{1}{\sqrt{2}}(\ket{4}+\ket{8})
\end{align}
摂動Hamiltonianを基底$\{\ket{2},\ket{10},\ket{D_{4,8}},\ket{B_{4,8}}\}$で展開する:
\begin{align}
     \hat{V}
    &=
   \bordermatrix{     
    & \bra{2} &  \bra{10} &  \bra{D_{4,8}}&  \bra{B_{4,8}}\cr
   \ket{2}&\langle{2|\hat{V}|2}\rangle&\langle{2|\hat{V}|10}\rangle&\langle{2|\hat{V}|D_{4,8}}\rangle
   &\langle{2|\hat{V}|B_{4,8}}\rangle\cr
  \ket{10}&\langle{10|\hat{V}|2}\rangle&\langle{10|\hat{V}|10}\rangle&\langle{10|\hat{V}|D_{4,8}}\rangle
   &\langle{10|\hat{V}|B_{4,8}}\rangle\cr
  \ket{D_{4,8}}&\langle{D_{4,8}|\hat{V}|2}\rangle&\langle{D_{4,8}|\hat{V}|10}\rangle&\langle{D_{4,8}|\hat{V}|D_{4,8}}\rangle
   &\langle{D_{4,8}|\hat{V}|B_{4,8}}\rangle\cr
  \ket{B_{4,8}}&\langle{B_{4,8}|\hat{V}|2}\rangle&\langle{B_{4,8}|\hat{V}|10}\rangle&\langle{B_{4,8}|\hat{V}|D_{4,8}}\rangle
   &\langle{B_{4,8}|\hat{V}|B_{4,8}}\rangle\cr
    }\\[10pt]
    &=
   \bordermatrix{     
   & \bra{2} &  \bra{10} &  \bra{D_{4,8}}&  \bra{B_{4,8}}\cr
   \ket{2}&0&0&-\sqrt{4\cdot3}\beta/\sqrt{2}&\sqrt{4\cdot3}\beta/\sqrt{2}\cr
  \ket{10}&0&0&\sqrt{10\cdot9}\beta/\sqrt{2}&\sqrt{10\cdot9}\beta/\sqrt{2}\cr
  \ket{D_{4,8}}&-\sqrt{4\cdot3}\beta/\sqrt{2}&\sqrt{4\cdot3}\beta/\sqrt{2}&0&0\cr
  \ket{B_{4,8}}&\sqrt{10\cdot9}\beta/\sqrt{2}&\sqrt{10\cdot9}\beta/\sqrt{2}&0&0&\cr
            }
\end{align}

\begin{align}
    \langle{2|\hat{V}|D_{4,8}}\rangle&=\langle{D_{4,8}|\hat{V}|2}\rangle
    =-\sqrt{4\cdot3}\beta/\sqrt{2}\\[10pt]
    \langle{2|\hat{V}|B_{4,8}}\rangle&=\langle{B_{4,8}|\hat{V}|2}\rangle
    =\sqrt{4\cdot3}\beta/\sqrt{2}\\[10pt]
    \langle{10|\hat{V}|D_{4,8}}\rangle&=\langle{D_{4,8}|\hat{V}|10}\rangle
    =\sqrt{10\cdot9}\beta/\sqrt{2}\\[10pt]
   \langle{10|\hat{V}|B_{4,8}}\rangle&=\langle{B_{4,8}|\hat{V}|10}\rangle
   =\sqrt{10\cdot9}\beta/\sqrt{2}
\end{align}


\begin{align}
    \ket{\varphi_{n,0}^{(1)}}
    &=
    \sum_{\textcolor{red}{\gamma\neq0}}|\varphi^{(0)}_n;8\rangle\rangle
    \textcolor{red}{\frac{1}{(E^{(2)}_{n,0}-E^{(2)}_{n,8})}
    \sum_{m\neq n}\Biggl[\sum_{p\neq n}
    \frac{\langle\braket{\varphi^{(0)}_{n};\gamma|\hat{V}|\varphi^{(0)}_m}
    \langle{\varphi^{(0)}_{m}|\hat{V}|\varphi^{(0)}_p}\rangle
    \braket{\varphi^{(0)}_{p}|\hat{V}|\varphi^{(0)}_n;0}\rangle}
    {(\epsilon_n-\epsilon_m)(\epsilon_n-\epsilon_p)}}\nn[10pt]
    %
    &\hspace{100pt}
    \textcolor{red}{-E^{(1)}_n
    \frac{\langle\braket{\varphi^{(0)}_{n};\gamma|\hat{V}|\varphi^{(0)}_m}
    \braket{\varphi^{(0)}_{m}|\hat{V}|\varphi^{(0)}_n;0}\rangle}{(\epsilon_n-\epsilon_m)^2}
    \Biggr]}
    \nn[10pt]
    %
    &\hspace{200pt}+
    \sum_{m\neq n}\ket{\varphi^{(0)}_m}
    \textcolor{blue}{
    \frac{\braket{\varphi^{(0)}_{m}|\hat{V}|\varphi^{(0)}_n;0}\rangle}{(\epsilon_n-\epsilon_m)}}\nn[10pt]
    %%%%%%%%%%%%%%%%%%%%
    %%%%%%%%%%%%%%%%%%%%
    &=
    |\varphi^{(0)}_n;8\rangle\rangle
    \textcolor{red}{\frac{1}{(E^{(2)}_{n,0}-E^{(2)}_{n,8})}
    \sum_{m\neq n}\Biggl[\sum_{p\neq n}
    \frac{\langle\braket{\varphi^{(0)}_{n};8|\hat{V}|\varphi^{(0)}_m}
    \langle{\varphi^{(0)}_{m}|\hat{V}|\varphi^{(0)}_p}\rangle
    \braket{\varphi^{(0)}_{p}|\hat{V}|\varphi^{(0)}_n;0}\rangle}
    {(\epsilon_n-\epsilon_m)(\epsilon_n-\epsilon_p)}}\nn[10pt]
    %
    &\hspace{100pt}
    \textcolor{red}{-E^{(1)}_n
    \frac{\langle\braket{\varphi^{(0)}_{n};8|\hat{V}|\varphi^{(0)}_m}
    \braket{\varphi^{(0)}_{m}|\hat{V}|\varphi^{(0)}_n;0}\rangle}{(\epsilon_n-\epsilon_m)^2}
    \Biggr]}
    \nn[10pt]
    %
    &\hspace{200pt}+
    \sum_{m\neq n}\ket{\varphi^{(0)}_m}
    \textcolor{blue}{
    \frac{\braket{\varphi^{(0)}_{m}|\hat{V}|\varphi^{(0)}_n;0}\rangle}{(\epsilon_n-\epsilon_m)}}\nn[10pt]
    %%%%%%%%%%%%%%%%%%%%
    %%%%%%%%%%%%%%%%%%%%
    &=
    |\varphi^{(0)}_n;8\rangle\rangle
    \textcolor{red}{\frac{1}{(E^{(2)}_{n,0}-E^{(2)}_{n,8})}
    \sum_{m\neq n}\Biggl[
    \frac{\langle\braket{\varphi^{(0)}_{n};8|\hat{V}|\varphi^{(0)}_m}
    \langle{\varphi^{(0)}_{m}|\hat{V}|\varphi^{(0)}_D}\rangle
    \braket{\varphi^{(0)}_{D}|\hat{V}|\varphi^{(0)}_n;0}\rangle}
    {(\epsilon_n-\epsilon_m)(\epsilon_n-\epsilon_D)}}\nn[10pt]
    %
    &\hspace{100pt}
    +{\frac{\langle\braket{\varphi^{(0)}_{n};8|\hat{V}|\varphi^{(0)}_m}
    \langle{\varphi^{(0)}_{m}|\hat{V}|\varphi^{(0)}_B}\rangle
    \braket{B|\hat{V}|\varphi^{(0)}_n;0}\rangle}
    {(\epsilon_n-\epsilon_m)(\epsilon_n-\epsilon_B)}}\nn[10pt]
    &\hspace{100pt}
    \textcolor{red}{-E^{(1)}_n
    \frac{\langle\braket{\varphi^{(0)}_{n};8|\hat{V}|\varphi^{(0)}_m}
    \braket{\varphi^{(0)}_{m}|\hat{V}|\varphi^{(0)}_n;0}\rangle}{(\epsilon_n-\epsilon_m)^2}
    \Biggr]}
    \nn[10pt]
    %
    &\hspace{200pt}+
    \sum_{m\neq n}\ket{\varphi^{(0)}_m}
    \textcolor{blue}{
    \frac{\braket{\varphi^{(0)}_{m}|\hat{V}|\varphi^{(0)}_n;0}\rangle}{(\epsilon_n-\epsilon_m)}}
\end{align}



\begin{align}
    \ket{\varphi_{n,0}^{(1)}}
    &=
    |\varphi^{(0)}_n;8\rangle\rangle
    \textcolor{red}{\frac{1}{(E^{(2)}_{n,0}-E^{(2)}_{n,8})}}\nn[10pt]
    &\times
    \textcolor{red}{\Biggl[\Biggl\{
    \frac{\langle\braket{\varphi^{(0)}_{n};8|\hat{V}|\varphi^{(0)}_D}
    \langle{\varphi^{(0)}_{D}|\hat{V}|\varphi^{(0)}_D}\rangle
    \braket{\varphi^{(0)}_{D}|\hat{V}|\varphi^{(0)}_n;0}\rangle}
    {(\epsilon_n-\epsilon_D)(\epsilon_n-\epsilon_D)}}\nn[10pt]
    %
    &\hspace{100pt}
    +{\frac{\langle\braket{\varphi^{(0)}_{n};8|\hat{V}|\varphi^{(0)}_D}
    \langle{\varphi^{(0)}_{D}|\hat{V}|\varphi^{(0)}_B}\rangle
    \braket{\varphi^{(0)}_{B}|\hat{V}|\varphi^{(0)}_n;0}\rangle}
    {(\epsilon_n-\epsilon_D)(\epsilon_n-\epsilon_B)}}\nn[10pt]
    &\hspace{100pt}
    \textcolor{red}{-E^{(1)}_n
    \frac{\langle\braket{\varphi^{(0)}_{n};8|\hat{V}|\varphi^{(0)}_D}
    \braket{\varphi^{(0)}_{D}|\hat{V}|\varphi^{(0)}_n;0}\rangle}{(\epsilon_n-\epsilon_D)^2}
    \Biggr\}}
    \nn[10pt]
    %
    %
    &+\textcolor{red}{\Biggl\{
    \frac{\langle\braket{\varphi^{(0)}_{n};8|\hat{V}|\varphi^{(0)}_B}
    \langle{\varphi^{(0)}_{B}|\hat{V}|\varphi^{(0)}_D}\rangle
    \braket{\varphi^{(0)}_{D}|\hat{V}|\varphi^{(0)}_n;0}\rangle}
    {(\epsilon_n-\epsilon_B)(\epsilon_n-\epsilon_D)}}\nn[10pt]
    %
    &\hspace{100pt}
    +{\frac{\langle\braket{\varphi^{(0)}_{n};8|\hat{V}|\varphi^{(0)}_B}
    \langle{\varphi^{(0)}_{B}|\hat{V}|\varphi^{(0)}_B}\rangle
    \braket{\varphi^{(0)}_{B}|\hat{V}|\varphi^{(0)}_n;0}\rangle}
    {(\epsilon_n-\epsilon_B)(\epsilon_n-\epsilon_B)}}\nn[10pt]
    &\hspace{100pt}
    \textcolor{red}{-E^{(1)}_n
    \frac{\langle\braket{\varphi^{(0)}_{n};8|\hat{V}|\varphi^{(0)}_B}
    \braket{\varphi^{(0)}_{B}|\hat{V}|\varphi^{(0)}_n;0}\rangle}{(\epsilon_n-\epsilon_B)^2}
    \Biggr\}\Biggr]}
    \nn[10pt]
    %
    %
    &+
    \ket{\varphi^{(0)}_D}
    \textcolor{blue}{
    \frac{\braket{\varphi^{(0)}_{D}|\hat{V}|\varphi^{(0)}_n;0}\rangle}{(\epsilon_n-\epsilon_D)}}
    +
    \ket{\varphi^{(0)}_B}
    \textcolor{blue}{
    \frac{\braket{\varphi^{(0)}_{B}|\hat{V}|\varphi^{(0)}_n;0}\rangle}{(\epsilon_n-\epsilon_B)}}
\end{align}

$|\varphi^{(0)}_{n};0\rangle\rangle=C_{0,0}\ket{0}+C_{8,0}\ket{8}$, 
$|\varphi^{(0)}_{n};8\rangle\rangle=C_{0,8}\ket{0}+C_{8,8}\ket{8}$,  $|\varphi^{(0)}_{D}\rangle=\ket{D}$, $|\varphi^{(0)}_{B}\rangle=\ket{B}$であるから,


\begin{align}
    \ket{\varphi_{n,0}^{(1)}}
    &=
    |\varphi^{(0)}_n;8\rangle\rangle
    \textcolor{red}{\frac{1}{(E^{(2)}_{n,0}-E^{(2)}_{n,8})}}\nn[10pt]
    &\times
    \textcolor{red}{\Biggl[\Biggl\{
    \frac{\langle\braket{\varphi^{(0)}_{n};8|\hat{V}|D}
    \langle{D|\hat{V}|D}\rangle
    \braket{D|\hat{V}|\varphi^{(0)}_n;0}\rangle}
    {(\epsilon_n-\epsilon_D)(\epsilon_n-\epsilon_D)}}\nn[10pt]
    %
    &\hspace{100pt}
    +{\frac{\langle\braket{\varphi^{(0)}_{n};8|\hat{V}|D}
    \langle{D|\hat{V}|B}\rangle
    \braket{B|\hat{V}|\varphi^{(0)}_n;0}\rangle}
    {(\epsilon_n-\epsilon_D)(\epsilon_n-\epsilon_B)}}\nn[10pt]
    &\hspace{100pt}
    \textcolor{red}{-E^{(1)}_n
    \frac{\langle\braket{\varphi^{(0)}_{n};8|\hat{V}|D}
    \braket{D|\hat{V}|\varphi^{(0)}_n;0}\rangle}{(\epsilon_n-\epsilon_D)^2}
    \Biggr\}}
    \nn[10pt]
    %
    %
    &+\textcolor{red}{\Biggl\{
    \frac{\langle\braket{\varphi^{(0)}_{n};8|\hat{V}|B}
    \langle{B|\hat{V}|D}\rangle
    \braket{D|\hat{V}|\varphi^{(0)}_n;0}\rangle}
    {(\epsilon_n-\epsilon_B)(\epsilon_n-\epsilon_D)}}\nn[10pt]
    %
    &\hspace{100pt}
    +{\frac{\langle\braket{\varphi^{(0)}_{n};8|\hat{V}|B}
    \langle{B|\hat{V}|B}\rangle
    \braket{B|\hat{V}|\varphi^{(0)}_n;0}\rangle}
    {(\epsilon_n-\epsilon_B)(\epsilon_n-\epsilon_B)}}\nn[10pt]
    &\hspace{100pt}
    \textcolor{red}{-E^{(1)}_n
    \frac{\langle\braket{\varphi^{(0)}_{n};8|\hat{V}|B}
    \braket{B|\hat{V}|\varphi^{(0)}_n;0}\rangle}{(\epsilon_n-\epsilon_B)^2}
    \Biggr\}\Biggr]}
    \nn[10pt]
    %
    %
    &+
    \ket{D}
    \textcolor{blue}{
    \frac{\braket{D|\hat{V}|\varphi^{(0)}_n;0}\rangle}{(\epsilon_n-\epsilon_D)}}
    +
    \ket{B}
    \textcolor{blue}{
    \frac{\braket{B|\hat{V}|\varphi^{(0)}_n;0}\rangle}{(\epsilon_n-\epsilon_B)}}
\end{align}

ここで,
\begin{equation}
    \langle{D|\hat{V}|D}\rangle
    =\langle{B|\hat{V}|B}\rangle
    =\langle{D|\hat{V}|B}\rangle
    =\langle{B|\hat{V}|D}\rangle=0
\end{equation}

\begin{equation}
    E^{(1)}_n = 0
\end{equation}

\begin{align}
    \braket{D|\hat{V}|\varphi^{(0)}_n;0}\rangle
    &=C_{0,0}\braket{D|\hat{V}|0} + C_{8,0}\braket{D|\hat{V}|8}
    =C_{0,0}(-\sqrt{2}\beta/\sqrt{2}) + C_{8,0} (\sqrt{8\cdot7}\beta/\sqrt{2})\\[10pt]
    \braket{B|\hat{V}|\varphi^{(0)}_n;0}\rangle
    &=C_{0,0}\braket{B|\hat{V}|0} + C_{8,0}\braket{B|\hat{V}|8}
    =C_{0,0}(\sqrt{2}\beta/\sqrt{2}) + C_{8,0} (\sqrt{8\cdot7}\beta/\sqrt{2})
\end{align}

であるから,

\begin{align}
    \ket{\varphi_{n,0}^{(1)}}
    &=
    \ket{D}
    \textcolor{blue}{
    \frac{\braket{D|\hat{V}|\varphi^{(0)}_n;0}\rangle}{(\epsilon_{0}-\epsilon_D)}}
    +
    \ket{B}
    \textcolor{blue}{
    \frac{\braket{B|\hat{V}|\varphi^{(0)}_n;0}\rangle}{(\epsilon_0-\epsilon_B)}}\\[10pt]
    &=\ket{D}
    \frac{1}{(\epsilon_{0}-\epsilon_D)}
    \Bigl\{C_{0,0}(-\sqrt{2}\beta/\sqrt{2}) + C_{8,0} (\sqrt{8\cdot7}\beta/\sqrt{2})\Bigr\}\nn[10pt]
    &+
    \ket{B}
    \frac{1}{(\epsilon_{0}-\epsilon_B)}
    \Bigl\{C_{0,0}(\sqrt{2}\beta/\sqrt{2}) + C_{8,0} (\sqrt{8\cdot7}\beta/\sqrt{2})\Bigr\}
\end{align}


\subsection*{}

\begin{align}
    \hat{H}_0 &= E_{0}\ket{0}\bra{0} + E_{12} \ket{12}\bra{12} 
    + E_{D_{2,10}} \ket{D_{2,10}}\bra{D_{2,10}}
    + E_{B_{2,10}} \ket{B_{2,10}}\bra{B_{2,10}}\\[10pt]
    \hat{V}&= \sqrt{2\cdot1}\beta\ket{0}\bra{2} + \sqrt{12\cdot11}\beta\ket{12}\bra{10} + {\rm{h.c.}}
\end{align}


\begin{align}
     \hat{H}_{\rm{KPO}}
    &=
   \bordermatrix{     
    & \bra{0} &  \bra{2} &  \bra{6}&  \bra{8}\cr
   \ket{0}&0&\sqrt{2\cdot1}\beta&0&0\cr
  \ket{2}&\sqrt{2\cdot1}\beta&\Delta_1&0&0\cr
  \ket{6}&0&0&\Delta_1&\sqrt{8\cdot7}\beta\cr
  \ket{8}&0&0&\sqrt{8\cdot7}\beta&0\cr
            }
\end{align}
次の状態を定義する:
\begin{align}
    \ket{D_{2,10}} &= \frac{1}{\sqrt{2}}(-\ket{2}+\ket{10})\\[10pt]
    \ket{B_{2,10}} &= \frac{1}{\sqrt{2}}(\ket{2}+\ket{10})
\end{align}
摂動Hamiltonianを基底$\{\ket{0},\ket{12},\ket{D_{2,10}},\ket{B_{2,10}}\}$で展開する:
\begin{align}
     \hat{V}
    &=
   \bordermatrix{     
    & \bra{0} &  \bra{12} &  \bra{D_{2,10}}&  \bra{B_{2,10}}\cr
   \ket{0}&\langle{0|\hat{V}|0}\rangle&\langle{0|\hat{V}|12}\rangle
   &\langle{0|\hat{V}|D_{2,10}}\rangle
   &\langle{0|\hat{V}|B_{2,10}}\rangle\cr
   %
  \ket{12}&\langle{12|\hat{V}|0}\rangle&\langle{12|\hat{V}|12}\rangle&\langle{12|\hat{V}|D_{2,10}}\rangle
   &\langle{12|\hat{V}|B_{2,10}}\rangle\cr
   %
  \ket{D_{2,10}}&\langle{D_{2,10}|\hat{V}|0}\rangle&\langle{D_{2,10}|\hat{V}|12}\rangle
  &\langle{D_{2,10}|\hat{V}|D_{2,10}}\rangle
   &\langle{D_{2,10}|\hat{V}|B_{2,10}}\rangle\cr
   %
  \ket{B_{2,10}}&\langle{B_{2,10}|\hat{V}|0}\rangle&\langle{B_{2,10}|\hat{V}|12}\rangle&\langle{B_{2,10}|\hat{V}|D_{2,10}}\rangle
   &\langle{B_{2,10}|\hat{V}|B_{2,10}}\rangle\cr
    }\\[10pt]
    &=
   \bordermatrix{     
    & \bra{0} &  \bra{12} &  \bra{D}&  \bra{B}\cr
   \ket{0}&0&0&-\sqrt{2}\beta/\sqrt{2}&\sqrt{2}\beta/\sqrt{2}\cr
  \ket{12}&0&0&\sqrt{12\cdot11}\beta/\sqrt{2}&\sqrt{12\cdot11}\beta/\sqrt{2}\cr
  \ket{D}&-\sqrt{2}\beta/\sqrt{2}&\sqrt{2}\beta/\sqrt{2}&0&0\cr
  \ket{B}&\sqrt{12\cdot11}\beta/\sqrt{2}&\sqrt{12\cdot11}\beta/\sqrt{2}&0&0&\cr
            }
\end{align}

\begin{align}
    \langle{0|\hat{V}|D}\rangle&=\langle{D|\hat{V}|0}\rangle=-\sqrt{2}\beta/\sqrt{2}\\[10pt]
    \langle{0|\hat{V}|B}\rangle&=\langle{B|\hat{V}|0}\rangle=\sqrt{2}\beta/\sqrt{2}\\[10pt]
    \langle{12|\hat{V}|D}\rangle&=\langle{D|\hat{V}|8}\rangle=\sqrt{12\cdot11}\beta/\sqrt{2}\\[10pt]
   \langle{12|\hat{V}|B}\rangle&=\langle{B|\hat{V}|8}\rangle=\sqrt{12\cdot11}\beta/\sqrt{2}
\end{align}




