\section{時間に依存しない縮退のある摂動論}
\subsection{縮退について}
エネルギー固有値$\epsilon_n$に属する$\hat{H}_0$の固有値が$N$重縮退している場合を考える.これらの縮退している固有状態を区別するために,量子数$\alpha=1,2,\ldots,N$を導入し,$|\varphi^{(0)}_n;\alpha\rangle$と書くことにする.そして,これらについて規格直交化しておく:
\begin{equation}
    \braket{\varphi^{(0)}_n;\alpha|\varphi^{(0)}_n;\beta} = \delta_{\alpha,\beta}
\end{equation}
また,$\epsilon_{n}$以外の他の固有状態$\epsilon_m$に属する固有ベクトルを$|\varphi^{(0)}_m\rangle$で表し,$\braket{\varphi^{(0)}_{m^\prime}|\varphi^{(0)}_m}=\delta_{m^\prime,m}$とする.さらにこのとき,$\braket{\varphi^{(0)}_n;\alpha|\varphi^{(0)}_m}=0$, $(\alpha=1,2,\ldots,N)$である.

縮退のある場合,$N$個の固有ベクトル$|\varphi^{(0)}_n;\alpha\rangle$のほかに,これらの任意の重ね合わせもまた$\hat{H}_0$の固有ベクトルである.そこで,Sch.eqの解として,次も考えられる:
\begin{equation}\label{Eq:dege_state_superposition}
    |\varphi^{(0)}_n\rangle\rangle = \sum_{\beta=1}^N |\varphi^{(0)}_n;\beta\rangle C_{\beta},
\end{equation}
後に,重ね合わせ状態$|\varphi^{(0)}_n\rangle\rangle$は,縮退がとけた場合のインデックスを書き加え,修正を行うことに注意.
係数$C_{\beta}$を求めることで,$\hat{H}_0$の固有状態$|\varphi^{(0)}_n\rangle\rangle$が決まる.このとき,$\hat{H}$の固有状態,固有値に関する摂動展開と摂動方程式は,それぞれ,次のように与えられる:
\begin{align}
\label{Ee}
E_n&=\epsilon_n+\lambda E^{(1)}_n+\lambda^2E^{(2)}_n+\cdots\cdots\\[10pt]
\label{pe}
\ket{\varphi_n}&=|\varphi^{(0)}_n\rangle\rangle +\lambda \ket{\varphi^{(1)}_n}+\lambda^2\ket{\varphi^{(2)}_n}+\cdots\cdots
\end{align}

\begin{align}
\label{pereq0}
\lambda^0&:&\hat{H}_0|\varphi^{(0)}_n\rangle\rangle &=\epsilon_n|\varphi^{(0)}_n\rangle\rangle \\[5pt]
%
\label{pereq1}
\lambda^1&:&(\epsilon_n-\hat{H}_0)\ket{\varphi^{(1)}_n}
&=(\hat{V}-E^{(1)}_n)|\varphi^{(0)}_n\rangle\rangle\\[5pt]
%
\label{pereq2}
\lambda^2&:&(\epsilon_n-\hat{H}_0)\ket{\varphi^{(2)}_n}
&=(\hat{V}-E^{(1)}_n)\ket{\varphi^{(1)}_n}-E^{(2)}_n|\varphi^{(0)}_n\rangle\rangle\\
%
\label{pereq3}
\lambda^3&:&(\epsilon_n-\hat{H}_0)\ket{\varphi^{(3)}_n}
&=(\hat{V}-E^{(1)}_n)\ket{\varphi^{(2)}_n}-E^{(2)}_n|\varphi^{(1)}_n\rangle-E^{(3)}_n|\varphi^{(0)}_n\rangle\rangle\\
& & &\ \ \vdots\nn[5pt]
%
\label{pereqN}
\lambda^N&:&(\epsilon_n-\hat{H}_0)\ket{\varphi^{(N)}_n}
&=(\hat{V}-E^{(1)}_n)\ket{\varphi^{(N)}_n}-E^{(2)}_n|\varphi^{(1)}_n\rangle-\cdots-E^{(N)}_n|\varphi^{(0)}_n\rangle\rangle
\end{align}


\subsection{基本解法}






\subsection{エネルギー固有値の補正値}
エネルギー固有値の補正値の1次近似について考える.そのために,Eq.~\eqref{pereq1}
\begin{align}
(
\epsilon_n-\hat{H}_0)\ket{\varphi^{(1)}_n}
&=(\hat{V}-E^{(1)}_n)|\varphi^{(0)}_n\rangle\rangle
\end{align}
に左から,$\ket{\varphi^{(0)}_n;\alpha}$を書けると,
\begin{align}\label{pereq1-1_degenerate}
(\epsilon_n-\epsilon_n)\braket{\varphi^{(0)}_n;\alpha|\varphi^{(1)}_n}
&=\langle{\varphi^{(0)}_n;\alpha|\hat{V}|\varphi^{(0)}_n}\rangle\rangle-E^{(1)}_n
\langle \varphi^{(0)}_n;\alpha|\varphi^{(0)}_n\rangle\rangle\\[10pt]
%
(\epsilon_n-\epsilon_n)\braket{\varphi^{(0)}_n;\alpha|\varphi^{(1)}_n}
&=\sum_{\beta=1}^{N}\braket{\varphi^{(0)}_n;\alpha|\hat{V}|\varphi^{(0)}_n;\beta}C_{\beta}
-\sum_{\beta=1}^{N}E^{(1)}_n
\langle \varphi^{(0)}_n;\alpha|\varphi^{(0)}_n;\beta\rangle C_\beta
\end{align}
を得る.ここで,Eq.~\eqref{Eq:dege_state_superposition}を使い,右辺を展開した.左辺第一項は明らかにゼロ,よって
\begin{align}\label{1stpertubation_matrix}
\sum_{\beta=1}^{N}[
E^{(1)}_n \delta_{\alpha,\beta}
-\braket{\varphi^{(0)}_n;\alpha|\hat{V}|\varphi^{(0)}_n;\beta}
]C_\beta=0
\end{align}
となる.この$C_{\beta}$に関する斉1次連立方程式が0以外の解を持つためには,$C_{\beta}$の係数のつくる行列が0でなくてはならない.すなわち,固有値$E^{(1)}_{n}$は以下の特性方程式の解である:
\begin{equation}\label{1st_eigen_eq}
    \det{(E^{(1)}_n \delta_{\alpha,\beta}
    -\braket{\varphi^{(0)}_n;\alpha|\hat{V}|\varphi^{(0)}_n;\beta})}=0
\end{equation}
この固有値方程式は重解も含めて$N$個の解 : $E^{(1)}_{n,\alpha}$, $(\alpha=1,2,\ldots,N)$を持つ.そして,$N$個のそれぞれの解$E^{(1)}_{n,\alpha}$を行列方程式に代入し,規格化条件$\sum_{\beta}|C_{\beta}|=1$のもとで,\eqref{1stpertubation_matrix}を解くことにより,それぞれの解$E^{(1)}_{n,\alpha}$に対する係数が決定する.その係数を改めて$C_{\beta,\alpha}$と書き,これに対応する固有ベクトルを
\begin{equation}
    |\varphi^{(0)}_{n};\alpha\rangle\rangle
    =\sum_{\beta=1}^{N}
    \ket{\varphi^{(0)}_{n};\beta}C_{\beta,\alpha}
\end{equation}
と書き直す.これで第0近似での固有状態を決めることができた.固有状態$|\varphi^{(0)}_{n};\alpha\rangle\rangle$を\eqref{pereq1-1_degenerate}の$|\varphi^{(0)}_{n}\rangle\rangle$へ代入すると,
\begin{align}
\langle{\varphi^{(0)}_n;\gamma|\hat{V}|\varphi^{(0)}_n;\alpha}\rangle\rangle
=E^{(1)}_n
\langle \varphi^{(0)}_n;\gamma|\varphi^{(0)}_n;\alpha\rangle\rangle
\end{align}
を得る.ここで,係数$C_{\gamma,\beta}^{\ast}$をかけて,$\gamma$について和をとると,
\begin{align}
    E^{(1)}_n
    \sum_{\gamma}C_{\gamma,\beta}^{\ast}\langle \varphi^{(0)}_n;\gamma|\varphi^{(0)}_n;\alpha\rangle\rangle
    &=\sum_{\gamma}C_{\gamma,\beta}^{\ast}\langle{\varphi^{(0)}_n;\gamma|\hat{V}|\varphi^{(0)}_n;\alpha}\rangle\rangle\nn[10pt]
    %
    E^{(1)}_n
    \langle\langle \varphi^{(0)}_n;\beta|\varphi^{(0)}_n;\alpha\rangle\rangle
    &=\langle\langle{\varphi^{(0)}_n;\beta|\hat{V}|\varphi^{(0)}_n;\alpha}\rangle\rangle
\end{align}
となる.したがって,
\begin{align}
    E^{(1)}_n\delta_{\alpha,\beta}
    &=\langle\langle{\varphi^{(0)}_n;\beta|\hat{V}|\varphi^{(0)}_n;\alpha}\rangle\rangle\\[10pt]
    \textcolor{red}{
    E^{(1)}_n}
    &=\textcolor{red}{\langle\langle{\varphi^{(0)}_n;\alpha|\hat{V}|\varphi^{(0)}_n;\alpha}\rangle\rangle}
\end{align}
を得る.これが,縮退のある場合のエネルギーの1次の補正項である.




\eqref{1st_eigen_eq}の解に重根がなく,第1次近似で状態$|\varphi^{(0)}_n;\alpha\rangle\rangle$のすべての縮退がとけたたきの第2次近似のエネルギー補正項について計算する.まず\eqref{pereq1}へ,状態ベクトル$\bra{\varphi^{(0)}_{m}}$をかけると,
\begin{align}
    (\epsilon_n-\epsilon_m)\braket{\varphi^{(0)}_{m}|\varphi^{(1)}_n;\alpha}
    &=\braket{\varphi^{(0)}_{m}|\hat{V}|\varphi^{(0)}_n;\alpha}\rangle
    -E^{(1)}_n\langle\varphi^{(0)}_{m}|\varphi^{(0)}_n;\alpha\rangle\rangle
\end{align}
ここで,右辺第二項は
\begin{equation}
    \langle\varphi^{(0)}_{m}|\varphi^{(0)}_n;\alpha\rangle\rangle
    = \sum_{\beta}C_{\beta,\alpha}\langle\varphi^{(0)}_{m}|\varphi^{(0)}_n;\beta\rangle
    =0
\end{equation}
となり,$\epsilon_n\neq\epsilon_m$であるから,
\begin{align}
    \Tilde{C}^{(1)}_{m,\alpha}=\braket{\varphi^{(0)}_{m}|\varphi^{(1)}_n;\alpha}
    &=\frac{\braket{\varphi^{(0)}_{m}|\hat{V}|\varphi^{(0)}_n;\alpha}\rangle}{(\epsilon_n-\epsilon_m)}
\end{align}
を得る.

また,\eqref{pereq2}へ,状態ベクトル$\langle\langle{\varphi^{(0)}_{n};\alpha}|$をかけると,
\begin{align}
    (\epsilon_n-\epsilon_n)\langle\braket{\varphi^{(0)}_{n};\alpha|\varphi^{(2)}_n;\alpha}
    &=\langle\braket{\varphi^{(0)}_{n};\alpha|\hat{V}|\varphi^{(1)}_n;\alpha}
    -E^{(1)}_n\langle\braket{\varphi^{(0)}_{n};\alpha|\varphi^{(1)}_n;\alpha}
    -E^{(2)}_n\langle\langle\varphi^{(0)}_{n};\alpha|\varphi^{(0)}_n;\alpha\rangle\rangle
\end{align}
ここで,左辺は0,右辺第二項は(証明できていないが)0になるので,
% \begin{equation}
%     \braket{\varphi^{(0)}_{m}|\varphi^{(1)}_n}
%     = 
% \end{equation}

\begin{align}
    E^{(2)}_n=
    \langle\braket{\varphi^{(0)}_{n};\alpha|\hat{V}|\varphi^{(1)}_n;\alpha}
\end{align}
となる.これに完全系$\hat{1}=\hat{P}+\hat{Q}$, where $\hat{P}=\sum_{\beta}|\varphi^{(0)}_{n};\beta\rangle\rangle\langle\langle\varphi^{(0)}_{n};\beta|$, $\hat{Q}=\sum_{n\neq m}\ket{\varphi^{(0)}_{m}}\bra{\varphi^{(0)}_{m}}$を挿入し,
\begin{align}
    E^{(2)}_{n,\alpha}&=
    \langle\braket{\varphi^{(0)}_{n};\alpha|\hat{V}\hat{1}|\varphi^{(1)}_n;\alpha}
    =\langle\braket{\varphi^{(0)}_{n};\alpha|\hat{V}\hat{P}|\varphi^{(1)}_n;\alpha}
    +\langle\braket{\varphi^{(0)}_{n};\alpha|\hat{V}\hat{Q}|\varphi^{(1)}_n;\alpha}
    \nn[10pt]
    &=\sum_{\beta=1}^{N}
    \langle\braket{\varphi^{(0)}_{n};\alpha|\hat{V}|\varphi^{(0)}_n;\beta}\rangle
    \langle\braket{\varphi^{(0)}_n;\beta|\varphi^{(1)}_n;\alpha}
    %
    +\sum_{m\neq n}
    \langle\braket{\varphi^{(0)}_{n};\alpha|\hat{V}|\varphi^{(0)}_m}
    \braket{\varphi^{(0)}_m|\varphi^{(1)}_n;\alpha}\nn[10pt]
    &=\sum_{\beta=1}^{N}
    \langle\braket{\varphi^{(0)}_{n};\alpha|\hat{V}|\varphi^{(0)}_n;\beta}\rangle
    C^{(1)}_{\beta,\alpha}
    %
    +\sum_{m\neq n}
    \langle\braket{\varphi^{(0)}_{n};\alpha|\hat{V}|\varphi^{(0)}_m}
    \Tilde{C}^{(1)}_{m,\alpha}
\end{align}
ここで,右辺第一項に
\begin{equation}
    E^{(1)}_{n,\alpha}\delta_{\alpha,\beta}
    =\langle\langle{\varphi^{(0)}_n;\beta|\hat{V}|\varphi^{(0)}_n;\alpha}\rangle\rangle
\end{equation}
を代入すると
\begin{align}
    E^{(2)}_{n,\alpha}
    &=\sum_{\beta=1}^{N}
    E^{(1)}_n\delta_{\alpha,\beta}
    \langle\braket{\varphi^{(0)}_n;\beta|\varphi^{(1)}_n;\alpha}
    %
    +\sum_{m\neq n}
    \langle\braket{\varphi^{(0)}_{n};\alpha|\hat{V}|\varphi^{(0)}_m}
    \braket{\varphi^{(0)}_m|\varphi^{(1)}_n;\alpha}\nn[10pt]
    %
    &=E^{(1)}_{n,\alpha}
    \textcolor{blue}{\langle\braket{\varphi^{(0)}_n;\alpha|\varphi^{(1)}_n;\alpha}}
    %
    +\sum_{m\neq n}
    \langle\braket{\varphi^{(0)}_{n};\alpha|\hat{V}|\varphi^{(0)}_m}
    \textcolor{blue}{\braket{\varphi^{(0)}_m|\varphi^{(1)}_n;\alpha}}\nn[10pt]
    %
    &=E^{(1)}_{n,\alpha}
    \textcolor{blue}{C^{(1)}_{\alpha,\alpha}}
    %
    +\sum_{m\neq n}
    \langle\braket{\varphi^{(0)}_{n};\alpha|\hat{V}|\varphi^{(0)}_m}
    \textcolor{blue}{\Tilde{C}^{(1)}_{m,\alpha}}
\end{align}
を得る.ここで,右辺第一項が0,右辺第二項に
\begin{align}
    \Tilde{C}^{(1)}_{m,\alpha}=\braket{\varphi^{(0)}_{m}|\varphi^{(1)}_n;\alpha}
    &=\frac{\braket{\varphi^{(0)}_{m}|\hat{V}|\varphi^{(0)}_n;\alpha}\rangle}{(\epsilon_n-\epsilon_m)}
\end{align}
を代入することで,
\begin{align}
    E^{(2)}_{n,\alpha}
    &=
    %
    \sum_{m\neq n}
    \frac{\langle\braket{\varphi^{(0)}_{n};\alpha|\hat{V}|\varphi^{(0)}_m}
    \braket{\varphi^{(0)}_{m}|\hat{V}|\varphi^{(0)}_n;\alpha}\rangle}{(\epsilon_n-\epsilon_m)}
    =
    \frac{\langle\braket{\varphi^{(0)}_{n};\alpha|\hat{V}
    \hat{Q}\hat{V}|\varphi^{(0)}_n;\alpha}\rangle}{(\epsilon_n-\epsilon_m)}
\end{align}
を得る.これが第2次のエネルギーの補正項である.以上をまとめると,1次の摂動によって,縮退が解かれるとき,エネルギー固有値は
\begin{align}\label{1st_2nd_energy}
    E_{n,\alpha}
    &=\epsilon_n + \lambda E^{(1)}_{n,\alpha}
    +\lambda^2
    E^{(2)}_{n,\alpha}\nn[10pt]
    &=\epsilon_n + \lambda \langle\langle{\varphi^{(0)}_n;\alpha|\hat{V}|\varphi^{(0)}_n;\alpha}\rangle\rangle
    +\lambda^2 
    \sum_{m\neq n}
    \frac{\langle\braket{\varphi^{(0)}_{n};\alpha|\hat{V}|\varphi^{(0)}_m}
    \braket{\varphi^{(0)}_{m}|\hat{V}|\varphi^{(0)}_n;\alpha}\rangle}{(\epsilon_n-\epsilon_m)}
\end{align}
で与えられる.



\subsection{固有状態について}
ここでは\eqref{1st_2nd_energy}が成立するときの固有状態を第1近似で求める.\eqref{pereq2}に左から$\langle\langle\varphi^{(0)}_{n};\beta|$, $(\beta\neq\alpha)$,をかけると
\begin{align}
    (\epsilon_n-\epsilon_n)\langle\braket{\varphi^{(0)}_{n};\beta|\varphi^{(2)}_n;\alpha}
    &=\langle\braket{\varphi^{(0)}_{n};\beta|\hat{V}|\varphi^{(1)}_n;\alpha}
    -E^{(1)}_n\langle\braket{\varphi^{(0)}_{n};\beta|\varphi^{(1)}_n;\alpha}
    -E^{(2)}_n\langle\langle\varphi^{(0)}_{n};\beta|\varphi^{(0)}_n;\alpha\rangle\rangle
\end{align}
ここで,第1項は消え,第3項もまた$\langle\varphi^{(0)}_{n};\beta|\varphi^{(0)}_n;\alpha\rangle\rangle=0$となるから
\begin{align}
    E^{(1)}_{n,\alpha}\langle\braket{\varphi^{(0)}_{n};\beta|\varphi^{(1)}_n;\alpha}
    &=\langle\braket{\varphi^{(0)}_{n};\beta|\hat{V}|\varphi^{(1)}_n;\alpha}\\[10pt]
    E^{(1)}_{n,\alpha}C^{(1)}_{\beta,\alpha}
    &=\langle\braket{\varphi^{(0)}_{n};\beta|\hat{V}|\varphi^{(1)}_n;\alpha}
\end{align}
ここで右辺に,完全性関係$\hat{1}=\hat{P}+\hat{Q}$を代入すると,
\begin{align}
    E^{(1)}_{n,\alpha}C^{(1)}_{\beta,\alpha}
    &=\langle\braket{\varphi^{(0)}_{n};\beta|\hat{V}|\varphi^{(1)}_n;\alpha}\nn[10pt]
    &=\langle\braket{\varphi^{(0)}_{n};\beta|\hat{V}\hat{P}|\varphi^{(1)}_n;\alpha}
    +\langle\braket{\varphi^{(0)}_{n};\beta|\hat{V}\hat{Q}|\varphi^{(1)}_n;\alpha}\nn[10pt]
    &=\sum_{\gamma=1}^{N}\langle\braket{\varphi^{(0)}_{n};\beta|\hat{V}|\varphi^{(0)}_n;\gamma}\rangle
    \langle\braket{\varphi^{(0)}_{n};\gamma|\varphi^{(1)}_n;\alpha}
    +\sum_{m}\langle\braket{\varphi^{(0)}_{n};\beta|\hat{V}|\varphi^{(0)}_m}
    \braket{\varphi^{(0)}_{m}|\varphi^{(1)}_n;\alpha}\nn[10pt]
    &=\sum_{\gamma=1}^{N}E^{(1)}_{n,\beta}\delta_{\beta,\gamma}
    \langle\braket{\varphi^{(0)}_{n};\gamma|\varphi^{(1)}_n;\alpha}
    +\sum_{m}\langle\braket{\varphi^{(0)}_{n};\beta|\hat{V}|\varphi^{(0)}_m}
    \braket{\varphi^{(0)}_{m}|\varphi^{(1)}_n;\alpha}\nn[10pt]
    &=E^{(1)}_{n,\beta}
    \langle\braket{\varphi^{(0)}_{n};\beta|\varphi^{(1)}_n;\alpha}
    +\sum_{m}\langle\braket{\varphi^{(0)}_{n};\beta|\hat{V}|\varphi^{(0)}_m}
    \braket{\varphi^{(0)}_{m}|\varphi^{(1)}_n;\alpha}\nn[10pt]
    &=E^{(1)}_{n,\beta}
    C^{(1)}_{\beta,\alpha}
    +\sum_{m}\langle\braket{\varphi^{(0)}_{n};\beta|\hat{V}|\varphi^{(0)}_m}
    \Tilde{C}^{(1)}_{m,\alpha}
\end{align}
となる.すなわち,
\begin{equation}
    (E^{(1)}_{n,\alpha}-E^{(1)}_{n,\beta})\langle\braket{\varphi^{(0)}_{n};\beta|\varphi^{(1)}_n;\alpha}
    =\sum_{m}\langle\braket{\varphi^{(0)}_{n};\beta|\hat{V}|\varphi^{(0)}_m}
    \braket{\varphi^{(0)}_{m}|\varphi^{(1)}_n;\alpha}
\end{equation}
を得る.1次の摂動で$\epsilon_n$の縮退はすべて解かれているから,$E^{(1)}_{n,\alpha}\neq E^{(1)}_{n,\beta}$である.したがって,$\beta\neq\alpha$に対して,
\begin{align}
    \Tilde{C}^{(1)}_{m,\alpha}=\braket{\varphi^{(0)}_{m}|\varphi^{(1)}_n;\alpha}
    &=\frac{\braket{\varphi^{(0)}_{m}|\hat{V}|\varphi^{(0)}_n;\alpha}\rangle}{(\epsilon_n-\epsilon_m)}
\end{align}
をより,
\begin{align}
    C^{(1)}_{\beta,\alpha}=\langle\braket{\varphi^{(0)}_{n};\beta|\varphi^{(1)}_n;\alpha}
    &=\sum_{m\neq n}\frac{1}{(E^{(1)}_{n,\alpha}-E^{(1)}_{n,\beta})}
    \langle\braket{\varphi^{(0)}_{n};\beta|\hat{V}|\varphi^{(0)}_m}
    \braket{\varphi^{(0)}_{m}|\varphi^{(1)}_n;\alpha}\nn[10pt]
    &=\sum_{m\neq n}
    \frac{\langle\braket{\varphi^{(0)}_{n};\beta|\hat{V}|\varphi^{(0)}_m}
    \braket{\varphi^{(0)}_{m}|\hat{V}|\varphi^{(0)}_n;\alpha}\rangle}
    {(E^{(1)}_{n,\alpha}-E^{(1)}_{n,\beta})(\epsilon_n-\epsilon_m)},\ \ \ \beta\neq\alpha
\end{align}

よって,固有状態の1次補正は
\begin{align}
    \ket{\varphi_{n,\alpha}^{(1)}}
    &=\hat{P}\ket{\varphi_{n,\alpha}^{(1)}}+\hat{Q}\ket{\varphi_{n,\alpha}^{(1)}}\nn[10pt]
    &=\sum_{\beta=1}^{N}|\varphi^{(0)}_n;\beta\rangle\rangle
    \langle\langle\varphi^{(0)}_n;\beta|\varphi_{n,\alpha}^{(1)}\rangle
    +\sum_{m\neq n}\ket{\varphi^{(0)}_m}\braket{\varphi^{(0)}_m|\varphi_{n,\alpha}^{(1)}}\nn[10pt]
    &=\sum_{\textcolor{red}{\beta\neq\alpha}}|\varphi^{(0)}_n;\beta\rangle\rangle
    \langle\langle\varphi^{(0)}_n;\beta|\varphi_{n,\alpha}^{(1)}\rangle
    +\sum_{m\neq n}\ket{\varphi^{(0)}_m}\braket{\varphi^{(0)}_m|\varphi_{n,\alpha}^{(1)}}\nn[10pt]
    &=\sum_{\textcolor{red}{\beta\neq\alpha}}|\varphi^{(0)}_n;\beta\rangle\rangle
    C^{(1)}_{\beta,\alpha}
    +\sum_{m\neq n}\ket{\varphi^{(0)}_m}\Tilde{C}^{(1)}_{m,\alpha}\nn[10pt]
    &=\sum_{m\neq n}\sum_{\textcolor{red}{\beta\neq\alpha}}|\varphi^{(0)}_n;\beta\rangle\rangle
    \frac{\langle\braket{\varphi^{(0)}_{n};\beta|\hat{V}|\varphi^{(0)}_m}
    \braket{\varphi^{(0)}_{m}|\hat{V}|\varphi^{(0)}_n;\alpha}\rangle}
    {(E^{(1)}_{n,\alpha}-E^{(1)}_{n,\beta})(\epsilon_n-\epsilon_m)}
    +\sum_{m\neq n}\ket{\varphi^{(0)}_m}
    \frac{\braket{\varphi^{(0)}_{m}|\hat{V}|\varphi^{(0)}_n;\alpha}\rangle}{(\epsilon_n-\epsilon_m)}
\end{align}

以上の結果をまとめると,$\lambda$の1次近似で固有状態$\ket{\varphi_{n,\alpha}}$は
\begin{align}
    \ket{\varphi_{n};\alpha}
    &=|\varphi^{(0)}_n;\alpha\rangle\rangle
    +\lambda \ket{\varphi_{n,\alpha}^{(1)}}\\[10pt]
    &=
    |\varphi^{(0)}_n;\alpha\rangle\rangle
    +\lambda\sum_{m\neq n}
    \Biggl\{\sum_{\textcolor{red}{\beta\neq\alpha}}|\varphi^{(0)}_n;\beta\rangle\rangle
    \frac{\langle\braket{\varphi^{(0)}_{n};\beta|\hat{V}|\varphi^{(0)}_m}
    \braket{\varphi^{(0)}_{m}|\hat{V}|\varphi^{(0)}_n;\alpha}\rangle}
    {(E^{(1)}_{n,\alpha}-E^{(1)}_{n,\beta})(\epsilon_n-\epsilon_m)}
    +\ket{\varphi_m^{(0)}}\frac{\braket{\varphi^{(0)}_{m}|\hat{V}|\varphi^{(0)}_n;\alpha}\rangle}{(\epsilon_n-\epsilon_m)}
    \Biggr\}
\end{align}
で与えられる.


\subsection{1次摂動で縮退が解けなかった場合}
1次の摂動によって,縮退が解けなかった場合を考える.\eqref{pereq2}

\subsection{エネルギー固有値の補正値}
1次近似について考える.
\begin{align}
(\epsilon_n-\hat{H}_0)\ket{\varphi^{(2)}_n}
&=(\hat{V}-E^{(1)}_n)\ket{\varphi^{(1)}_n}-E^{(2)}_n|\varphi^{(0)}_n\rangle\rangle
\end{align}
これに,左から$\bra{\varphi^{(0)}_n;\alpha}$を書けると,
\begin{align}\label{pereq2-1_degenerate}
(\epsilon_n-\epsilon_n)\braket{\varphi^{(0)}_n;\alpha|\varphi^{(2)}_n}
&=\langle{\varphi^{(0)}_n;\alpha|\hat{V}|\varphi^{(1)}_n}\rangle
-E^{(1)}_n\langle{\varphi^{(0)}_n;\alpha|\varphi^{(1)}_n}\rangle
-E^{(2)}\langle{\varphi^{(0)}_n;\alpha|\varphi^{(0)}_n}\rangle\rangle\\[10pt]
%
&=\langle{\varphi^{(0)}_n;\alpha|\hat{V}|\varphi^{(1)}_n}\rangle
-E^{(1)}_n\langle{\varphi^{(0)}_n;\alpha|\varphi^{(1)}_n}\rangle
-\sum_{\beta=1}^{N}E^{(2)}_n
\langle \varphi^{(0)}_n;\alpha|\varphi^{(0)}_n;\beta\rangle C_{\beta,\alpha}
\end{align}
左辺第一項は明らかにゼロ,右辺第二項もゼロ,よって
\begin{align}\label{2ndpertubation_matrix}
\sum_{\beta=1}^{N}E^{(2)}_n
\delta_{\alpha,\beta}C_{\beta,\alpha}&=
\langle{\varphi^{(0)}_n;\alpha|\hat{V}|\varphi^{(1)}_n}\rangle
\end{align}
を得る.ここで,右辺を次のように展開する:
\begin{align}
    \langle{\varphi^{(0)}_n;\alpha|\hat{V}|\varphi^{(1)}_n}\rangle
    &=\langle{\varphi^{(0)}_n;\alpha|\hat{V}\hat{1}|\varphi^{(1)}_n}\rangle\nn[10pt]
    &=\langle{\varphi^{(0)}_n;\alpha|\hat{V}\hat{P}|\varphi^{(1)}_n}\rangle\
    + \langle{\varphi^{(0)}_n;\alpha|\hat{V}\hat{Q}|\varphi^{(1)}_n}\rangle\nn[10pt]
    %%%%%%
    &=\sum_{\beta=1}^{N}\langle{\varphi^{(0)}_n;\alpha|\hat{V}|\varphi^{(0)}_n;\beta}\rangle
    \langle{\varphi^{(0)}_n;\beta|\varphi^{(1)}_n}\rangle\
    +\sum_{m\neq n}\langle{\varphi^{(0)}_n;\alpha|\hat{V}|\varphi^{(0)}_m}\rangle
    \langle{\varphi^{(0)}_m|\varphi^{(1)}_n}\rangle\nn[10pt]
    %%%%%%
    &=\sum_{\beta=1}^{N}\langle{\varphi^{(0)}_n;\alpha|\hat{V}|\varphi^{(0)}_n;\beta}\rangle
    \langle{\varphi^{(0)}_n;\beta|\varphi^{(1)}_n;\alpha}\rangle\
    +\sum_{m\neq n}\langle{\varphi^{(0)}_n;\alpha|\hat{V}|\varphi^{(0)}_m}\rangle
    \textcolor{red}{\langle{\varphi^{(0)}_m|\varphi^{(1)}_n;\alpha}\rangle}\nn[10pt]
    %%%%%%
    &=\sum_{\beta=1}^{N}E^{(1)}_{n}\delta_{\alpha,\beta}
    \langle{\varphi^{(0)}_n;\beta|\varphi^{(1)}_n;\alpha}\rangle\
    +\sum_{m\neq n}
    \frac{\langle{\varphi^{(0)}_n;\alpha|\hat{V}|\varphi^{(0)}_m}\rangle
    \braket{\varphi^{(0)}_{m}|\hat{V}|\varphi^{(0)}_n;\alpha}\rangle}{(\epsilon_n-\epsilon_m)}\nn[10pt]
    %%%%%%
    &=\sum_{\beta=1}^{N}E^{(1)}_{n,\alpha}
    \langle{\varphi^{(0)}_n;\alpha|\varphi^{(1)}_n;\alpha}\rangle\
    +\sum_{m\neq n}
    \frac{\langle{\varphi^{(0)}_n;\alpha|\hat{V}|\varphi^{(0)}_m}\rangle
    \braket{\varphi^{(0)}_{m}|\hat{V}|\varphi^{(0)}_n;\alpha}\rangle}{(\epsilon_n-\epsilon_m)}
\end{align}
ここで,右辺第一項は0,右辺第二項は
\begin{align}
    \sum_{m\neq n}
    \frac{\langle{\varphi^{(0)}_n;\alpha|\hat{V}|\varphi^{(0)}_m}\rangle
    \braket{\varphi^{(0)}_{m}|\hat{V}|\varphi^{(0)}_n;\alpha}\rangle}{(\epsilon_n-\epsilon_m)}
    &= \sum_{m\neq n}
    \frac{\langle{\varphi^{(0)}_n;\alpha|\hat{V}|\varphi^{(0)}_m}\rangle
    \langle{\varphi^{(0)}_{m}|\hat{V}|\Bigl(\sum_{\beta=1}^{N}|\varphi^{(0)}_n;\beta}\rangle C_{\beta,\alpha}\Bigr)}
    {(\epsilon_n-\epsilon_m)}\nn[10pt]
    &=\sum_{\beta=1}^{N} \sum_{m\neq n}
    \frac{\langle{\varphi^{(0)}_n;\alpha|\hat{V}|\varphi^{(0)}_m}\rangle
    \langle{\varphi^{(0)}_{m}|\hat{V}|\varphi^{(0)}_n;\beta}\rangle}
    {(\epsilon_n-\epsilon_m)}C_{\beta,\alpha}
\end{align}
以上より,

\begin{align}\label{2ndpertubation_matrix}
\sum_{\beta=1}^{N}\Biggl[
E^{(2)}_n \delta_{\alpha,\beta}
-(\hat{V}^{(2)})_{\alpha,\beta}
\biggr]C_{\beta,\alpha}=0,
\end{align}
ここで,
\begin{equation}
    (\hat{V}^{(2)})_{\alpha,\beta}
    \equiv\sum_{m\neq n}
    \frac{\langle{\varphi^{(0)}_n;\alpha|\hat{V}|\varphi^{(0)}_m}\rangle
    \langle{\varphi^{(0)}_{m}|\hat{V}|\varphi^{(0)}_n;\beta}\rangle}
    {(\epsilon_n-\epsilon_m)}
\end{equation}
この$C_{\beta}$に関する斉1次連立方程式が0以外の解を持つためには,$C_{\beta}$の係数のつくる行列が0でなくてはならない.すなわち,固有値$E^{(2)}_{n}$は以下の特性方程式の解である:
\begin{equation}\label{2nd_eigen_eq}
    \det{(E^{(2)}_n \delta_{\alpha,\beta}
    -(\hat{V}^{(2)})_{\alpha,\beta})}=0
\end{equation}
この固有値方程式は重解も含めて$N$個の解 : $E^{(2)}_{n,\alpha}$, $(\alpha=1,2,\ldots,N)$を持つ.そして,$N$個のそれぞれの解$E^{(2)}_{n,\alpha}$を行列方程式に代入し,規格化条件$\sum_{\beta}|C_{\beta}|=1$のもとで,\eqref{2ndpertubation_matrix}を解くことにより,それぞれの解$E^{(2)}_{n,\alpha}$に対する係数が決定する.その係数を改めて$C_{\beta,\alpha}$と書き,これに対応する固有ベクトルを
\begin{equation}
    |\varphi^{(0)}_{n};\alpha\rangle\rangle
    =\sum_{\beta=1}^{N}
    \ket{\varphi^{(0)}_{n};\beta}C_{\beta,\alpha}
\end{equation}
と書き直す.これで第0近似での固有状態を決めることができた.固有状態$|\varphi^{(0)}_{n};\alpha\rangle\rangle$を\eqref{pereq2-1_degenerate}の$|\varphi^{(0)}_{n}\rangle\rangle$へ代入すると,
\begin{align}
(\epsilon_n-\epsilon_n)\braket{\varphi^{(0)}_n;\gamma|\varphi^{(2)}_n;\alpha}
&=\langle{\varphi^{(0)}_n;\gamma|\hat{V}|\varphi^{(1)}_n;\alpha}\rangle
-E^{(1)}_n\langle{\varphi^{(0)}_n;\gamma|\varphi^{(1)}_n;\alpha}\rangle
-E^{(2)}\langle{\varphi^{(0)}_n;\gamma|\varphi^{(0)}_n;\alpha}\rangle\rangle
%
\end{align}
を得る.左辺と右辺第二項は$\braket{\varphi^{(0)}_n;\alpha|\varphi^{(k)}_n;\alpha}=0$より,ゼロ.したがって,
\begin{align}
E^{(2)}_n\langle{\varphi^{(0)}_n;\alpha|\varphi^{(0)}_n;\alpha}\rangle\rangle
&=\langle{\varphi^{(0)}_n;\alpha|\hat{V}|\varphi^{(1)}_n;\alpha}\rangle
-E^{(1)}_n\langle{\varphi^{(0)}_n;\gamma|\varphi^{(1)}_n;\alpha}\rangle
%
\end{align}
右辺第二項を次のように展開する:
\begin{align}
\langle{\varphi^{(0)}_n;\alpha|\hat{V}|\varphi^{(1)}_n;\alpha}\rangle
&=\langle{\varphi^{(0)}_n;\alpha|\hat{V}\hat{1}|\varphi^{(1)}_n;\alpha}\rangle\nn[10pt]
&=\langle{\varphi^{(0)}_n;\alpha|\hat{V}\hat{P}|\varphi^{(1)}_n;\alpha}\rangle
+\langle{\varphi^{(0)}_n;\alpha|\hat{V}\hat{Q}|\varphi^{(1)}_n;\alpha}\rangle\nn[10pt]
%
&=\sum_{\beta=1}^{N}\langle{\varphi^{(0)}_n;\gamma|\hat{V}|\varphi^{(0)}_n;\beta}\rangle
\langle{\varphi^{(0)}_n;\beta|\varphi^{(1)}_n;\alpha}\rangle
+\sum_{m\neq n}\langle{\varphi^{(0)}_n;\gamma|\hat{V}|\varphi^{(0)}_m}\rangle
\langle{\varphi^{(0)}_m|\varphi^{(1)}_n;\alpha}\rangle\nn[10pt]
%
&=\sum_{\beta=1}^{N}E^{(1)}_n \delta_{\gamma,\beta}
\langle{\varphi^{(0)}_n;\beta|\varphi^{(1)}_n;\alpha}\rangle
+\sum_{m\neq n}\frac{\langle{\varphi^{(0)}_n;\gamma|\hat{V}|\varphi^{(0)}_m}\rangle
    \braket{\varphi^{(0)}_{m}|\hat{V}|\varphi^{(0)}_n;\alpha}\rangle}{(\epsilon_n-\epsilon_m)}\nn[10pt]
%
&=E^{(1)}_n 
\langle{\varphi^{(0)}_n;\gamma|\varphi^{(1)}_n;\alpha}\rangle
+\sum_{m\neq n}\frac{\langle{\varphi^{(0)}_n;\gamma|\hat{V}|\varphi^{(0)}_m}\rangle
    \braket{\varphi^{(0)}_{m}|\hat{V}|\varphi^{(0)}_n;\alpha}\rangle}{(\epsilon_n-\epsilon_m)}
\end{align}
したがって,$E^{1}_{n}$が重解のとき,

\begin{equation}
    E^{(2)}_n \langle{\varphi^{(0)}_n;\alpha|\varphi^{(0)}_n;\alpha}\rangle\rangle
    =\sum_{m\neq n}\frac{\langle{\varphi^{(0)}_n;\gamma|\hat{V}|\varphi^{(0)}_m}\rangle
    \braket{\varphi^{(0)}_{m}|\hat{V}|\varphi^{(0)}_n;\alpha}\rangle}{(\epsilon_n-\epsilon_m)}
\end{equation}

ここで,係数$C_{\gamma,\beta}^{\ast}$をかけて,$\gamma$について和をとると,
\begin{align}
    E^{(2)}_n
    \sum_{\gamma}C_{\gamma,\beta}^{\ast}\langle \varphi^{(0)}_n;\gamma|\varphi^{(0)}_n;\alpha\rangle\rangle
    &=\sum_{\gamma}C_{\gamma,\beta}^{\ast}
    \sum_{m\neq n}\frac{\langle{\varphi^{(0)}_n;\gamma|\hat{V}|\varphi^{(0)}_m}\rangle
    \braket{\varphi^{(0)}_{m}|\hat{V}|\varphi^{(0)}_n;\alpha}\rangle}{(\epsilon_n-\epsilon_m)}\nn[10pt]
    %
    E^{(2)}_n
    \langle\langle \varphi^{(0)}_n;\beta|\varphi^{(0)}_n;\alpha\rangle\rangle
    &=\sum_{m\neq n}\frac{\langle\langle{\varphi^{(0)}_n;\beta|\hat{V}|\varphi^{(0)}_m}\rangle
    \braket{\varphi^{(0)}_{m}|\hat{V}|\varphi^{(0)}_n;\alpha}\rangle}{(\epsilon_n-\epsilon_m)}
\end{align}
となる.したがって,
\begin{align}
    E^{(2)}_n\delta_{\alpha,\beta}
    &=\sum_{m\neq n}\frac{\langle\langle{\varphi^{(0)}_n;\beta|\hat{V}|\varphi^{(0)}_m}\rangle
    \braket{\varphi^{(0)}_{m}|\hat{V}|\varphi^{(0)}_n;\alpha}\rangle}{(\epsilon_n-\epsilon_m)}\\[10pt]
    \textcolor{red}{
    E^{(2)}_n}
    &=\textcolor{red}{\sum_{m\neq n}\frac{\langle\langle{\varphi^{(0)}_n;\alpha|\hat{V}|\varphi^{(0)}_m}\rangle
    \braket{\varphi^{(0)}_{m}|\hat{V}|\varphi^{(0)}_n;\alpha}\rangle}{(\epsilon_n-\epsilon_m)}}
\end{align}
を得る.これが,1次で縮退が解けなかった場合における,縮退のある場合のエネルギーの2次の補正項である.



\subsection{}
\eqref{2nd_eigen_eq}の解に重根がなく,第2次近似で状態$|\varphi^{(0)}_n;\alpha\rangle\rangle$のすべての縮退がとけたたきの第3次近似のエネルギー補正項について計算する.まず\eqref{pereq2}へ,状態ベクトル$\bra{\varphi^{(0)}_{m}}$をかけると,
\begin{align}
    (\epsilon_n-\epsilon_m)\braket{\varphi^{(0)}_{m}|\varphi^{(2)}_n;\alpha}
    &=\braket{\varphi^{(0)}_{m}|\hat{V}|\varphi^{(1)}_n;\alpha}
    -E^{(1)}_n\langle\varphi^{(0)}_{m}|\varphi^{(1)}_n;\alpha\rangle
    -E^{(2)}_n\langle\varphi^{(0)}_{m}|\varphi^{(0)}_n;\alpha\rangle\rangle
\end{align}
ここで,右辺第3項は
\begin{equation}
    \langle\varphi^{(0)}_{m}|\varphi^{(0)}_n;\alpha\rangle\rangle
    = \sum_{\beta}C_{\beta,\alpha}\langle\varphi^{(0)}_{m}|\varphi^{(0)}_n;\beta\rangle
    =0
\end{equation}
となり,$\epsilon_n\neq\epsilon_m$であるから,
\begin{align}
    (\epsilon_n-\epsilon_m)\braket{\varphi^{(0)}_{m}|\varphi^{(2)}_n;\alpha}
    &=\braket{\varphi^{(0)}_{m}|\hat{V}|\varphi^{(1)}_n;\alpha}
    -E^{(1)}_n\Tilde{C}^{(1)}_{m,\alpha}
    %-E^{(2)}_{n,\alpha}\langle\varphi^{(0)}_{m}|\varphi^{(0)}_n;\alpha\rangle\rangle
\end{align}
ここで,右辺を次のように展開すると
\begin{align}
    \braket{\varphi^{(0)}_{m}|\hat{V}|\varphi^{(1)}_n;\alpha}
    &=\braket{\varphi^{(0)}_{m}|\hat{V}\hat{1}|\varphi^{(1)}_n;\alpha}\nn[10pt]
    &=\braket{\varphi^{(0)}_{m}|\hat{V}\hat{P}|\varphi^{(1)}_n;\alpha}
    +\braket{\varphi^{(0)}_{m}|\hat{V}\hat{P}|\varphi^{(1)}_n;\alpha}\nn[10pt]
    &=\sum_{\beta=1}^N\langle{\varphi^{(0)}_{m}|\hat{V}|\varphi^{(0)}_n;\beta}\rangle\rangle
    \langle\langle{\varphi^{(0)}_n;\beta|\varphi^{(1)}_n;\alpha}\rangle
    +\sum_{p\neq n}\langle{\varphi^{(0)}_{m}|\hat{V}|\varphi^{(0)}_p}\rangle
    \textcolor{red}{\langle{\varphi^{(0)}_p|\varphi^{(1)}_n;\alpha}\rangle}
\end{align}
ここで,
\begin{align}
    \textcolor{red}{\Tilde{C}^{(1)}_{p,\alpha}=\langle{\varphi^{(0)}_p|\varphi^{(1)}_n;\alpha}\rangle
    =\frac{\braket{\varphi^{(0)}_{p}|\hat{V}|\varphi^{(0)}_n;\alpha}\rangle}{(\epsilon_n-\epsilon_p)}
    }
\end{align}
を使うと,
\begin{align}
    \braket{\varphi^{(0)}_{m}|\hat{V}|\varphi^{(1)}_n;\alpha}
    &=\sum_{\beta=1}^N\langle{\varphi^{(0)}_{m}|\hat{V}|\varphi^{(0)}_n;\beta}\rangle\rangle
    \langle\langle{\varphi^{(0)}_n;\beta|\varphi^{(1)}_n;\alpha}\rangle
    +\sum_{p\neq n}
    \frac{\langle{\varphi^{(0)}_{m}|\hat{V}|\varphi^{(0)}_p}\rangle
    \braket{\varphi^{(0)}_{p}|\hat{V}|\varphi^{(0)}_n;\alpha}\rangle}{(\epsilon_n-\epsilon_p)}
\end{align}
となる.したがって,
\begin{align}
    \textcolor{red}{\Tilde{C}^{(2)}_{m,\alpha}}
    &=\braket{\varphi^{(0)}_{m}|\varphi^{(2)}_n;\alpha}\nn[10pt]
    &=\textcolor{red}{\sum_{\beta=1}^N
    \frac{\langle{\varphi^{(0)}_{m}|\hat{V}|\varphi^{(0)}_n;\beta}\rangle\rangle
    \langle\langle{\varphi^{(0)}_n;\beta|\varphi^{(1)}_n;\alpha}\rangle}{(\epsilon_n-\epsilon_m)}
    +\sum_{p\neq n}
    \frac{\langle{\varphi^{(0)}_{m}|\hat{V}|\varphi^{(0)}_p}\rangle
    \braket{\varphi^{(0)}_{p}|\hat{V}|\varphi^{(0)}_n;\alpha}\rangle}
    {(\epsilon_n-\epsilon_m)(\epsilon_n-\epsilon_p)}
    -E^{(1)}_n\frac{\Tilde{C}^{(1)}_{m,\alpha}}{(\epsilon_n-\epsilon_p)}}
\end{align}
を得る.

% また,\eqref{pereq2}へ,状態ベクトル$\langle\langle{\varphi^{(0)}_{n};\alpha}|$をかけると,
% \begin{align}
%     (\epsilon_n-\epsilon_n)\langle\braket{\varphi^{(0)}_{n};\alpha|\varphi^{(2)}_n;\alpha}
%     &=\langle\braket{\varphi^{(0)}_{n};\alpha|\hat{V}|\varphi^{(1)}_n;\alpha}
%     -E^{(1)}_n\langle\braket{\varphi^{(0)}_{n};\alpha|\varphi^{(1)}_n;\alpha}
%     -E^{(2)}_n\langle\langle\varphi^{(0)}_{n};\alpha|\varphi^{(0)}_n;\alpha\rangle\rangle
% \end{align}
% ここで,左辺は0,右辺第二項は(証明できていないが)0になるので,
% % \begin{equation}
% %     \braket{\varphi^{(0)}_{m}|\varphi^{(1)}_n}
% %     = 
% % \end{equation}

% \begin{align}
%     E^{(2)}_n=
%     \langle\braket{\varphi^{(0)}_{n};\alpha|\hat{V}|\varphi^{(1)}_n;\alpha}
% \end{align}
% となる.これに完全系$\hat{1}=\hat{P}+\hat{Q}$, where $\hat{P}=\sum_{\beta}|\varphi^{(0)}_{n};\beta\rangle\rangle\langle\langle\varphi^{(0)}_{n};\beta|$, $\hat{Q}=\sum_{n\neq m}\ket{\varphi^{(0)}_{m}}\bra{\varphi^{(0)}_{m}}$を挿入し,
% \begin{align}
%     E^{(2)}_{n,\alpha}&=
%     \langle\braket{\varphi^{(0)}_{n};\alpha|\hat{V}\hat{1}|\varphi^{(1)}_n;\alpha}
%     =\langle\braket{\varphi^{(0)}_{n};\alpha|\hat{V}\hat{P}|\varphi^{(1)}_n;\alpha}
%     +\langle\braket{\varphi^{(0)}_{n};\alpha|\hat{V}\hat{Q}|\varphi^{(1)}_n;\alpha}
%     \nn[10pt]
%     &=\sum_{\beta=1}^{N}
%     \langle\braket{\varphi^{(0)}_{n};\alpha|\hat{V}|\varphi^{(0)}_n;\beta}\rangle
%     \langle\braket{\varphi^{(0)}_n;\beta|\varphi^{(1)}_n;\alpha}
%     %
%     +\sum_{m\neq n}
%     \langle\braket{\varphi^{(0)}_{n};\alpha|\hat{V}|\varphi^{(0)}_m}
%     \braket{\varphi^{(0)}_m|\varphi^{(1)}_n;\alpha}\nn[10pt]
%     &=\sum_{\beta=1}^{N}
%     \langle\braket{\varphi^{(0)}_{n};\alpha|\hat{V}|\varphi^{(0)}_n;\beta}\rangle
%     C^{(1)}_{\beta,\alpha}
%     %
%     +\sum_{m\neq n}
%     \langle\braket{\varphi^{(0)}_{n};\alpha|\hat{V}|\varphi^{(0)}_m}
%     \Tilde{C}^{(1)}_{m,\alpha}
% \end{align}
% ここで,右辺第一項に
% \begin{equation}
%     E^{(1)}_{n,\alpha}\delta_{\alpha,\beta}
%     =\langle\langle{\varphi^{(0)}_n;\beta|\hat{V}|\varphi^{(0)}_n;\alpha}\rangle\rangle
% \end{equation}
% を代入すると
% \begin{align}
%     E^{(2)}_{n,\alpha}
%     &=\sum_{\beta=1}^{N}
%     E^{(1)}_n\delta_{\alpha,\beta}
%     \langle\braket{\varphi^{(0)}_n;\beta|\varphi^{(1)}_n;\alpha}
%     %
%     +\sum_{m\neq n}
%     \langle\braket{\varphi^{(0)}_{n};\alpha|\hat{V}|\varphi^{(0)}_m}
%     \braket{\varphi^{(0)}_m|\varphi^{(1)}_n;\alpha}\nn[10pt]
%     %
%     &=E^{(1)}_{n,\alpha}
%     \textcolor{blue}{\langle\braket{\varphi^{(0)}_n;\alpha|\varphi^{(1)}_n;\alpha}}
%     %
%     +\sum_{m\neq n}
%     \langle\braket{\varphi^{(0)}_{n};\alpha|\hat{V}|\varphi^{(0)}_m}
%     \textcolor{blue}{\braket{\varphi^{(0)}_m|\varphi^{(1)}_n;\alpha}}\nn[10pt]
%     %
%     &=E^{(1)}_{n,\alpha}
%     \textcolor{blue}{C^{(1)}_{\alpha,\alpha}}
%     %
%     +\sum_{m\neq n}
%     \langle\braket{\varphi^{(0)}_{n};\alpha|\hat{V}|\varphi^{(0)}_m}
%     \textcolor{blue}{\Tilde{C}^{(1)}_{m,\alpha}}
% \end{align}
% を得る.ここで,右辺第一項が0,右辺第二項に
% \begin{align}
%     \Tilde{C}^{(1)}_{m,\alpha}=\braket{\varphi^{(0)}_{m}|\varphi^{(1)}_n;\alpha}
%     &=\frac{\braket{\varphi^{(0)}_{m}|\hat{V}|\varphi^{(0)}_n;\alpha}\rangle}{(\epsilon_n-\epsilon_m)}
% \end{align}
% を代入することで,
% \begin{align}
%     E^{(2)}_{n,\alpha}
%     &=
%     %
%     \sum_{m\neq n}
%     \frac{\langle\braket{\varphi^{(0)}_{n};\alpha|\hat{V}|\varphi^{(0)}_m}
%     \braket{\varphi^{(0)}_{m}|\hat{V}|\varphi^{(0)}_n;\alpha}\rangle}{(\epsilon_n-\epsilon_m)}
%     =
%     \frac{\langle\braket{\varphi^{(0)}_{n};\alpha|\hat{V}
%     \hat{Q}\hat{V}|\varphi^{(0)}_n;\alpha}\rangle}{(\epsilon_n-\epsilon_m)}
% \end{align}
% を得る.これが第2次のエネルギーの補正項である.以上をまとめると,1次の摂動によって,縮退が解かれるとき,エネルギー固有値は
% \begin{align}\label{1st_2nd_3rd_energy}
%     E_{n,\alpha}
%     &=\epsilon_n + \lambda E^{(1)}_{n,\alpha}
%     +\lambda^2
%     E^{(2)}_{n,\alpha}\nn[10pt]
%     &=\epsilon_n + \lambda \langle\langle{\varphi^{(0)}_n;\alpha|\hat{V}|\varphi^{(0)}_n;\alpha}\rangle\rangle
%     +\lambda^2 
%     \sum_{m\neq n}
%     \frac{\langle\braket{\varphi^{(0)}_{n};\alpha|\hat{V}|\varphi^{(0)}_m}
%     \braket{\varphi^{(0)}_{m}|\hat{V}|\varphi^{(0)}_n;\alpha}\rangle}{(\epsilon_n-\epsilon_m)}
% \end{align}
% で与えられる.



\subsection{固有状態について}
ここでは\eqref{1st_2nd_3rd_energy}が成立するときの固有状態を第2近似で求める.\eqref{pereq3}に左から$\langle\langle\varphi^{(0)}_{n};\beta|$, $(\beta\neq\alpha)$,をかけると
\begin{align}
    (\epsilon_n-\epsilon_n)\langle\braket{\varphi^{(0)}_{n};\beta|\varphi^{(3)}_n;\alpha}
    &=\langle\braket{\varphi^{(0)}_{n};\beta|\hat{V}|\varphi^{(2)}_n;\alpha}
    -E^{(1)}_{n,\alpha}\langle\braket{\varphi^{(0)}_{n};\beta|\varphi^{(2)}_n;\alpha}\nn[10pt]
    &\hspace{50pt}
    -E^{(2)}_{n,\alpha}\langle\langle\varphi^{(0)}_{n};\beta|\varphi^{(1)}_n;\alpha\rangle
    -E^{(3)}_{n,\alpha}\langle\langle\varphi^{(0)}_{n};\beta|\varphi^{(0)}_n;\alpha\rangle\rangle
\end{align}
ここで,%第2項は消え,
第4項もまた$\langle\varphi^{(0)}_{n};\beta|\varphi^{(0)}_n;\alpha\rangle\rangle=0$となるから
\begin{align}
    E^{(2)}_{n,\alpha}\langle\langle\varphi^{(0)}_{n};\beta|\varphi^{(1)}_n;\alpha\rangle
    &=\langle\braket{\varphi^{(0)}_{n};\beta|\hat{V}|\varphi^{(2)}_n;\alpha}
    -E^{(1)}_{n,\alpha}\langle\braket{\varphi^{(0)}_{n};\beta|\varphi^{(2)}_n;\alpha}\\[10pt]
    %
    E^{(2)}_{n,\alpha}C^{(1)}_{\beta,\alpha}
    &=\langle\braket{\varphi^{(0)}_{n};\beta|\hat{V}|\varphi^{(2)}_n;\alpha}
    -E^{(1)}_{n,\alpha}C^{(2)}_{\beta,\alpha}
\end{align}
ここで右辺第一項に,完全性関係$\hat{1}=\hat{P}+\hat{Q}$を代入すると,
\begin{align}
    \langle\braket{\varphi^{(0)}_{n};\beta|\hat{V}|\varphi^{(2)}_n;\alpha}
    &=\langle\braket{\varphi^{(0)}_{n};\beta|\hat{V}\hat{1}|\varphi^{(2)}_n;\alpha}\nn[10pt]
    &=\langle\braket{\varphi^{(0)}_{n};\beta|\hat{V}\hat{P}|\varphi^{(2)}_n;\alpha}
    +\langle\braket{\varphi^{(0)}_{n};\beta|\hat{V}\hat{Q}|\varphi^{(2)}_n;\alpha}\nn[10pt]
    &=\sum_{\gamma=1}^{N}\langle\braket{\varphi^{(0)}_{n};\beta|\hat{V}|\varphi^{(0)}_n;\gamma}\rangle
    \langle\braket{\varphi^{(0)}_{n};\gamma|\varphi^{(2)}_n;\alpha}
    +\sum_{m\neq n}\langle\braket{\varphi^{(0)}_{n};\beta|\hat{V}|\varphi^{(0)}_m}
    \braket{\varphi^{(0)}_{m}|\varphi^{(2)}_n;\alpha}\nn[10pt]
    %
    &=\sum_{\gamma=1}^{N}E^{(1)}_{n,\beta}\delta_{\beta,\gamma}
    \langle\braket{\varphi^{(0)}_{n};\gamma|\varphi^{(2)}_n;\alpha}
    +\sum_{m}\langle\braket{\varphi^{(0)}_{n};\beta|\hat{V}|\varphi^{(0)}_m}
    \braket{\varphi^{(0)}_{m}|\varphi^{(2)}_n;\alpha}\nn[10pt]
    %
    &=E^{(1)}_{n,\gamma}
    \langle\braket{\varphi^{(0)}_{n};\gamma|\varphi^{(2)}_n;\alpha}
    +\sum_{m}\langle\braket{\varphi^{(0)}_{n};\beta|\hat{V}|\varphi^{(0)}_m}
    \braket{\varphi^{(0)}_{m}|\varphi^{(2)}_n;\alpha}\nn[10pt]
    &=E^{(1)}_{n,\gamma}
    C^{(2)}_{\gamma,\alpha}
    +\sum_{m}\langle\braket{\varphi^{(0)}_{n};\beta|\hat{V}|\varphi^{(0)}_m}
    \Tilde{C}^{(2)}_{m,\alpha}
\end{align}
となる.$E^{(1)}_{n}=0$を仮定すると

\begin{align}
    %
    E^{(2)}_{n,\alpha}C^{(1)}_{\beta,\alpha}
    &=%E^{(1)}_{n,\gamma}C^{(2)}_{\gamma,\alpha}
    \sum_{m}\langle\braket{\varphi^{(0)}_{n};\beta|\hat{V}|\varphi^{(0)}_m}
    \Tilde{C}^{(2)}_{m,\alpha}
    %-E^{(1)}_{n,\alpha}C^{(2)}_{\beta,\alpha}
\end{align}
を得る.上式に
\begin{align}
    \textcolor{red}{\Tilde{C}^{(2)}_{m,\alpha}}
    &=\braket{\varphi^{(0)}_{m}|\varphi^{(2)}_n;\alpha}\nn[10pt]
    &=\textcolor{red}{\sum_{\beta=1}^N
    \frac{\langle{\varphi^{(0)}_{m}|\hat{V}|\varphi^{(0)}_n;\beta}\rangle\rangle
    \langle\langle{\varphi^{(0)}_n;\beta|\varphi^{(1)}_n;\alpha}\rangle}{(\epsilon_n-\epsilon_m)}
    +\sum_{p\neq n}
    \frac{\langle{\varphi^{(0)}_{m}|\hat{V}|\varphi^{(0)}_p}\rangle
    \braket{\varphi^{(0)}_{p}|\hat{V}|\varphi^{(0)}_n;\alpha}\rangle}
    {(\epsilon_n-\epsilon_m)(\epsilon_n-\epsilon_p)}
    -E^{(1)}_n\frac{\Tilde{C}^{(1)}_{m,\alpha}}{(\epsilon_n-\epsilon_p)}}
\end{align}
を代入すると
\begin{align}
    %
    E^{(2)}_{n,\alpha}C^{(1)}_{\beta,\alpha}
    &=%E^{(1)}_{n,\gamma}C^{(2)}_{\gamma,\alpha}
    \sum_{m\neq n}\langle\braket{\varphi^{(0)}_{n};\gamma|\hat{V}|\varphi^{(0)}_m}\nn[10pt]
    &\hspace{30pt}\times\Biggl[\sum_{\beta=1}^N
    \frac{\langle{\varphi^{(0)}_{m}|\hat{V}|\varphi^{(0)}_n;\beta}\rangle\rangle
    \langle\langle{\varphi^{(0)}_n;\beta|\varphi^{(1)}_n;\alpha}\rangle}{(\epsilon_n-\epsilon_m)}
    +\sum_{p\neq n}
    \frac{\langle{\varphi^{(0)}_{m}|\hat{V}|\varphi^{(0)}_p}\rangle
    \braket{\varphi^{(0)}_{p}|\hat{V}|\varphi^{(0)}_n;\alpha}\rangle}
    {(\epsilon_n-\epsilon_m)(\epsilon_n-\epsilon_p)}
    -E^{(1)}_n\frac{\Tilde{C}^{(1)}_{m,\alpha}}{(\epsilon_n-\epsilon_p)}
    \Biggr]\nn[10pt]
    %
    &=
    \sum_{m\neq n}
    \Biggl[\sum_{\beta=1}^N
    \frac{\langle\braket{\varphi^{(0)}_{n};\gamma|\hat{V}|\varphi^{(0)}_m}
    \langle{\varphi^{(0)}_{m}|\hat{V}|\varphi^{(0)}_n;\beta}\rangle\rangle
    \langle\langle{\varphi^{(0)}_n;\beta|\varphi^{(1)}_n;\alpha}\rangle}{(\epsilon_n-\epsilon_m)}\nn[10pt]
    &\hspace{70pt}+\sum_{p\neq n}
    \frac{\langle\braket{\varphi^{(0)}_{n};\gamma|\hat{V}|\varphi^{(0)}_m}
    \langle{\varphi^{(0)}_{m}|\hat{V}|\varphi^{(0)}_p}\rangle
    \braket{\varphi^{(0)}_{p}|\hat{V}|\varphi^{(0)}_n;\alpha}\rangle}
    {(\epsilon_n-\epsilon_m)(\epsilon_n-\epsilon_p)}\nn[10pt]
    %
    &\hspace{100pt}
    -E^{(1)}_n
    \frac{\langle\braket{\varphi^{(0)}_{n};\gamma|\hat{V}|\varphi^{(0)}_m}
    \braket{\varphi^{(0)}_{m}|\hat{V}|\varphi^{(0)}_n;\alpha}\rangle}{(\epsilon_n-\epsilon_m)^2}
    \Biggr]\\[20pt]
    %
    %
    %
    %
    &=
    \sum_{m\neq n}
    \Biggl[\sum_{\beta=1}^N
    \frac{\langle\braket{\varphi^{(0)}_{n};\gamma|\hat{V}|\varphi^{(0)}_m}
    \langle{\varphi^{(0)}_{m}|\hat{V}|\varphi^{(0)}_n;\beta}\rangle\rangle
    \langle\langle{\varphi^{(0)}_n;\beta|\varphi^{(1)}_n;\alpha}\rangle}{(\epsilon_n-\epsilon_m)}\nn[10pt]
    &\hspace{70pt}+\sum_{p\neq n}
    \frac{\langle\braket{\varphi^{(0)}_{n};\gamma|\hat{V}|\varphi^{(0)}_m}
    \langle{\varphi^{(0)}_{m}|\hat{V}|\varphi^{(0)}_p}\rangle
    \braket{\varphi^{(0)}_{p}|\hat{V}|\varphi^{(0)}_n;\alpha}\rangle}
    {(\epsilon_n-\epsilon_m)(\epsilon_n-\epsilon_p)}\Biggr]\nn[10pt]
    %%
    &\hspace{100pt}
    -E^{(1)}_n
    \frac{\langle\braket{\varphi^{(0)}_{n};\gamma|\hat{V}|\varphi^{(0)}_m}
    \braket{\varphi^{(0)}_{m}|\hat{V}|\varphi^{(0)}_n;\alpha}\rangle}{(\epsilon_n-\epsilon_m)^2}
    \Biggr]
\end{align}
ここで,右辺第一項は
\begin{align}
    &\sum_{m\neq n}
    \sum_{\beta=1}^N
    \frac{\langle\braket{\varphi^{(0)}_{n};\gamma|\hat{V}|\varphi^{(0)}_m}
    \langle{\varphi^{(0)}_{m}|\hat{V}|\varphi^{(0)}_n;\beta}\rangle\rangle
    \langle\langle{\varphi^{(0)}_n;\beta|\varphi^{(1)}_n;\alpha}\rangle}{(\epsilon_n-\epsilon_m)}\nn[10pt]
    &=
    \sum_{\beta=1}^N
    \textcolor{red}{
    \sum_{m\neq n}
    \frac{\langle\braket{\varphi^{(0)}_{n};\gamma|\hat{V}|\varphi^{(0)}_m}
    \langle{\varphi^{(0)}_{m}|\hat{V}|\varphi^{(0)}_n;\beta}\rangle\rangle
    }{(\epsilon_n-\epsilon_m)}}\langle\langle{\varphi^{(0)}_n;\beta|\varphi^{(1)}_n;\alpha}\rangle\nn[10pt]
    %
    &=
    \sum_{\beta=1}^N
    \textcolor{red}{
    E^{(2)}_{n,\gamma}\delta_{\gamma,\beta}
    }
    \langle\langle{\varphi^{(0)}_n;\beta|\varphi^{(1)}_n;\alpha}\rangle\nn[10pt]
    %
    &=E^{(2)}_{n,\beta}
    \langle\langle{\varphi^{(0)}_n;\beta|\varphi^{(1)}_n;\alpha}\rangle
    =E^{(2)}_{n,\beta}
    C^{(1)}_{\beta,\alpha}
\end{align}
となるから,
\begin{align}
    %
    E^{(2)}_{n,\alpha}C^{(1)}_{\beta,\alpha}
    &=
    E^{(2)}_{n,\beta}
    C^{(1)}_{\beta,\alpha}
    +\sum_{m\neq n}\Biggl[\sum_{p\neq n}
    \frac{\langle\braket{\varphi^{(0)}_{n};\gamma|\hat{V}|\varphi^{(0)}_m}
    \langle{\varphi^{(0)}_{m}|\hat{V}|\varphi^{(0)}_p}\rangle
    \braket{\varphi^{(0)}_{p}|\hat{V}|\varphi^{(0)}_n;\alpha}\rangle}
    {(\epsilon_n-\epsilon_m)(\epsilon_n-\epsilon_p)}\nn[10pt]
    %
    &\hspace{100pt}
    -E^{(1)}_n
    \frac{\langle\braket{\varphi^{(0)}_{n};\gamma|\hat{V}|\varphi^{(0)}_m}
    \braket{\varphi^{(0)}_{m}|\hat{V}|\varphi^{(0)}_n;\alpha}\rangle}{(\epsilon_n-\epsilon_m)^2}
    \Biggr]
    \nn[10pt]
    %
    \therefore
    (E^{(2)}_{n,\alpha}-E^{(2)}_{n,\beta})C^{(1)}_{\beta,\alpha}&=
    \sum_{m\neq n}\Biggl[\sum_{p\neq n}
    \frac{\langle\braket{\varphi^{(0)}_{n};\gamma|\hat{V}|\varphi^{(0)}_m}
    \langle{\varphi^{(0)}_{m}|\hat{V}|\varphi^{(0)}_p}\rangle
    \braket{\varphi^{(0)}_{p}|\hat{V}|\varphi^{(0)}_n;\alpha}\rangle}
    {(\epsilon_n-\epsilon_m)(\epsilon_n-\epsilon_p)}\nn[10pt]
    %
    &\hspace{100pt}
    -E^{(1)}_n
    \frac{\langle\braket{\varphi^{(0)}_{n};\gamma|\hat{V}|\varphi^{(0)}_m}
    \braket{\varphi^{(0)}_{m}|\hat{V}|\varphi^{(0)}_n;\alpha}\rangle}{(\epsilon_n-\epsilon_m)^2}
    \Biggr]
\end{align}
したがって,
\begin{align}
    C^{(1)}_{\beta,\alpha}
    &=
    \frac{1}{(E^{(2)}_{n,\alpha}-E^{(2)}_{n,\beta})}
    \sum_{m\neq n}\Biggl[\sum_{p\neq n}
    \frac{\langle\braket{\varphi^{(0)}_{n};\gamma|\hat{V}|\varphi^{(0)}_m}
    \langle{\varphi^{(0)}_{m}|\hat{V}|\varphi^{(0)}_p}\rangle
    \braket{\varphi^{(0)}_{p}|\hat{V}|\varphi^{(0)}_n;\alpha}\rangle}
    {(\epsilon_n-\epsilon_m)(\epsilon_n-\epsilon_p)}\nn[10pt]
    %
    &\hspace{100pt}
    -E^{(1)}_n
    \frac{\langle\braket{\varphi^{(0)}_{n};\gamma|\hat{V}|\varphi^{(0)}_m}
    \braket{\varphi^{(0)}_{m}|\hat{V}|\varphi^{(0)}_n;\alpha}\rangle}{(\epsilon_n-\epsilon_m)^2}
    \Biggr]
\end{align}






% 1次の摂動で$\epsilon_n$の縮退はすべて解かれているから,$E^{(1)}_{n,\alpha}\neq E^{(1)}_{n,\beta}$である.したがって,$\beta\neq\alpha$に対して,
% \begin{align}
%     \Tilde{C}^{(1)}_{m,\alpha}=\braket{\varphi^{(0)}_{m}|\varphi^{(1)}_n;\alpha}
%     &=\frac{\braket{\varphi^{(0)}_{m}|\hat{V}|\varphi^{(0)}_n;\alpha}\rangle}{(\epsilon_n-\epsilon_m)}
% \end{align}
% をより,
% \begin{align}
%     C^{(1)}_{\beta,\alpha}=\langle\braket{\varphi^{(0)}_{n};\beta|\varphi^{(1)}_n;\alpha}
%     &=\sum_{m\neq n}\frac{1}{(E^{(1)}_{n,\alpha}-E^{(1)}_{n,\beta})}
%     \langle\braket{\varphi^{(0)}_{n};\beta|\hat{V}|\varphi^{(0)}_m}
%     \braket{\varphi^{(0)}_{m}|\varphi^{(1)}_n;\alpha}\nn[10pt]
%     &=\sum_{m\neq n}
%     \frac{\langle\braket{\varphi^{(0)}_{n};\beta|\hat{V}|\varphi^{(0)}_m}
%     \braket{\varphi^{(0)}_{m}|\hat{V}|\varphi^{(0)}_n;\alpha}\rangle}
%     {(E^{(1)}_{n,\alpha}-E^{(1)}_{n,\beta})(\epsilon_n-\epsilon_m)},\ \ \ \beta\neq\alpha
% \end{align}

よって,固有状態の1次補正は
\begin{align}
    \ket{\varphi_{n,\alpha}^{(1)}}
    &=\hat{P}\ket{\varphi_{n,\alpha}^{(1)}}+\hat{Q}\ket{\varphi_{n,\alpha}^{(1)}}\nn[10pt]
    &=\sum_{\beta=1}^{N}|\varphi^{(0)}_n;\beta\rangle\rangle
    \langle\langle\varphi^{(0)}_n;\beta|\varphi_{n,\alpha}^{(1)}\rangle
    +\sum_{m\neq n}\ket{\varphi^{(0)}_m}\braket{\varphi^{(0)}_m|\varphi_{n,\alpha}^{(1)}}\nn[10pt]
    &=\sum_{\textcolor{red}{\beta\neq\alpha}}|\varphi^{(0)}_n;\beta\rangle\rangle
    \langle\langle\varphi^{(0)}_n;\beta|\varphi_{n,\alpha}^{(1)}\rangle
    +\sum_{m\neq n}\ket{\varphi^{(0)}_m}\braket{\varphi^{(0)}_m|\varphi_{n,\alpha}^{(1)}}\nn[10pt]
    %
    &=\sum_{\textcolor{red}{\beta\neq\alpha}}|\varphi^{(0)}_n;\beta\rangle\rangle
    \textcolor{red}{C^{(1)}_{\beta,\alpha}}
    +\sum_{m\neq n}\ket{\varphi^{(0)}_m}
    \textcolor{blue}{\braket{\varphi^{(0)}_m|\varphi_{n,\alpha}^{(1)}}}\nn[10pt]
    %
    %
    &=\sum_{\textcolor{red}{\beta\neq\alpha}}|\varphi^{(0)}_n;\beta\rangle\rangle
    \textcolor{red}{\frac{1}{(E^{(2)}_{n,\alpha}-E^{(2)}_{n,\beta})}
    \sum_{m\neq n}\Biggl[\sum_{p\neq n}
    \frac{\langle\braket{\varphi^{(0)}_{n};\gamma|\hat{V}|\varphi^{(0)}_m}
    \langle{\varphi^{(0)}_{m}|\hat{V}|\varphi^{(0)}_p}\rangle
    \braket{\varphi^{(0)}_{p}|\hat{V}|\varphi^{(0)}_n;\alpha}\rangle}
    {(\epsilon_n-\epsilon_m)(\epsilon_n-\epsilon_p)}}\nn[10pt]
    %
    &\hspace{100pt}
    \textcolor{red}{-E^{(1)}_n
    \frac{\langle\braket{\varphi^{(0)}_{n};\gamma|\hat{V}|\varphi^{(0)}_m}
    \braket{\varphi^{(0)}_{m}|\hat{V}|\varphi^{(0)}_n;\alpha}\rangle}{(\epsilon_n-\epsilon_m)^2}
    \Biggr]}
    \nn[10pt]
    %
    &\hspace{200pt}+
    \sum_{m\neq n}\ket{\varphi^{(0)}_m}
    \textcolor{blue}{
    \frac{\braket{\varphi^{(0)}_{m}|\hat{V}|\varphi^{(0)}_n;\alpha}\rangle}{(\epsilon_n-\epsilon_m)}}
\end{align}

以上の結果をまとめると,$\lambda$の1次近似で固有状態$\ket{\varphi_{n,\alpha}}$は
\begin{align}
    \ket{\varphi_{n,\alpha}}
    &=|\varphi^{(0)}_n;\alpha\rangle\rangle
    +\lambda \ket{\varphi_{n,\alpha}^{(1)}}\\[10pt]
    &=
    |\varphi^{(0)}_n;\alpha\rangle\rangle
    +\lambda\sum_{\textcolor{red}{\beta\neq\alpha}}|\varphi^{(0)}_n;\beta\rangle\rangle
    \textcolor{red}{\frac{1}{(E^{(2)}_{n,\alpha}-E^{(2)}_{n,\beta})}
    \sum_{m\neq n}\Biggl[\sum_{p\neq n}
    \frac{\langle\braket{\varphi^{(0)}_{n};\gamma|\hat{V}|\varphi^{(0)}_m}
    \langle{\varphi^{(0)}_{m}|\hat{V}|\varphi^{(0)}_p}\rangle
    \braket{\varphi^{(0)}_{p}|\hat{V}|\varphi^{(0)}_n;\alpha}\rangle}
    {(\epsilon_n-\epsilon_m)(\epsilon_n-\epsilon_p)}}\nn[10pt]
    %
    &\hspace{100pt}
    \textcolor{red}{-E^{(1)}_n
    \frac{\langle\braket{\varphi^{(0)}_{n};\gamma|\hat{V}|\varphi^{(0)}_m}
    \braket{\varphi^{(0)}_{m}|\hat{V}|\varphi^{(0)}_n;\alpha}\rangle}{(\epsilon_n-\epsilon_m)^2}
    \Biggr]}
    \nn[10pt]
    %
    &\hspace{200pt}+
    \sum_{m\neq n}\ket{\varphi^{(0)}_m}
    \textcolor{blue}{
    \frac{\braket{\varphi^{(0)}_{m}|\hat{V}|\varphi^{(0)}_n;\alpha}\rangle}{(\epsilon_n-\epsilon_m)}}
\end{align}
で与えられる.









