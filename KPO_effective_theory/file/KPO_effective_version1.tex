\part{Boltzmann machine}
\section{時間に依存しない摂動論を用いた低ポンプ下におけるKPOの固有状態,エネルギー固有値の解析}
\subsection{property of the model Hamiltonian}
\begin{align}\label{Eq.GKSL}
    \hat{H}_{\rm{KPO}}&=\hat{H}_0 +\beta\hat{V}\\[5pt]
    \hat{H}_0&=\Delta \hat{a}^\dagger\hat{a} - \frac{\chi}{2} \hat{a}^\dagger\hat{a}^\dagger\hat{a}\hat{a}\\[5pt]
    \hat{V}&=(\hat{a}^2 + \hat{a}^{\dagger 2})
\end{align}
ここで,$\hat{H}_0$は対角化できており,エネルギー固有状態とエネルギー固有値は以下で与えられる:
\begin{align}
    \hat{H}_0&=\Delta \hat{n} - \frac{\chi}{2}\hat{n}(\hat{n}-1) \\[5pt]
    \hat{H}_0\ket{n}&=\epsilon_n\ket{n},\ \ \ 
    \epsilon_n = \Delta n - \frac{\chi}{2}n(n-1)
\end{align}
ここで,基底状態と各励起状態が縮退する条件は
\begin{align}
        \epsilon_n &= \Delta n - \frac{\chi}{2}n(n-1) =0\\[10pt]
        \Delta &= \frac{\chi}{2}(n-1)
\end{align}
となる.
\subsection{基底状態と第二励起状態が縮退している場合}
$\ket{0}$と$\ket{2}$が縮退している場合を考えてみる.このとき,摂動を受ける部分空間Pは$\{\ket{0},\ \ket{2}\}$であり,完全性条件は
\begin{equation}
    \hat{1} = \hat{P} + \hat{Q} 
    = \ket{0}\bra{0} + \ket{2}\bra{2} + \sum_{i\neq 0,2}\ket{i}\bra{i} 
\end{equation}
となる.また,$\ket{0}$と$\ket{2}$が縮退している場合,$\Delta=\chi/2$が縮退する条件となる.このとき,$\epsilon_0=\epsilon_2=0$となる.

\paragraph{基底状態}
\begin{align}
    \ket{\varphi_{n,0}^{(1)}}
    &=\sum_{m\neq 0,2}
    \Biggl\{\sum_{\textcolor{red}{\beta\neq0}}
    |\varphi^{(0)}_n;\beta\rangle\rangle
    \frac{\langle\braket{\varphi^{(0)}_{n};\beta|\hat{V}|\varphi^{(0)}_m}
    \braket{\varphi^{(0)}_{m}|\hat{V}|\varphi^{(0)}_n;0}\rangle}
    {(E^{(1)}_{n,0}-E^{(1)}_{n,\beta})(\epsilon_n-\epsilon_m)}
    +\ket{\varphi_m^{(0)}}\frac{\braket{\varphi^{(0)}_{m}|\hat{V}|\varphi^{(0)}_n;0}\rangle}{(\epsilon_n-\epsilon_m)}
    \Biggr\}\nn[20pt]
    %%%%%%%%%%%%%%%%%
    &=\sum_{m\neq 0,2}
    \Biggl\{
    |\varphi^{(0)}_n;2\rangle\rangle
    \frac{\langle\braket{\varphi^{(0)}_{n};2|\hat{V}|\varphi^{(0)}_m}
    \braket{\varphi^{(0)}_{m}|\hat{V}|\varphi^{(0)}_n;0}\rangle}
    {(E^{(1)}_{n,0}-E^{(1)}_{n,2})(\epsilon_n-\epsilon_m)}
    +\ket{\varphi_m^{(0)}}\frac{\braket{\varphi^{(0)}_{m}|\hat{V}|\varphi^{(0)}_n;0}\rangle}{(\epsilon_n-\epsilon_m)}
    \Biggr\}\nn[20pt]
    %%%%%%%%%%%%%%%%%
    &=
    |\varphi^{(0)}_n;2\rangle\rangle
    \frac{\langle\braket{\varphi^{(0)}_{n};2|\hat{V}|\varphi^{(0)}_4}
    \braket{\varphi^{(0)}_{4}|\hat{V}|\varphi^{(0)}_n;0}\rangle}
    {(E^{(1)}_{n,0}-E^{(1)}_{n,2})(-\epsilon_4)}
    +\ket{\varphi_4^{(0)}}\frac{\braket{\varphi^{(0)}_{4}|\hat{V}|\varphi^{(0)}_n;0}\rangle}{(-\epsilon_4)}
\end{align}
ここで,$|\varphi^{(0)}_n;0\rangle\equiv\ket{0}$, $|\varphi^{(0)}_n;2\rangle\equiv\ket{2}$, $|\varphi^{(0)}_m\rangle\equiv\ket{m}$と対応させると,
\begin{align}
    \langle\braket{\varphi^{(0)}_{n};2|\hat{V}|\varphi^{(0)}_4}
    &=\langle\braket{\varphi^{(0)}_{n};2|\hat{V}|4}
    =\frac{1}{\sqrt{2}}(-\braket{0|\hat{V}|4} + \braket{2|\hat{V}|4})\nn[10pt]
    &=\frac{1}{\sqrt{2}}\braket{2|\hat{a}^{\dagger2}+\hat{a}^2|4}
    =\frac{1}{\sqrt{2}}\sqrt{4}\sqrt{3}=\sqrt{2}\sqrt{3}
\end{align}
\begin{align}
    \langle\braket{\varphi^{(0)}_{n};0|\hat{V}|\varphi^{(0)}_4}
    &=\langle\braket{\varphi^{(0)}_{n};0|\hat{V}|4}
    =\frac{1}{\sqrt{2}}(\braket{0|\hat{V}|4} + \braket{2|\hat{V}|4})\nn[10pt]
    &=\frac{1}{\sqrt{2}}\braket{2|\hat{a}^{\dagger2}+\hat{a}^2|4}
    =\frac{1}{\sqrt{2}}\sqrt{4}\sqrt{3}=\sqrt{2}\sqrt{3}
\end{align}
また, $\Delta=K/2$なので,
\begin{equation}
    \epsilon_4 = 4\Delta - \frac{\chi}{2}3\cdot4 = 4\frac{\chi}{2} - \frac{\chi}{2}3\cdot4
    =-4\chi
\end{equation}
\begin{align}
    \hat{a}^\dag\ket{n} &= \sqrt{n+1}\ket{n+1}\\[10pt]
    \hat{a}\ket{n} &= \sqrt{n}\ket{n-1}
\end{align}
\begin{align}
    \ket{\varphi_{n,0}^{(1)}}
    &=
    |\varphi^{(0)}_n;2\rangle\rangle
    \frac{\langle\braket{\varphi^{(0)}_{n};2|\hat{V}|\varphi^{(0)}_4}
    \braket{\varphi^{(0)}_{4}|\hat{V}|\varphi^{(0)}_n;0}\rangle}
    {(E^{(1)}_{n,0}-E^{(1)}_{n,2})(-\epsilon_4)}
    +\ket{\varphi_4^{(0)}}\frac{\braket{\varphi^{(0)}_{4}|\hat{V}|\varphi^{(0)}_n;0}\rangle}{(-\epsilon_4)}\nn[10pt]
    &=|\varphi^{(0)}_n;2\rangle\rangle
    \frac{(\sqrt{2}\sqrt{3})(-\sqrt{2}\sqrt{3})}
    {(-2\sqrt{2})(-\epsilon_4)}
    +\ket{\varphi_4^{(0)}}\frac{\sqrt{2}\sqrt{3}}{(-\epsilon_4)}\nn[10pt]
    &=|\varphi^{(0)}_n;2\rangle\rangle
    \frac{(\sqrt{2}\sqrt{3})(-\sqrt{2}\sqrt{3})}
    {(-2\sqrt{2})(-(-4\chi))}
    +\ket{\varphi_4^{(0)}}\frac{\sqrt{2}\sqrt{3}}{(-(-4\chi))}\nn[10pt]
    &=|\varphi^{(0)}_n;2\rangle\rangle
    \frac{2\cdot3}
    {8\sqrt{2}\chi}
    +\ket{\varphi_4^{(0)}}\frac{\sqrt{2}\sqrt{3}}{4\chi}\nn[10pt]
    &=\frac{1}{\sqrt{2}}
    \frac{2\cdot3}
    {8\sqrt{2}\chi}
    (\ket{0}-\ket{2})
    +\frac{\sqrt{2}\sqrt{3}}{4\chi}\ket{4}
    =
    \frac{3}
    {8\chi}
    (\ket{0}-\ket{2})
    +\frac{\sqrt{2}\sqrt{3}}{4\chi}\ket{4}
\end{align}
よって,
\begin{align}
    \ket{\varphi_{n,0}}
    =
    \frac{1}
    {\sqrt{2}}
    (\ket{0}+\ket{2})
    +\frac{3\beta}
    {8\chi}
    (\ket{0}-\ket{2})
    +\frac{\sqrt{2}\sqrt{3}\beta}{4\chi}\ket{4}
\end{align}

%%%%%%%%%%%%%%%%%%%%%%%%%%%%%%%%%%%%%%
\paragraph{第2励起状態}
\begin{align}
    \ket{\varphi_{n,2}^{(1)}}
    &=\sum_{m\neq 0,2}
    \Biggl\{\sum_{\textcolor{red}{\beta\neq2}}
    |\varphi^{(0)}_n;\beta\rangle\rangle
    \frac{\langle\braket{\varphi^{(0)}_{n};\beta|\hat{V}|\varphi^{(0)}_m}
    \braket{\varphi^{(0)}_{m}|\hat{V}|\varphi^{(0)}_n;2}\rangle}
    {(E^{(1)}_{n,2}-E^{(1)}_{n,\beta})(\epsilon_n-\epsilon_m)}
    +\ket{\varphi_m^{(0)}}\frac{\braket{\varphi^{(0)}_{m}|\hat{V}|\varphi^{(0)}_n;2}\rangle}{(\epsilon_n-\epsilon_m)}
    \Biggr\}\nn[20pt]
    %%%%%%%%%%%%%%%%%
    &=\sum_{m\neq 0,2}
    \Biggl\{
    |\varphi^{(0)}_n;0\rangle\rangle
    \frac{\langle\braket{\varphi^{(0)}_{n};0|\hat{V}|\varphi^{(0)}_m}
    \braket{\varphi^{(0)}_{m}|\hat{V}|\varphi^{(0)}_n;2}\rangle}
    {(E^{(1)}_{n,2}-E^{(1)}_{n,0})(\epsilon_n-\epsilon_m)}
    +\ket{\varphi_m^{(0)}}\frac{\braket{\varphi^{(0)}_{m}|\hat{V}|\varphi^{(0)}_n;2}\rangle}{(\epsilon_n-\epsilon_m)}
    \Biggr\}\nn[20pt]
    %%%%%%%%%%%%%%%%%
    &=
    |\varphi^{(0)}_n;0\rangle\rangle
    \frac{\langle\braket{\varphi^{(0)}_{n};0|\hat{V}|\varphi^{(0)}_4}
    \braket{\varphi^{(0)}_{4}|\hat{V}|\varphi^{(0)}_n;2}\rangle}
    {(E^{(1)}_{n,2}-E^{(1)}_{n,0})(-\epsilon_4)}
    +\ket{\varphi_4^{(0)}}\frac{\braket{\varphi^{(0)}_{4}|\hat{V}|\varphi^{(0)}_n;2}\rangle}{(-\epsilon_4)}
\end{align}
ここで,$|\varphi^{(0)}_n;0\rangle\equiv\ket{0}$, $|\varphi^{(0)}_n;2\rangle\equiv\ket{2}$, $|\varphi^{(0)}_m\rangle\equiv\ket{m}$と対応させると,
\begin{align}
    \langle\braket{\varphi^{(0)}_{n};2|\hat{V}|\varphi^{(0)}_4}
    &=\langle\braket{\varphi^{(0)}_{n};2|\hat{V}|4}
    =\frac{1}{\sqrt{2}}(-\braket{0|\hat{V}|4} + \braket{2|\hat{V}|4})\nn[10pt]
    &=\frac{1}{\sqrt{2}}\braket{2|\hat{a}^{\dagger2}+\hat{a}^2|4}
    =\frac{1}{\sqrt{2}}\sqrt{4}\sqrt{3}=-\sqrt{2}\sqrt{3}
\end{align}
\begin{align}
    \langle\braket{\varphi^{(0)}_{n};0|\hat{V}|\varphi^{(0)}_4}
    &=\langle\braket{\varphi^{(0)}_{n};0|\hat{V}|4}
    =\frac{1}{\sqrt{2}}(\braket{0|\hat{V}|4} + \braket{2|\hat{V}|4})\nn[10pt]
    &=\frac{1}{\sqrt{2}}\braket{2|\hat{a}^{\dagger2}+\hat{a}^2|4}
    =\frac{1}{\sqrt{2}}\sqrt{4}\sqrt{3}=\sqrt{2}\sqrt{3}
\end{align}
また, $\Delta=K/2$なので,
\begin{equation}
    \epsilon_4 = 4\Delta - \frac{\chi}{2}3\cdot4 = 4\frac{\chi}{2} - \frac{\chi}{2}3\cdot4
    =-4\chi
\end{equation}
\begin{align}
    \hat{a}^\dag\ket{n} &= \sqrt{n+1}\ket{n+1}\\[10pt]
    \hat{a}\ket{n} &= \sqrt{n}\ket{n-1}
\end{align}
\begin{align}
    \ket{\varphi_{n,2}^{(1)}}
    &=
    |\varphi^{(0)}_n;0\rangle\rangle
    \frac{\langle\braket{\varphi^{(0)}_{n};0|\hat{V}|\varphi^{(0)}_4}
    \braket{\varphi^{(0)}_{4}|\hat{V}|\varphi^{(0)}_n;2}\rangle}
    {(E^{(1)}_{n,2}-E^{(1)}_{n,0})(-\epsilon_4)}
    +\ket{\varphi_4^{(0)}}\frac{\braket{\varphi^{(0)}_{4}|\hat{V}|\varphi^{(0)}_n;2}\rangle}{(-\epsilon_4)}\nn[10pt]
    &=|\varphi^{(0)}_n;0\rangle\rangle
    \frac{(\sqrt{2}\sqrt{3})(-\sqrt{2}\sqrt{3})}
    {(2\sqrt{2})(-\epsilon_4)}
    +\ket{\varphi_4^{(0)}}\frac{-\sqrt{2}\sqrt{3}}{(-\epsilon_4)}\nn[10pt]
    &=|\varphi^{(0)}_n;0\rangle\rangle
    \frac{(\sqrt{2}\sqrt{3})(-\sqrt{2}\sqrt{3})}
    {(2\sqrt{2})(-(-4\chi))}
    +\ket{\varphi_4^{(0)}}\frac{-\sqrt{2}\sqrt{3}}{(-(-4\chi))}\nn[10pt]
    &=|\varphi^{(0)}_n;0\rangle\rangle
    \frac{2\cdot3}
    {8\sqrt{2}\chi}
    -\ket{\varphi_4^{(0)}}\frac{\sqrt{2}\sqrt{3}}{4\chi}\nn[10pt]
    &=\frac{1}{\sqrt{2}}
    \frac{2\cdot3}
    {8\sqrt{2}\chi}
    (\ket{0}+\ket{2})
    -\frac{\sqrt{2}\sqrt{3}}{4\chi}\ket{4}
    =
    \frac{3}
    {8\chi}
    (\ket{0}+\ket{2})
    -\frac{\sqrt{2}\sqrt{3}}{4\chi}\ket{4}
\end{align}
よって,
\begin{align}
    \ket{\varphi_{n,2}}
    =
    \frac{1}
    {\sqrt{2}}
    (\ket{0}-\ket{2})
    +\frac{3\beta}
    {8\chi}
    (\ket{0}+\ket{2})
    -\frac{\sqrt{2}\sqrt{3}\beta}{4\chi}\ket{4}
\end{align}



\subsection{縮退のない場合}
\section*{固有エネルギーの補正値}
縮退のない摂動論より,固有エネルギーの補正値は以下の式で与えられる:
\begin{align}
E_n&\simeq E_{n}^{(0)}+\lambda E_{n}^{(1)}+\lambda^2 E_{n}^{(2)}\notag\\[10pt]
&=\epsilon_n+\Braket{\varphi_{n}^{(0)}|\lambda\hat{V}|\varphi_{n}^{(0)}}
+\lambda^2\displaystyle\sum_{\substack{m=1 \\ m\neq n}}^\infty
 \frac{
 \braket{\varphi_{m}^{(0)}|\hat{V}|\varphi_{n}^{(0)}}
  \bra{\varphi_{n}^{(0)}}\hat{V}\ket{\varphi_{m}^{(0)}}}{(\epsilon_{n}-\epsilon_{m})}.
\end{align}


\section*{固有状態の補正}
\paragraph{基底状態}
\begin{align}
\ket{\varphi_{n}}&\simeq\ket{\varphi_{n}^{(0)}}+\lambda\ket{\varphi_{n}^{(1)}}+\lambda^2\ket{\varphi_{n}^{(2)}}\notag\\[10pt]
&=\ket{\varphi_{n}^{(0)}}
+
\displaystyle\sum_{\substack{m=1 \\ m\neq n}}^\infty
 \frac{\braket{\varphi_{n}^{(0)}|\lambda\hat{V}|\varphi_{n}^{(0)}}}{(\epsilon_{n}-\epsilon_{m})}
\ket{\varphi_{m}^{(0)}}\notag\\[10pt]
&+\lambda^2
\displaystyle\sum_{\substack{m=1 \\ m\neq n}}^\infty
 \left[\left(
\displaystyle\sum_{\substack{k=1 \\ k\neq n}}^\infty
 \frac{
 \braket{\varphi_{k}^{(0)}|\hat{V}|\varphi_{n}^{(0)}}
  \bra{\varphi_{m}^{(0)}}\hat{V}\ket{\varphi_{k}^{(0)}}}{(\epsilon_{n}-\epsilon_{k})(\epsilon_{n}-\epsilon_{m})}
    \right)
  -
   \frac{
 \braket{\varphi_{n}^{(0)}|\hat{V}|\varphi_{n}^{(0)}}\braket{\varphi_{m}^{(0)}|\hat{V}|\varphi_{n}^{(0)}}}{(\epsilon_{n}-\epsilon_{m})^2}
  \right]
\ket{\varphi_{m}^{(0)}}
\end{align}

\begin{align}
    \ket{\varphi_{0}^{(1)}}
    &=\displaystyle\sum_{\substack{m\neq 0}}^\infty
     \frac{\braket{\varphi_{m}^{(0)}|\hat{V}|\varphi_{0}^{(0)}}}{(\epsilon_{0}-\epsilon_{m})}
    \ket{\varphi_{m}^{(0)}}
    =
     \frac{\braket{\varphi_{2}^{(0)}|\hat{V}|\varphi_{0}^{(0)}}}{(\epsilon_{0}-\epsilon_{2})}
    \ket{\varphi_{2}^{(0)}}
    %
    = \frac{\braket{2|\hat{V}|0}}{(-\epsilon_{2})}
    \ket{2}\nn[10pt]
    & = \frac{\sqrt{2}}{\chi-2\Delta}
    \ket{2}
\end{align}

\begin{align}
\ket{\varphi_{0}^{(2)}}
& = 
    \displaystyle\sum_{\substack{m=1 \\ m\neq n}}^\infty
     \left[\left(
    \displaystyle\sum_{\substack{k=1 \\ k\neq n}}^\infty
     \frac{
     \braket{\varphi_{k}^{(0)}|\hat{V}|\varphi_{0}^{(0)}}
      \bra{\varphi_{m}^{(0)}}\hat{V}\ket{\varphi_{k}^{(0)}}}{(\epsilon_{0}-\epsilon_{k})(\epsilon_{0}-\epsilon_{m})}
        \right)
      -
       \frac{
     \braket{\varphi_{0}^{(0)}|\hat{V}|\varphi_{0}^{(0)}}\braket{\varphi_{m}^{(0)}|\hat{V}|\varphi_{0}^{(0)}}}{(\epsilon_{0}-\epsilon_{m})^2}
      \right]
    \ket{\varphi_{m}^{(0)}}\nn[10pt]
    &=\displaystyle\sum_{\substack{m=1 \\ m\neq n}}^\infty
    \left(
     \frac{
     \braket{\varphi_{2}^{(0)}|\hat{V}|\varphi_{0}^{(0)}}
      \bra{\varphi_{m}^{(0)}}\hat{V}\ket{\varphi_{2}^{(0)}}}{(\epsilon_{0}-\epsilon_{2})(\epsilon_{0}-\epsilon_{m})}
      +\frac{
     \braket{\varphi_{4}^{(0)}|\hat{V}|\varphi_{0}^{(0)}}
      \bra{\varphi_{m}^{(0)}}\hat{V}\ket{\varphi_{4}^{(0)}}}{(\epsilon_{0}-\epsilon_{4})(\epsilon_{0}-\epsilon_{m})}
        \right)
    \ket{\varphi_{m}^{(0)}}\nn[10pt]
    &=
     \frac{
     \braket{\varphi_{2}^{(0)}|\hat{V}|\varphi_{0}^{(0)}}
      \bra{\varphi_{4}^{(0)}}\hat{V}\ket{\varphi_{2}^{(0)}}}{(\epsilon_{0}-\epsilon_{2})(\epsilon_{0}-\epsilon_{4})}
    \ket{\varphi_{4}^{(0)}}
    =\frac{
     \braket{2|\hat{V}|0}
      \bra{4}\hat{V}\ket{2}}{(-\epsilon_{2})(-\epsilon_{4})}
    \ket{4}\nn[10pt]
    &=
     \frac{
     \braket{\varphi_{2}^{(0)}|\hat{V}|\varphi_{0}^{(0)}}
      \bra{\varphi_{4}^{(0)}}\hat{V}\ket{\varphi_{2}^{(0)}}}{(\epsilon_{0}-\epsilon_{2})(\epsilon_{0}-\epsilon_{4})}
    \ket{\varphi_{4}^{(0)}}
    =\frac{
     \braket{2|\hat{V}|0}
      \bra{4}\hat{V}\ket{2}}{(\chi-2\Delta)(2\cdot3\chi-4\Delta)}
    \ket{4}\nn[10pt]
    &
    =\frac{
     \sqrt{2}\sqrt{3}\sqrt{4}}{2(\chi-2\Delta)(3\chi-2\Delta)}
    \ket{4}
    =\frac{
     \sqrt{2}\sqrt{3}}{(\chi-2\Delta)(3\chi-2\Delta)}
    \ket{4}\nn[10pt]
\end{align}


\begin{align}
\ket{\varphi_{n}}&\simeq\ket{\varphi_{n}^{(0)}}+\beta\ket{\varphi_{n}^{(1)}}+\beta^2\ket{\varphi_{n}^{(2)}}\notag\\[10pt]
&=\ket{0}
+\beta\frac{\sqrt{2}}{\chi-2\Delta}
    \ket{2}
+\beta^2\frac{
     \sqrt{2}\sqrt{3}}{(\chi-2\Delta)(3\chi-2\Delta)}
    \ket{4}
\end{align}









