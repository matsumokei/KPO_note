
\part{その他の摂動論について}
\section{Schrieffer-Wolff変換}
Schrieffer-Wolff変換とは,非対角項を持つHamiltonianに対して,近似的に対角化を行う,一種の摂動論である.

Schrieffer-Wolff transiformation is a perturbation theory which a Hamiltonian with off-diagonal elements approximately diagonalize.

We consider the following Hamiltonian
\begin{equation}
    \hat{H} = \hat{H}_0 + \lambda\hat{V}
\end{equation}
where, $\hat{H}_0$ is the non-perturbative Hamiltonian and $\hat{V}$ is the interaction Hamiltonian. Here, we decompose $\hat{V}=\hat{H}_1 + \hat{H}_2$, where $\hat{H}_1$ and $\hat{H}_2$ denote ブロック対角Hamiltonian and  非対角ブロックHamiltonian.


The Schrieffer-Wolff transiformation is defined by 

\begin{equation}
    \hat{H}^{\prime}\equiv e^{\hat{S}}\hat{H} e^{-\hat{S}},
\end{equation}
where $\hat{S}$ is an anti-Hermitian operator. 

We choose $\hat{S}$ such that $\hat{H}^{\prime}$ satisfies

Using BCH-fomula, 
\begin{equation}
    e^{\hat{S}}\hat{H}e^{-\hat{S}}
    =\hat{H}+[\hat{S}, \hat{H}]
    +\frac{1}{2!}\bigl[
    \hat{S}, [\hat{S}, \hat{H}]
    \bigr]
    +\frac{1}{3!}
    \Bigl[\hat{S},
    \bigl[
    \hat{S}, [\hat{S}, \hat{H}]
    \bigr]
    \Bigr]
    +\cdots
\end{equation}

\begin{align}
    e^{\hat{S}}\hat{H}e^{-\hat{S}}
    &=\hat{H}_0 + \lambda\hat{V}+[\hat{S}, \hat{H}_0 + \lambda\hat{V}]
    +\frac{1}{2!}\bigl[
    \hat{S}, [\hat{S}, \hat{H}_0 + \lambda\hat{V}]
    \bigr]
    +\frac{1}{3!}
    \Bigl[\hat{S},
    \bigl[
    \hat{S}, [\hat{S}, \hat{H}_0 + \lambda\hat{V}]
    \bigr]
    \Bigr]
    +\cdots\nn[10pt]
    %
    &=\hat{H}_0 + \lambda\hat{V}+[\hat{S}, \hat{H}_0] +
    \lambda[\hat{S}, \hat{V}]
    +\frac{1}{2!}\bigl[
    \hat{S}, [\hat{S}, \hat{H}_0]
    \bigr]
    +\frac{1}{2!}\lambda\bigl[
    \hat{S}, [\hat{S}, \hat{V}]
    \bigr]
    +\cdots \\[10pt]
    &=\hat{H}_0 + \hat{W},
\end{align}
where
\begin{equation}
    \hat{W} = \lambda\hat{V}+[\hat{S}, \hat{H}_0] +
    \lambda[\hat{S}, \hat{V}]
    +\frac{1}{2!}\bigl[
    \hat{S}, [\hat{S}, \hat{H}_0]
    \bigr]
    +\frac{1}{2!}\lambda\bigl[
    \hat{S}, [\hat{S}, \hat{V}]
    \bigr]
    +\cdots
\end{equation}
この演算子はlevel-shift 演算子と呼ばれている.


\subsection{Iterative calculation of $\hat{S}$}

We expand $\hat{S}$ as $\sum_{n}\lambda^{n}\hat{S_n}$, so we obtain




\begin{align}
    e^{\hat{S}}\hat{H}e^{-\hat{S}}
    &=\hat{H}_0 + \lambda\hat{V}+\left[\sum_{n}\lambda^{n}\hat{S_n}, \hat{H}_0\right] +
    \lambda\left[\sum_{n}\lambda^{n}\hat{S_n}, \hat{V}\right]\nn[10pt]
    &+\frac{1}{2!}\biggl[
    \sum_{n}\lambda^{n}\hat{S_n}, \left[\sum_{n}\lambda^{n}\hat{S_n}, \hat{H}_0\right]
    \biggr]
    +\frac{1}{2!}\lambda\biggl[
    \sum_{n}\lambda^{n}\hat{S_n}, \left[\sum_{n}\lambda^{n}\hat{S_n}, \hat{V}\right]
    \biggr]
    +\cdots\nn[10pt]
    %
    &=\hat{H}_0 + \lambda\hat{V}+\lambda[\hat{S_1}, \hat{H}_0] + \lambda^2[\hat{S_2}, \hat{H}_0]
    +\lambda^2[\hat{S_1}, \hat{V}]
    +\frac{1}{2!}\lambda^2\biggl[
    \hat{S_1}, [\hat{S}_{1}, \hat{H}_0]
    \biggr]
    +\mathcal{O}(\lambda^3)\nn[10pt]
    %
    &=\hat{H}_0 + \lambda
    \Bigl\{\hat{V}+[\hat{S_1}, \hat{H}_0]\Bigr\}
    %
    +\lambda^2
    \biggl\{[\hat{S_2}, \hat{H}_0]
    +[\hat{S_1}, \hat{V}]
    +\frac{1}{2}\biggl[
    \hat{S_1}, [\hat{S}_{1}, \hat{H}_0]
    \biggr]
    \biggr\}
    +\mathcal{O}(\lambda^3)
\end{align}



$\lambda$の次数について,整理すると,
\begin{align}
    e^{\hat{S}}\hat{H}e^{-\hat{S}}
    &=\hat{H}_0 + \lambda
    \Bigl\{\hat{V}+[\hat{S_1}, \hat{H}_0]\Bigr\}\nn[10pt]
    %
    &+\lambda^2
    \biggl\{[\hat{S_2}, \hat{H}_0]
    +[\hat{S_1}, \hat{V}]
    +\frac{1}{2}\biggl[
    \hat{S_1}, [\hat{S}_{1}, \hat{H}_0]
    \biggr]
    \biggr\}
    +\mathcal{O}(\lambda^3)
\end{align}




\begin{align}
    e^{\hat{S}}\hat{H}e^{-\hat{S}}
    &=\hat{H}_0 + \lambda
    \Bigl\{\hat{V}+[\hat{S_1}, \hat{H}_0]\Bigr\}
    %
    +\lambda^2
    \biggl\{[\hat{S_2}, \hat{H}_0]
    +[\hat{S_1}, \hat{V}]
    +\frac{1}{2}\biggl[
    \hat{S_1}, [\hat{S}_{1}, \hat{H}_0]
    \biggr]
    \biggr\}
    +\mathcal{O}(\lambda^3)
\end{align}


We have $\hat{V}=\hat{H}_1 + \hat{H}_2$ and obtain 
\begin{align}
    e^{\hat{S}}\hat{H}e^{-\hat{S}}
    &=\hat{H}_0 + \lambda
    \Bigl\{\hat{H}_1+\hat{H}_2+[\hat{S_1}, \hat{H}_0]\Bigr\}
    %
    +\lambda^2
    \biggl\{[\hat{S_2}, \hat{H}_0]
    +[\hat{S_1}, \hat{H}_1]
    +[\hat{S_1}, \hat{H}_2]
    +\frac{1}{2}\biggl[
    \hat{S_1}, [\hat{S}_{1}, \hat{H}_0]
    \biggr]
    \biggr\}
    +\mathcal{O}(\lambda^3)\nn[10pt]
    &=\hat{H}_0 + \lambda\hat{H}_1
    +\lambda^2
    \biggl\{
    [\hat{S_1}, \hat{H}_2]
    +\frac{1}{2}\biggl[
    \hat{S_1}, [\hat{S}_{1}, \hat{H}_0]
    \biggr]
    \biggr\}\nn[10pt]
    %
    &\hspace{50pt}+\lambda
    \Bigl\{\hat{H}_2+[\hat{S_1}, \hat{H}_0]\Bigr\}
    %
    +\lambda^2
    \biggl\{[\hat{S_2}, \hat{H}_0]
    +[\hat{S_1}, \hat{H}_1]
    \biggr\}
    +\mathcal{O}(\lambda^3)
\end{align}
ここで,$[\hat{S_1}, [\hat{S}_{1}, \hat{H}_0]]$は[非ブロック対角, [非ブロック対角,ブロック対角]]=[非ブロック対角,ブロック対角]=ブロック対角よりブロック対角,$[\hat{S_2}, \hat{H}_0]$, $[\hat{S_1}, \hat{H}_1]$は非ブロック対角行列である.なぜならば
,ブロック対角行列と非ブロック対角行列の交換関係は非ブロック対角であり、2つの非ブロック対角行列の交換関係はブロック対角になるからである.

Here, we choose $\hat{S}_1$ and $\hat{S}_2$ to cancel the off-diagonal elements of $\hat{H}^{\prime}$, 
\begin{equation}
    [\hat{S_1}, \hat{H}_0]=-\hat{H}_2
\end{equation}
\begin{equation}
    [\hat{S_2}, \hat{H}_0]
    =-[\hat{S_1}, \hat{H}_1]
\end{equation}

Therefore we obtain the effective Hamiltonian
\begin{align}
    \hat{H}^{\prime}
    &=\hat{H}_0 + \lambda\hat{H}_1
    +\lambda^2
    \biggl\{
    [\hat{S_1}, \hat{H}_2]
    +\frac{1}{2}\biggl[
    \hat{S_1}, \textcolor{red}{[\hat{S}_{1}, \hat{H}_0]}
    \biggr]
    \biggr\}\\[10pt]
    &=\hat{H}_0 + \lambda\hat{H}_1
    +\lambda^2
    \biggl\{
    [\hat{S_1}, \hat{H}_2]
    -\frac{1}{2}[
    \hat{S_1}, \hat{H}_2
    ]
    \biggr\}\nn[10pt]
    &=\hat{H}_0 + \lambda\hat{H}_1
    +\frac{1}{2}\lambda^2
    [\hat{S_1}, \hat{H}_2]
\end{align}

\subsection{Example : Jaynes-Cumings model}
We introduce the Jaynes-Cumings model.
The Hamiltonian is given by
\begin{equation}
    \hat{H}_{\rm{JC}}
    =\frac{1}{2}\hbar\omega\hsig_z
    +\hbar\omega_0\hat{a}^\dagger\ha
    +\hbar g(\hsig_+\hat{a}+\hsig_-\hat{a}^\dagger)
\end{equation}

To construct of an effective Hamiltonian, we can list out the diagonal and off-diagonal parts of the Hamiltonian as $\hat{H}_0$ and $\hat{V}=\hat{H}_2$ respectively:
\begin{align}
    \hat{H}_0&=\frac{1}{2}\hbar\omega\hsig_z
    +\hbar\omega_0\hat{a}^\dagger\ha\\[10pt]
    \hat{V}&=\hbar g(\hsig_+\hat{a}+\hsig_-\hat{a}^\dagger)
\end{align}

We assume that $\hat{S}_{1}$ is given by
\begin{equation}
    \hat{S}_{1}=-\sum_{m,n}\frac{\braket{m|\hat{V}|n}}{E_{m}-E_{n}}\ket{m}\bra{n}.
\end{equation}
これは,$\hat{H}_0$の固有状態で,

\begin{equation}
    \hat{S}_1 = \sum_{m,n}\ket{m}\braket{m|\hat{S}_1|n}\bra{n}
    =\sum_{m,n}c_{m,n}\ket{m}\bra{n}
\end{equation}をと展開し,\eqref{}へ代入することで得られる.

Using $\hat{S}_1$, we obtain
\begin{align}
    \hat{S}_{1}
    &=-\sum_{m,n}
    \left[
    \frac{\braket{m|\hsig_+\hat{a}|n}}{E_{m}-E_{n}}\ket{m}\bra{n}
    +\frac{\braket{m|\hsig_-\hat{a}^\dagger|n}}{E_{m}-E_{n}}\ket{m}\bra{n}
    \right]\nn[10pt]
    &=\frac{g}{\Delta}(\hsig_+\hat{a}-\hsig_-\hat{a}^\dagger)
\end{align}
where $\Delta = \omega-\omega_0$.
We adapt the Schrieffer-Wolff transiformation of $\hat{S}_1$, and we obtain
\begin{align}
    \hat{H}^{\prime}
   &=\hat{H}_0
    +\frac{1}{2}\lambda^2
    [\hat{S}_1, \hat{H}_2]\nn[10pt]
    &=\frac{1}{2}\hbar\omega\hsig_z
    +\hbar\omega_0\hat{a}^\dagger\ha
    +\frac{\hbar g^2}{\Delta}(\hsig_+\hsig_-
    +\hsig_z\hat{a}^\dagger\hat{a})\nn[10pt]
    %%
    &=\frac{1}{2}\hbar\omega(\hsig_+\hsig_- - \hsig_-\hsig_+)
    +\hbar\omega_0(\hsig_+\hsig_- + \hsig_-\hsig_+)\hat{a}^\dagger\ha
    +\frac{\hbar g^2}{\Delta}\hsig_+\hsig_-
    +\frac{\hbar g^2}{\Delta}(\hsig_+\hsig_- - \hsig_-\hsig_+)\hat{a}^\dagger\hat{a}\nn[10pt]
    %%
    &=\frac{1}{2}\hbar\omega\hsig_+\hsig_- 
    - \frac{1}{2}\hbar\omega\hsig_-\hsig_+
    +\hbar\omega_0\hsig_+\hsig_- \hat{a}^\dagger\ha
    + \hbar\omega_0\hsig_-\hsig_+\hat{a}^\dagger\ha
    +\frac{\hbar g^2}{\Delta}\hsig_+\hsig_-
    +\frac{\hbar g^2}{\Delta}\hsig_+\hsig_- \hat{a}^\dagger\hat{a}
    -\frac{\hbar g^2}{\Delta} \hsig_-\hsig_+\hat{a}^\dagger\hat{a}\nn[10pt]
     %%
    &=\frac{1}{2}\hbar\omega\ket{e}\bra{e}
    - \frac{1}{2}\hbar\omega\ket{g}\bra{g}
    +\hbar\omega_0\ket{e}\bra{e} \hat{a}^\dagger\ha
    + \hbar\omega_0\ket{g}\bra{g}\hat{a}^\dagger\ha
    +\frac{\hbar g^2}{\Delta}\ket{e}\bra{e}
    +\frac{\hbar g^2}{\Delta}\ket{e}\bra{e} \hat{a}^\dagger\hat{a}
    -\frac{\hbar g^2}{\Delta} \ket{g}\bra{g}\hat{a}^\dagger\hat{a}\nn[10pt]
     %%
    &=\frac{1}{2}\hbar\omega\ket{e}\bra{e}
    +\frac{\hbar g^2}{\Delta}\ket{e}\bra{e}
    +\hbar\omega_0\ket{e}\bra{e} \hat{a}^\dagger\ha
    +\frac{\hbar g^2}{\Delta}\ket{e}\bra{e} \hat{a}^\dagger\hat{a}
    - \frac{1}{2}\hbar\omega\ket{g}\bra{g}
    + \hbar\omega_0\ket{g}\bra{g}\hat{a}^\dagger\ha
    -\frac{\hbar g^2}{\Delta} \ket{g}\bra{g}\hat{a}^\dagger\hat{a}\nn[10pt]
    %%
    &=\ket{e}\bra{e}\otimes\left\{
    \left(\frac{1}{2}\hbar\omega
    +\frac{\hbar g^2}{\Delta}
    \right)
    +\left(\hbar\omega_0
    +\frac{\hbar g^2}{\Delta} 
    \right)\hat{a}^\dagger\hat{a}
    \right\}
    +\ket{g}\bra{g}\left\{
    -\frac{1}{2}\hbar\omega
    +\left(
    \hbar\omega_0
    -\frac{\hbar g^2}{\Delta}
    \right)\hat{a}^\dagger\hat{a}
    \right\}\nn[10pt]
\end{align}

where we calculate
\begin{align}
    [\hat{S_1}, \hat{H}_2]
    &=\hat{S_1}\hat{H}_2-\hat{H}_2\hat{S_1}\nn[10pt]
    &=\frac{g^2}{\Delta}[(\hsig_+\hat{a}-\hsig_-\hat{a}^\dagger),(\hsig_+\hat{a}+\hsig_-\hat{a}^\dagger)]\nn[10pt]
    &=\frac{g^2}{\Delta}\Bigl\{
    [\hsig_+\hat{a},\hsig_+\hat{a}]
    +[\hsig_+\hat{a},\hsig_-\hat{a}^\dagger]
    -[\hsig_-\hat{a}^\dagger,\hsig_+\hat{a}]
    -[\hsig_-\hat{a}^\dagger,\hsig_-\hat{a}^\dagger]
    \Bigr\}\nn[10pt]
    &=\frac{g^2}{\Delta}\Bigl\{
    [\hsig_+\hat{a},\hsig_-\hat{a}^\dagger]
    -[\hsig_-\hat{a}^\dagger,\hsig_+\hat{a}]
    \Bigr\}\nn[10pt]
    &=\frac{g^2}{\Delta}\Bigl\{
    (\hsig_+\hsig_- \hat{a}\hat{a}^\dagger-\hsig_-\hsig_+\hat{a}^\dagger\hat{a})
    -(
    \hsig_-\hsig_+\hat{a}^\dagger\hat{a}
    -\hsig_+\hsig_-\hat{a}\hat{a}^\dagger
    )
    \Bigr\}\nn[10pt]
    &=\frac{2g^2}{\Delta}(
    \hsig_+\hsig_- \hat{a}\hat{a}^\dagger-\hsig_-\hsig_+\hat{a}^\dagger\hat{a})\nn[10pt]
\end{align}

\begin{align}
    \hsig_+\hsig_- \hat{a}\hat{a}^\dagger-\hsig_-\hsig_+\hat{a}^\dagger\hat{a}
    &= \hsig_+\hsig_-\hat{a}^\dagger\hat{a}
    +\hsig_+\hsig_-
    -\hsig_-\hsig_+\hat{a}^\dagger\hat{a}\nn[10pt]
    &=\hsig_+\hsig_-
    +(\hsig_+\hsig_-
    -\hsig_-\hsig_+)\hat{a}^\dagger\hat{a}\nn[10pt]
    &=\hsig_+\hsig_-
    +\hsig_z\hat{a}^\dagger\hat{a}
\end{align}

