\part{時間に依存しない摂動論}
ここでは,時間に依存しない摂動論の一般論について,解説し,摂動論を使う際に有用な近似公式を導出する.
\section{時間に依存しない摂動論 (time-independent perturbation theory)}
調和振動子や水素原子など,Schr\"{o}dinger方程式を解析的に解ける例は極めて少ない.そこで,近似的にSchr\"{o}dinger方程式を解く必要がある.Schr\"{o}dinger方程式を近似的に解く方法の一つとして,摂動論がある.摂動論は完全に解ける問題に対して,わずかな補正(摂動)を与えた場合を近似的に解く方法である.本書では,時間に依存せず,縮退のない場合の摂動論について論じる.
\subsection{問題設定}
いま,解くべきSchr\"{o}dingerの固有値方程式は
\begin{align}\label{Hsch}
\hat{H}\ket{\varphi_n}=E_n\ket{\varphi_n}\ \ \ n=1,2,\cdots
\end{align}
とする.いま,我々が求めたいのは(\ref{Hsch})のエネルギー固有値$E_n$とその固有状態$\ket{\varphi_n}$であるが,それらの厳密解をえることができないとする.すなわち,(\ref{Hsch})は厳密には解けないとする.
(\ref{Hsch})の与えられた物理系のハミルトニアン$\hat{H}$が
\begin{align}\label{H}
\hat{H}=\hat{H}_0+\lambda\hat{V}
\end{align}
の形をもつとする.すると,(\ref{Hsch})は
\begin{align}\label{Hsch1}
(\hat{H}_0+\lambda\hat{V})\ket{\varphi_n}=E_n\ket{\varphi_n}\ \ \ n=1,2,\cdots
\end{align}
とかける.ここで,$\lambda$は無次元のパラメータで十分に小さい($\lambda\ll1$)とする.$\hat{H}_0$を非摂動ハミルトニアン,$\hat{V}$を摂動ハミルトニアン,または摂動ポテンシャルとよぶ.
ハミルトニアンが(\ref{H})でかかれるとき,$\hat{H}_0$のSchr\"{o}dingerの固有値方程式
\begin{align}\label{H0sch}
\hat{H}_0\ket{\varphi^{(0)}_n}=\epsilon_n\ket{\varphi^{(0)}_n}
\end{align}
は完全に解けるものとし,(\ref{H0sch})の固有状態$\ket{\varphi^{(0)}_n}$の無限個の集合$\left\{\ket{\varphi^{(0)}_n}\right\}_{n=1,2,\cdots}$は正規直交完全系をなすものとする.また,$\epsilon_n$は$\hat{H}_0$のエネルギー固有値であり,ここでは,縮退のある場合を考える.
固有値方程式(\ref{Hsch1})において,次のことを要請する.
 %%%%%
 \begin{request}\label{re1}
$\lambda\to0$のとき$E\to\epsilon_n$となる
\end{request}
つまり,固有値方程式(\ref{Hsch1})の解で要請\ref{re1}をみたすものを求める.











%
\subsection{摂動方程式}
摂動を受けたハミルトニアン(\ref{H})の固有値問題(\ref{Hsch1}),すなわち,
\begin{align}\label{Hsch2}
(\hat{H}_0+\lambda\hat{V})\ket{\varphi_n}=E_n\ket{\varphi_n}\ \ \ n=1,2,\cdots
\end{align}
を解きたい.$\lambda$は十分小さいとするから,摂動の影響は$\lambda$のべき級数に展開して考えることができるだろう.(\ref{Hsch2})のエネルギー固有値$E_n$と固有状態$\ket{\varphi_n}$を$\lambda$でべき級数展開する:
\begin{align}
\label{Ee}
E_n&=\epsilon_n+\lambda E^{(1)}_n+\lambda^2E^{(2)}_n+\cdots\cdots\\[10pt]
\label{pe}
\ket{\varphi_n}&=\ket{\varphi^{(0)}_n}+\lambda \ket{\varphi^{(1)}_n}+\lambda^2\ket{\varphi^{(2)}_n}+\cdots\cdots
\end{align}
ここで,$E^{(0)}_n$に当たるところを$\epsilon_n$としたのは,要請\ref{re1}による.\\
 (\ref{Hsch2})両辺へ(\ref{Ee})と(\ref{pe})を代入すると,
\begin{align}\label{pereq}
\left(\hat{H}_0+\lambda\hat{V}\right)&
\left
(\ket{\varphi^{(0)}_n}+\lambda \ket{\varphi^{(1)}_n}+\lambda^2\ket{\varphi^{(2)}_n}+\cdots\right)\notag\\[5pt]
&=\left(\epsilon_n+\lambda E^{(1)}_n+\lambda^2E^{(2)}_n+\cdots\right)
\left(\ket{\varphi^{(0)}_n}+\lambda \ket{\varphi^{(1)}_n}+\lambda^2\ket{\varphi^{(2)}_n}+\cdots\right)
\end{align}
となる.次にこの式の両辺を$\lambda$のべきで整理して,同じべきの項を等しいとおく.これが摂動論の原理である.これはこの等式が十分小さい$\lambda$に対して常に成り立つという要求にほかならない.\\
 (\ref{pereq})の両辺を$\lambda$のべきで整理し,同べきの項をとりだす:
 \begin{align}
\label{pereq0}
\lambda^0&:&\hat{H}_0\ket{\varphi^{(0)}_n}&=\epsilon_n\ket{\varphi^{(0)}_n}\\[5pt]
%
\label{pereq1}
\lambda^1&:&\hat{H}_0\ket{\varphi^{(1)}_n}+\hat{V}\ket{\varphi^{(0)}_n}
&=\epsilon_n\ket{\varphi^{(1)}_n}+E^{(1)}_n\ket{\varphi^{(0)}_n}\\[5pt]
%
\label{pereq2}
\lambda^2&:&\hat{H}_0\ket{\varphi^{(2)}_n}+\hat{V}\ket{\varphi^{(1)}_n}
&=\epsilon_n\ket{\varphi^{(2)}_n}+E^{(1)}_n\ket{\varphi^{(1)}_n}+E^{(2)}_n\ket{\varphi^{(0)}_n}\\
%
\label{pereq3}
\lambda^3&:&\hat{H}_0\ket{\varphi^{(3)}_n}+\hat{V}\ket{\varphi^{(2)}_n}
&=\epsilon_n\ket{\varphi^{(3)}_n}+E^{(1)}_n\ket{\varphi^{(2)}_n}+E^{(2)}_n\ket{\varphi^{(1)}_n}
+E^{(3)}_n\ket{\varphi^{(0)}_n}\\
& & &\ \ \vdots\nn[5pt]
%
\label{pereqN}
\lambda^N&:&\hat{H}_0\ket{\varphi^{(N)}_n}+\hat{V}\ket{\varphi^{(N-1)}_n}
&=\epsilon_n\ket{\varphi^{(N)}_n}+\sum_{k=1}^{N}E^{(N)}_n\ket{\varphi^{(N-k)}_n}
\end{align}
これを摂動方程式という.




