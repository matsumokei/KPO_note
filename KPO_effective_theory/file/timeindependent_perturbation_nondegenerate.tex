%
\section{縮退のない場合}
\subsection{1次摂動}
\paragraph*{\large{固有エネルギーの1次摂動\\}}
はじめに固有エネルギーの1次摂動$E_{n}^{(1)}$を求める.(\ref{pereq1})を次のように整理する:
\begin{align}\label{eq11}
(E^{(1)}_n-\hat{V})\ket{\varphi^{(0)}_n}=
(\hat{H}_0-\epsilon_n)\ket{\varphi^{(1)}_n}
\end{align}
(\ref{eq11})の両辺に左から,$\bra{\varphi_{n}^{(0)}}$をかけると,
 \begin{align}
 \label{eq12}
\Braket{\varphi_{n}^{(0)}|(E_{n}^{(1)}-\hat{V})|\varphi_{n}^{(0)}}=\Braket{\varphi_{n}^{(0)}|(\hat{H_0}-\epsilon_{n})|\varphi_{n}^{(1)}}
\end{align}
となる.ここで,$\ket{\varphi_{n}^{(1)}}$を正規直交完全系$\left\{\ket{\varphi_n^{(0)}}\right\}_{n=1,2,3,\cdots}$で展開すると
 \begin{align}
 \label{phi1CONS}
\ket{\varphi_{n}^{(1)}}=\displaystyle\sum_{m=1}^\infty C_{m}\ket{\varphi_{m}^{(0)}},\ \ \ \ C_m=\Braket{\varphi_{n}^{(0)}|\varphi_{n}^{(1)}}
\end{align}
と展開できる.(\ref{phi1CONS})を(\ref{eq12})へ代入すると
 \begin{align}
 \label{eq13}
\Braket{\varphi_{n}^{(0)}|(E_{n}^{(1)}-\hat{V})|\varphi_{n}^{(0)}}=\displaystyle\sum_{m=1}^\infty\Braket{\varphi_{n}^{(0)}|(\hat{H_0}-\epsilon_{n})|C_{m}\varphi_{m}^{(0)}}
\end{align}
となる.このとき,(\ref{eq13})の左辺は,
 \begin{align}
\Braket{\varphi_{n}^{(0)}|(E_{n}^{(1)}-\hat{V})|\varphi_{n}^{(0)}}
&=\Braket{\varphi_{n}^{(0)}|E_{n}^{(1)}|\varphi_{n}^{(0)}}-\Braket{\varphi_{n}^{(0)}|\hat{V}|\varphi_{n}^{(0)}}\notag\\[10pt]
&=E_{n}^{(1)}-\Braket{\varphi_{n}^{(0)}|\hat{V}|\varphi_{n}^{(0)}}
\end{align}
となる.また,(\ref{eq13})の右辺は$\hat{H_0}\ket{\varphi_m^{(0)}}=\epsilon_m\ket{\varphi_m^{(0)}}$より,
 \begin{align}
 \label{eq13RHS}
\displaystyle\sum_{m=1}^\infty\Braket{\varphi_{n}^{(0)}|(\hat{H_0}-\epsilon_{n})|C_{m}\varphi_{m}^{(0)}}
&=\displaystyle\sum_{m=1}^\infty\Braket{\varphi_{n}^{(0)}|(\epsilon_{m}-\epsilon_{n})C_{m}|\varphi_{m}^{(0)}}\notag
\\[5pt]
&=\displaystyle\sum_{m=1}^\infty(\epsilon_{m}-\epsilon_{n})C_{m}\Braket{\varphi_{n}^{(0)}|\varphi_{m}^{(0)}}\notag
\\[5pt]
&=\displaystyle\sum_{m=1}^\infty(\epsilon_{m}-\epsilon_{n})C_{m}\delta_{n,m}
\end{align}
となる.ここで$\delta_{n,m}$はクロネッカーのデルタである.(\ref{eq13RHS})の右辺に対して,$m$で和をとると,$m\neq n$の項では$\delta_{n,m}=0$に,$m=n$の項では$(\epsilon_{m}-\epsilon_{n})=0$となる.したがって,(\ref{eq13RHS})は恒等的に$0$となることがわかる.
%
 \begin{align}
\therefore\displaystyle\sum_{m=1}^\infty\Braket{\varphi_{n}^{(0)}|(\hat{H_0}-\epsilon_{n})|C_{m}\varphi_{m}^{(0)}}=0
\end{align}
よって,(\ref{eq13})は
 \begin{align}
\Braket{\varphi_{n}^{(0)}|(E_{n}^{(1)}-\hat{V})|\varphi_{n}^{(0)}}=0
\end{align}
となる.したがって,固有エネルギーの1次の摂動は
%
 \begin{align}\label{parE1}
E_{n}^{(1)}=\Braket{\varphi_{n}^{(0)}|\hat{V}|\varphi_{n}^{(0)}}
\end{align}
と求まる.すでにわかっている固有状態$\ket{\varphi_{n}^{(0)}}$に対する摂動$\hat{V}$の期待値を求めることで$E_{n}^{(1)}$を求めることができる.これは,極めて重要なことである.$\ket{\varphi_{n}^{(0)}}$にとって,摂動$\hat{V}$の影響はどれくらいか,(\ref{parE1})を用いて,調べることができるのである.まとめると,固有エネルギーの1次の近似は
\begin{align}
E_n\simeq E_{n}^{(0)}+\lambda E_{n}^{(1)}=\epsilon_n+\Braket{\varphi_{n}^{(0)}|\lambda\hat{V}|\varphi_{n}^{(0)}}
\end{align}
で与えられる.\\
 




























%
\paragraph*{\large{固有状態の1次摂動\\}}
次に固有状態の1次摂動$\ket{\varphi_{n}^{(1)}}$を求める.(\ref{eq11})の両辺に左から,$\bra{\varphi_{m}^{(0)}}$をかけると,
 \begin{align}
 \label{es12}
\Braket{\varphi_{m}^{(0)}|(E_{n}^{(1)}-\hat{V})|\varphi_{n}^{(0)}}
&=\Braket{\varphi_{m}^{(0)}|(\hat{H_0}-\epsilon_{n})|\varphi_{n}^{(1)}}\notag\\[10pt]
&=\Braket{\varphi_{m}^{(0)}|\hat{H_0}|\varphi_{n}^{(1)}}
-\epsilon_{n}\Braket{\varphi_{m}^{(0)}|\varphi_{n}^{(1)}}
\end{align}
となる.(\ref{es12})の右辺第1項について,共役をとり,$\hat{H}_0$のエルミート性$\hat{H}_0^\dag=\hat{H}_0$を用いると
%
 \begin{align}
\Braket{\varphi_{m}^{(0)}|\hat{H_0}|\varphi_{n}^{(1)}}^{\ast}
=\Braket{\varphi_{n}^{(1)}|\hat{H_0}^\dag|\varphi_{m}^{(0)}}
=\Braket{\varphi_{n}^{(1)}|\hat{H_0}|\varphi_{m}^{(0)}}
=\epsilon_{m}\Braket{\varphi_{n}^{(1)}|\varphi_{m}^{(0)}}
\end{align}
%
そして,もう一度上式に対して共役とると,
 \begin{align}
\Braket{\varphi_{m}^{(0)}|\hat{H_0}|\varphi_{n}^{(1)}}
=\epsilon_{m}^{\ast}\Braket{\varphi_{n}^{(1)}|\varphi_{m}^{(0)}}^{\ast}
=\epsilon_{m}\Braket{\varphi_{m}^{(0)}|\varphi_{n}^{(1)}}
\end{align}
となる.よって,(\ref{es12})は
 \begin{align}
 \label{es13}
\Braket{\varphi_{m}^{(0)}|(E_{n}^{(1)}-\hat{V})|\varphi_{n}^{(0)}}
=(\epsilon_{m}-\epsilon_{n})\Braket{\varphi_{m}^{(0)}|\varphi_{n}^{(1)}}
\end{align}
となる.
%
%
%
ここで再び,$\ket{\varphi_{n}^{(1)}}$を正規直交完全系$\left\{\ket{\varphi_n^{(0)}}\right\}_{n=1,2,3,\cdots}$で展開し,
 \begin{align}
 \label{phi1CONSk}
\ket{\varphi_{n}^{(1)}}=\displaystyle\sum_{k=1}^\infty C_{k}\ket{\varphi_{k}^{(0)}},\ \ \ \ C_k=\Braket{\varphi_{n}^{(0)}|\varphi_{n}^{(1)}}
\end{align}
(\ref{phi1CONSk})を(\ref{es13})へ代入すると
%
 \begin{align}
 \label{es14}
\Braket{\varphi_{m}^{(0)}|(E_{n}^{(1)}-\hat{V})|\varphi_{n}^{(0)}}
=\displaystyle\sum_{k=1}^\infty (\epsilon_{m}-\epsilon_{n})\Braket{\varphi_{m}^{(0)}|C_{k}\varphi_{k}^{(0)}}
\end{align}
となる.このとき,(\ref{es14})の左辺は,
 \begin{align}
\Braket{\varphi_{m}^{(0)}|(E_{n}^{(1)}-\hat{V})|\varphi_{n}^{(0)}}
&=\Braket{\varphi_{m}^{(0)}|E_{n}^{(1)}|\varphi_{n}^{(0)}}-\Braket{\varphi_{m}^{(0)}|\hat{V}|\varphi_{n}^{(0)}}\notag\\[10pt]
&=E_{n}^{(1)}\Braket{\varphi_{m}^{(0)}|\varphi_{n}^{(0)}}-\Braket{\varphi_{m}^{(0)}|\hat{V}|\varphi_{n}^{(0)}}\notag\\[10pt]
&=-\Braket{\varphi_{m}^{(0)}|\hat{V}|\varphi_{n}^{(0)}}
\end{align}
となる.最後の等式で,$\braket{\varphi_{m}^{(0)}|\varphi_{n}^{(0)}}=0$を用いた.また,(\ref{es14})の右辺は$\hat{H_0}\ket{\varphi_m^{(0)}}=\epsilon_m\ket{\varphi_m^{(0)}}$より,
 \begin{align}
 \label{es14RHS}
\displaystyle\sum_{k=1}^\infty (\epsilon_{m}-\epsilon_{n})\Braket{\varphi_{m}^{(0)}|C_{k}\varphi_{k}^{(0)}}
&=\displaystyle\sum_{k=1}^\infty(\epsilon_{m}-\epsilon_{n})C_{k}\Braket{\varphi_{m}^{(0)}|\varphi_{k}^{(0)}}\notag
\\[5pt]
&=\displaystyle\sum_{k=1}^\infty(\epsilon_{m}-\epsilon_{n})C_{k}\delta_{m,k}
\end{align}
となる.ここで$\delta_{m,k}$はクロネッカーのデルタである.(\ref{es14RHS})の右辺に対して,$k$で和をとると,$k=m$の項のみ残る.したがって,(\ref{eq13RHS})は
%
 \begin{align}
\displaystyle\sum_{k=1}^\infty (\epsilon_{m}-\epsilon_{n})\braket{\varphi_{m}^{(0)}|C_{k}\varphi_{k}^{(0)}}
=(\epsilon_{m}-\epsilon_{n})C_{m}
\end{align}
よって,(\ref{es14})は
 \begin{align}
-\Braket{\varphi_{m}^{(0)}|\hat{V}|\varphi_{n}^{(0)}}=(\epsilon_{m}-\epsilon_{n})C_{m}
\end{align}
となる.$m\neq n$のとき,$(\epsilon_{m}-\epsilon_{n})\neq0$であるから,展開係数$C_m$は
%
 \begin{align}\label{parcm}
C_{m}=\frac{\braket{\varphi_{m}^{(0)}|\hat{V}|\varphi_{n}^{(0)}}}{(\epsilon_{n}-\epsilon_{m})}
\end{align}
と求まる.展開係数(\ref{parcm})を用いて,$\ket{\varphi_{n}^{(1)}}$を展開すれば,固有状態の1次摂動
 \begin{align}\label{pars1}
 \ket{\varphi_{n}^{(1)}}
 =\displaystyle\sum_{\substack{m=1 \\ m\neq n}}^\infty C_{m}\ket{\varphi_{m}^{(0)}}=
\displaystyle\sum_{\substack{m=1 \\ m\neq n}}^\infty
 \frac{\braket{\varphi_{m}^{(0)}|\hat{V}|\varphi_{n}^{(0)}}}{(\epsilon_{n}-\epsilon_{m})}
\ket{\varphi_{m}^{(0)}}
\end{align}
と求まる.ここで,$\displaystyle\sum_{\substack{m=1 \\ m\neq n}}^\infty$は$m=n$を除いて,$m$について和をとることを意味する.$m=n$を除き和をとるという条件は展開係数$C_m$を決めるときの$m\neq n$からきている.まとめると,固有状態の1次の近似は
 \begin{align}
\ket{\varphi_{n}}\simeq\ket{\varphi_{n}^{(0)}}+\lambda\ket{\varphi_{n}^{(1)}}=\ket{\varphi_{n}^{(0)}}
+
\displaystyle\sum_{\substack{m=1 \\ m\neq n}}^\infty
 \frac{\braket{\varphi_{m}^{(0)}|\lambda\hat{V}|\varphi_{n}^{(0)}}}{(\epsilon_{n}-\epsilon_{m})}
\ket{\varphi_{m}^{(0)}}
\end{align}
で与えられる.













%
\subsection{2次摂動}
\paragraph*{\large{固有エネルギーの2次摂動\\}}
はじめに固有エネルギーの2次摂動$E_{n}^{(1)}$を求める.(\ref{pereq2})を次のように整理する:
\begin{align}\label{eq21}
(\hat{V}-E^{(1)}_n)\ket{\varphi^{(1)}_n}
=(\epsilon_n-\hat{H}_0)\ket{\varphi^{(2)}_n}+E^{(2)}_n\ket{\varphi^{(0)}_n}
\end{align}
(\ref{eq21})の両辺に左から,$\bra{\varphi_{n}^{(0)}}$をかけると,
 \begin{align}
 \label{eq22}
\Braket{\varphi_{n}^{(0)}|(\hat{V}-E_{n}^{(1)})|\varphi_{n}^{(1)}}
=\Braket{\varphi_{n}^{(0)}|(\epsilon_{n}-\hat{H_0})|\varphi_{n}^{(2)}}+E_n^{(2)}
\end{align}
となる.
%
このとき,(\ref{eq22})の左辺は,
 \begin{align}\label{eq22L}
\Braket{\varphi_{n}^{(0)}|(\hat{V}-E_{n}^{(1)})|\varphi_{n}^{(1)}}
&=\Braket{\varphi_{n}^{(0)}|\hat{V}|\varphi_{n}^{(1)}}-\Braket{\varphi_{n}^{(0)}|E_{n}^{(1)}|\varphi_{n}^{(1)}}
\end{align}
となる.(\ref{eq22L})の第1項に(\ref{pars1})を代入すると
%左辺第1項
 \begin{align}\label{eq22L1}
\Braket{\varphi_{n}^{(0)}|\hat{V}|\varphi_{n}^{(1)}}
&=\bra{\varphi_{n}^{(0)}}\hat{V}
\left(
\displaystyle\sum_{\substack{m=1 \\ m\neq n}}^\infty
 \frac{\braket{\varphi_{m}^{(0)}|\hat{V}|\varphi_{n}^{(0)}}}{(\epsilon_{n}-\epsilon_{m})}
\ket{\varphi_{m}^{(0)}}
\right)\notag\\[10pt]
%
&=\displaystyle\sum_{\substack{m=1 \\ m\neq n}}^\infty
 \frac{
 \braket{\varphi_{m}^{(0)}|\hat{V}|\varphi_{n}^{(0)}}
  \bra{\varphi_{n}^{(0)}}\hat{V}\ket{\varphi_{m}^{(0)}}}{(\epsilon_{n}-\epsilon_{m})}
\end{align}
%
また,(\ref{eq22L})の第2項に(\ref{pars1})を代入すると
%左辺第2項
 \begin{align}
\Braket{\varphi_{n}^{(0)}|E_{n}^{(1)}|\varphi_{n}^{(1)}}
&=E_{n}^{(1)}\bra{\varphi_{n}^{(0)}}
\left(
\displaystyle\sum_{\substack{m=1 \\ m\neq n}}^\infty
 \frac{\braket{\varphi_{m}^{(0)}|\hat{V}|\varphi_{n}^{(0)}}}{(\epsilon_{n}-\epsilon_{m})}
\ket{\varphi_{m}^{(0)}}
\right)\notag\\[10pt]
%
&=E_{n}^{(1)}\displaystyle\sum_{\substack{m=1 \\ m\neq n}}^\infty
 \frac{
 \braket{\varphi_{m}^{(0)}|\hat{V}|\varphi_{n}^{(0)}}}{(\epsilon_{n}-\epsilon_{m})}
   \braket{\varphi_{n}^{(0)}|\varphi_{m}^{(0)}}\notag\\[10pt]
   %
   &=E_{n}^{(1)}\displaystyle\sum_{\substack{m=1 \\ m\neq n}}^\infty
 \frac{
 \braket{\varphi_{m}^{(0)}|\hat{V}|\varphi_{n}^{(0)}}}{(\epsilon_{n}-\epsilon_{m})}
\delta_{n,m}
\end{align}
となる.最後の項で$m=n$は除いて$m$について和をとれば,$\delta_{n,m}=0$となるから,結局
 \begin{align}\label{eq22L2}
\Braket{\varphi_{n}^{(0)}|E_{n}^{(1)}|\varphi_{n}^{(1)}}=0
\end{align}
である.\\
%
%
 次に,(\ref{eq22})の右辺を計算する.$\ket{\varphi_{n}^{(2)}}$を正規直交完全系$\left\{\ket{\varphi_n^{(0)}}\right\}_{n=1,2,3,\cdots}$を用いて展開する.
 \begin{align}
 \label{phi2CONS}
\ket{\varphi_{n}^{(2)}}=\displaystyle\sum_{k=1}^\infty C_{k}^\prime\ket{\varphi_{k}^{(0)}},\ \ \ \ C_k^\prime=\Braket{\varphi_{n}^{(0)}|\varphi_{n}^{(2)}}
\end{align}
(\ref{phi2CONS})を(\ref{eq22})の右辺第1項へ代入すると
 \begin{align}
 \label{eq22R2}
\Braket{\varphi_{n}^{(0)}|(\epsilon_{n}-\hat{H_0})|\varphi_{n}^{(2)}}
&=\Braket{\varphi_{n}^{(0)}|(\epsilon_{n}-\hat{H_0})|\varphi_{n}^{(2)}}\notag\\[10pt]
%
&=\displaystyle\sum_{k=1}^\infty 
\Braket{\varphi_{n}^{(0)}|(\epsilon_{n}-\hat{H_0})|C_{k}^\prime\varphi_{k}^{(0)}}\notag\\[10pt]
%
&=\displaystyle\sum_{k=1}^\infty 
(\epsilon_{n}-\epsilon_{k})C_{k}^\prime
\Braket{\varphi_{n}^{(0)}|\varphi_{k}^{(0)}}\notag\\[10pt]
%%
&=\displaystyle\sum_{k=1}^\infty 
(\epsilon_{n}-\epsilon_{k})C_{k}^\prime
\delta_{n,k}=0
\end{align}
となる.(\ref{eq22L1}),(\ref{eq22L2}),(\ref{eq22R2})より,固有エネルギーの2次の摂動は
\begin{align}
 \label{parE2}
E_n^{(2)}
=\displaystyle\sum_{\substack{m=1 \\ m\neq n}}^\infty
 \frac{
 \braket{\varphi_{m}^{(0)}|\hat{V}|\varphi_{n}^{(0)}}
  \bra{\varphi_{n}^{(0)}}\hat{V}\ket{\varphi_{m}^{(0)}}}{(\epsilon_{n}-\epsilon_{m})}
\end{align}
と求まる.まとめると,固有エネルギーの2次の近似は
\begin{align}
E_n&\simeq E_{n}^{(0)}+\lambda E_{n}^{(1)}+\lambda^2 E_{n}^{(2)}\notag\\[10pt]
&=\epsilon_n+\Braket{\varphi_{n}^{(0)}|\lambda\hat{V}|\varphi_{n}^{(0)}}
+\lambda^2\displaystyle\sum_{\substack{m=1 \\ m\neq n}}^\infty
 \frac{
 \braket{\varphi_{m}^{(0)}|\hat{V}|\varphi_{n}^{(0)}}
  \bra{\varphi_{n}^{(0)}}\hat{V}\ket{\varphi_{m}^{(0)}}}{(\epsilon_{n}-\epsilon_{m})}
\end{align}
で与えられる.



























%
\paragraph*{\large{固有状態の2次摂動\\}}
次に固有状態の2次摂動$\ket{\varphi_{n}^{(2)}}$を求める.(\ref{eq21})の両辺に左から,$\bra{\varphi_{m}^{(0)}}$をかけると,
 \begin{align}
 \label{es22}
\Braket{\varphi_{m}^{(0)}|(\hat{V}-E_{n}^{(1)})|\varphi_{n}^{(1)}}
=\Braket{\varphi_{m}^{(0)}|(\epsilon_{n}-\hat{H_0})|\varphi_{n}^{(2)}}
+E_n^{(2)}\Braket{\varphi_{m}^{(0)}|\varphi_{n}^{(0)}}
\end{align}
となる.
%
%
このとき,(\ref{es22})の左辺は,
 \begin{align}\label{es22L}
\Braket{\varphi_{m}^{(0)}|(\hat{V}-E_{n}^{(1)})|\varphi_{n}^{(1)}}
&=\Braket{\varphi_{m}^{(0)}|\hat{V}|\varphi_{n}^{(1)}}-\Braket{\varphi_{m}^{(0)}|E_{n}^{(1)}|\varphi_{n}^{(1)}}
\end{align}
となる.(\ref{es22L})の第1項に(\ref{pars1})を代入すると(ただし,和をとる記号は$m\to k$と変える.)
%左辺第1項
 \begin{align}\label{es22L1}
\Braket{\varphi_{m}^{(0)}|\hat{V}|\varphi_{n}^{(1)}}
&=\bra{\varphi_{m}^{(0)}}\hat{V}
\left(
\displaystyle\sum_{\substack{k=1 \\ k\neq n}}^\infty
 \frac{\braket{\varphi_{k}^{(0)}|\hat{V}|\varphi_{n}^{(0)}}}{(\epsilon_{n}-\epsilon_{k})}
\ket{\varphi_{k}^{(0)}}
\right)\notag\\[10pt]
%
&=\displaystyle\sum_{\substack{k=1 \\ k\neq n}}^\infty
 \frac{
 \braket{\varphi_{k}^{(0)}|\hat{V}|\varphi_{n}^{(0)}}
  \bra{\varphi_{m}^{(0)}}\hat{V}\ket{\varphi_{k}^{(0)}}}{(\epsilon_{n}-\epsilon_{k})}
\end{align}
%
また,(\ref{es22L})の第2項に(\ref{pars1})を代入すると,
%左辺第2項
 \begin{align}\label{es22L2}
\Braket{\varphi_{m}^{(0)}|E_{n}^{(1)}|\varphi_{n}^{(1)}}
&=E_{n}^{(1)}\bra{\varphi_{m}^{(0)}}
\left(
\displaystyle\sum_{\substack{k=1 \\ k\neq n}}^\infty
 \frac{\braket{\varphi_{k}^{(0)}|\hat{V}|\varphi_{n}^{(0)}}}{(\epsilon_{n}-\epsilon_{k})}
\ket{\varphi_{k}^{(0)}}
\right)\notag\\[10pt]
&=E_{n}^{(1)}\displaystyle\sum_{\substack{k=1 \\ k\neq n}}^\infty
 \frac{
 \braket{\varphi_{k}^{(0)}|\hat{V}|\varphi_{n}^{(0)}}}{(\epsilon_{n}-\epsilon_{k})}
   \braket{\varphi_{m}^{(0)}|\varphi_{k}^{(0)}}\notag\\[10pt]
   %
   &=E_{n}^{(1)}\displaystyle\sum_{\substack{k=1 \\ k\neq n}}^\infty
 \frac{
 \braket{\varphi_{k}^{(0)}|\hat{V}|\varphi_{n}^{(0)}}}{(\epsilon_{n}-\epsilon_{k})}
\delta_{m,k}
%
=E_{n}^{(1)}
 \frac{
 \braket{\varphi_{m}^{(0)}|\hat{V}|\varphi_{n}^{(0)}}}{(\epsilon_{n}-\epsilon_{m})}\notag\\[10pt]
%
&=
 \frac{
 \braket{\varphi_{n}^{(0)}|\hat{V}|\varphi_{n}^{(0)}}\braket{\varphi_{m}^{(0)}|\hat{V}|\varphi_{n}^{(0)}}}{(\epsilon_{n}-\epsilon_{m})}
\end{align}
となる.最後の等式で$E_{n}^{(1)}$へ(\ref{parE1})を代入した.\\
%
%
%
%
 次に,(\ref{es22})の右辺を計算する.$\ket{\varphi_{n}^{(2)}}$を正規直交完全系$\left\{\ket{\varphi_n^{(0)}}\right\}_{n=1,2,3,\cdots}$を用いて展開する.
 \begin{align}
 \label{phi2CONSl}
\ket{\varphi_{n}^{(2)}}=\displaystyle\sum_{k=1}^\infty C_{l}^\prime\ket{\varphi_{l}^{(0)}},\ \ \ \ C_l^\prime=\Braket{\varphi_{n}^{(0)}|\varphi_{n}^{(2)}}
\end{align}
(\ref{phi2CONSl})を(\ref{es22})の右辺第1項へ代入すると
 \begin{align}
 \label{es22R1}
\Braket{\varphi_{m}^{(0)}|(\epsilon_{n}-\hat{H_0})|\varphi_{n}^{(2)}}
%
&=\displaystyle\sum_{l=1}^\infty 
\Braket{\varphi_{m}^{(0)}|(\epsilon_{n}-\hat{H_0})|C_{l}^\prime\varphi_{l}^{(0)}}\notag\\[10pt]
%
&=\displaystyle\sum_{l=1}^\infty 
(\epsilon_{n}-\epsilon_{l})C_{l}^\prime
\Braket{\varphi_{m}^{(0)}|\varphi_{l}^{(0)}}\notag\\[10pt]
%%
&=\displaystyle\sum_{l=1}^\infty 
(\epsilon_{n}-\epsilon_{l})C_{l}^\prime
\delta_{m,l}
=(\epsilon_{n}-\epsilon_{m})C_{m}^\prime
\end{align}
となる.そして,(\ref{es22})右辺第2項に(\ref{parE2})を代入すると,
\begin{align}
 \label{es22R2}
E_n^{(2)}\Braket{\varphi_{m}^{(0)}|\varphi_{n}^{(0)}}
=\displaystyle\sum_{\substack{m=1 \\ m\neq n}}^\infty
 \frac{
 \braket{\varphi_{m}^{(0)}|\hat{V}|\varphi_{n}^{(0)}}
  \bra{\varphi_{n}^{(0)}}\hat{V}\ket{\varphi_{m}^{(0)}}}{(\epsilon_{n}-\epsilon_{m})}
  \delta_{m,n}
  =0
\end{align}
となる.最後の項では,$m=n$を除いて,$m$で和をとったとき,$\delta_{m,n}=0$となることを用いた.
(\ref{es22L1}),(\ref{es22L2}),(\ref{es22R1}),(\ref{es22R2})より,(\ref{es22})は
 \begin{align}
 \label{es23}
 \left(
\displaystyle\sum_{\substack{k=1 \\ k\neq n}}^\infty
 \frac{
 \braket{\varphi_{k}^{(0)}|\hat{V}|\varphi_{n}^{(0)}}
  \bra{\varphi_{m}^{(0)}}\hat{V}\ket{\varphi_{k}^{(0)}}}{(\epsilon_{n}-\epsilon_{k})}
  \right)
  -
   \frac{
 \braket{\varphi_{n}^{(0)}|\hat{V}|\varphi_{n}^{(0)}}\braket{\varphi_{m}^{(0)}|\hat{V}|\varphi_{n}^{(0)}}}{(\epsilon_{n}-\epsilon_{m})}
=(\epsilon_{n}-\epsilon_{m})C_{m}^\prime
\end{align}





となる.$m\neq n$のとき,$(\epsilon_{m}-\epsilon_{n})\neq0$であるから,展開係数$C_m$は
%
 \begin{align}\label{cm2}
C_{m}^\prime=
 \left(
\displaystyle\sum_{\substack{k=1 \\ k\neq n}}^\infty
 \frac{
 \braket{\varphi_{k}^{(0)}|\hat{V}|\varphi_{n}^{(0)}}
  \bra{\varphi_{m}^{(0)}}\hat{V}\ket{\varphi_{k}^{(0)}}}{(\epsilon_{n}-\epsilon_{k})(\epsilon_{n}-\epsilon_{m})}
    \right)
  -
   \frac{
 \braket{\varphi_{n}^{(0)}|\hat{V}|\varphi_{n}^{(0)}}\braket{\varphi_{m}^{(0)}|\hat{V}|\varphi_{n}^{(0)}}}{(\epsilon_{n}-\epsilon_{m})^2}
\end{align}
と求まる.展開係数(\ref{cm2})を用いて,$\ket{\varphi_{n}^{(2)}}$を展開すれば,固有状態の2次摂動
\footnotesize
 \begin{align}\label{pars2}
 \ket{\varphi_{n}^{(2)}}
 &=\displaystyle\sum_{\substack{m=1 \\ m\neq n}}^\infty C_{m}^\prime\ket{\varphi_{m}^{(0)}}\notag\\[10pt]
 &=\displaystyle\sum_{\substack{m=1 \\ m\neq n}}^\infty
 \left[\left(
\displaystyle\sum_{\substack{k=1 \\ k\neq n}}^\infty
 \frac{
 \braket{\varphi_{k}^{(0)}|\hat{V}|\varphi_{n}^{(0)}}
  \bra{\varphi_{m}^{(0)}}\hat{V}\ket{\varphi_{k}^{(0)}}}{(\epsilon_{n}-\epsilon_{k})(\epsilon_{n}-\epsilon_{m})}
    \right)
  -
   \frac{
 \braket{\varphi_{n}^{(0)}|\hat{V}|\varphi_{n}^{(0)}}\braket{\varphi_{m}^{(0)}|\hat{V}|\varphi_{n}^{(0)}}}{(\epsilon_{n}-\epsilon_{m})^2}
  \right]
\ket{\varphi_{m}^{(0)}}
\end{align}
\normalsize
と求まる.まとめると,固有状態の2次の近似は
\begin{align}
\ket{\varphi_{n}}&\simeq\ket{\varphi_{n}^{(0)}}+\lambda\ket{\varphi_{n}^{(1)}}+\lambda^2\ket{\varphi_{n}^{(2)}}\notag\\[10pt]
&=\ket{\varphi_{n}^{(0)}}
+
\displaystyle\sum_{\substack{m=1 \\ m\neq n}}^\infty
 \frac{\braket{\varphi_{n}^{(0)}|\lambda\hat{V}|\varphi_{n}^{(0)}}}{(\epsilon_{n}-\epsilon_{m})}
\ket{\varphi_{m}^{(0)}}\notag\\[10pt]
&+\lambda^2
\displaystyle\sum_{\substack{m=1 \\ m\neq n}}^\infty
 \left[\left(
\displaystyle\sum_{\substack{k=1 \\ k\neq n}}^\infty
 \frac{
 \braket{\varphi_{k}^{(0)}|\hat{V}|\varphi_{n}^{(0)}}
  \bra{\varphi_{m}^{(0)}}\hat{V}\ket{\varphi_{k}^{(0)}}}{(\epsilon_{n}-\epsilon_{k})(\epsilon_{n}-\epsilon_{m})}
    \right)
  -
   \frac{
 \braket{\varphi_{n}^{(0)}|\hat{V}|\varphi_{n}^{(0)}}\braket{\varphi_{m}^{(0)}|\hat{V}|\varphi_{n}^{(0)}}}{(\epsilon_{n}-\epsilon_{m})^2}
  \right]
\ket{\varphi_{m}^{(0)}}
\end{align}
で与えられる.


%
\subsection{摂動論が有効なための条件}
摂動論が有効なためには,摂動によって生まれる補正が小さいものでなければならない.その場合には,1次の摂動論の計算で十分である.しかし,固有エネルギー,または固有状態の1次の摂動を計算したときに,たまたま$0$となったときには,2次の摂動を計算する必要がある.\\
 状態$\ket{\varphi_n}$の1次の摂動が小さいための条件は,(\ref{pars1})から,
  \begin{align}\label{cond1}
 \frac{\braket{\varphi_{m}^{(0)}|\hat{V}|\varphi_{n}^{(0)}}}{(\epsilon_{n}-\epsilon_{m})}&\ll1\notag\\[10pt]
 %
\therefore
\braket{\varphi_{m}^{(0)}|\hat{V}|\varphi_{n}^{(0)}}&\ll(\epsilon_{n}-\epsilon_{m})
\end{align}
がすべての$m\neq n$に対して成り立つことである.つまり,摂動$\hat{V}$の行列要素$\braket{\varphi_{m}^{(0)}|\hat{V}|\varphi_{n}^{(0)}}$に比べて,非摂動固有エネルギーの準位の差が十分大きくなければならない.











